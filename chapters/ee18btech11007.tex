\begin{enumerate}[label=\thesection.\arabic*.,ref=\thesection.\theenumi]
\numberwithin{equation}{enumi}
\item The open loop transfer function of a unity feedback system is given by
\begin{align}
\label{eq:ee18btech11007_system}
 G(s)=\frac{\pi e^{-0.25s}}{s}
\end{align}
\item Find $\text{Re} \cbrak{G(\j \omega)}$ and $\text{Im} \cbrak{G(\j \omega)}$.
\\
\solution From \eqref{eq:ee18btech11007_system},
%
\begin{align}
G(j\omega)&=\frac{\pi}{\omega}(-\sin{0.25\omega}-j\cos{0.25\omega})
\\
\implies  \text{Re} \cbrak{G(\j \omega)}&=\frac{\pi}{\omega}(-\sin{0.25\omega}) 
\\
 \text{Im} \cbrak{G(\j \omega)}&=\frac{\pi}{\omega}(-j\cos{0.25\omega}) 
\end{align}
%
\item Sketch the Nyquist plot.
\\
\solution The Nyquist plot is a graph of $\text{Re} \cbrak{G(\j \omega)}$  vs $\text{Im} \cbrak{G(\j \omega)}$.
The following python code generates the Nyquist plot in Fig.  \ref{fig:ee18btech11007}
%
\begin{figure}[!h]
  \includegraphics[width=\columnwidth]{./figs/ee18btech11007.eps}
  \caption{}
  \label{fig:ee18btech11007}
\end{figure}
%
\item Find the point at which the Nyquist plot of G(s) passes through the negative real axis
\\
\solution  Nyquist plot cuts the negative real axis at $\omega $ for which 
\begin{align}
\angle G(\j\omega)=-\pi
\label{eq:ee18btech11007_system_neg_real}
\end{align}
From \eqref{eq:ee18btech11007_system},
\begin{align}
 G(\j\omega)&=\frac{\pi e^{-\frac{\j\omega}{4}}}{\j\omega} = \frac{\pi e^{-\j\brak{\frac{\omega}{4}+\frac{\pi}{2}}}}{\omega}
\\
\implies \angle{ G(\j\omega)} &= -\brak{\frac{\omega}{4}+\frac{\pi}{2}}
\label{eq:ee18btech11007_system_ang}
\end{align}
From \eqref{eq:ee18btech11007_system_ang} and \eqref{eq:ee18btech11007_system_neg_real}, 
\begin{align}
\frac{\omega}{4}+\frac{\pi}{2} &= \pi
\\
\implies \omega = 2\pi
\end{align}
Also, from \eqref{eq:ee18btech11007_system},
\begin{align}
\label{eq:ee18btech11007_system_mod}
\abs{ G(\j\omega)}&=\frac{\pi }{\abs{\omega}}
\\
\implies \abs{ G(\j2\pi)} &= \frac{1}{2}
\end{align}
%
%\item Find the value of $P$ defined in Table \ref{table:ee18btech11007} from Fig.  \ref{fig:ee18btech11007}.

%
%\solution $P = 0$.
%\item Find the value of $N$ defined in Table \ref{table:ee18btech11007} from  \eqref{eq:ee18btech11007_system}
%\\
%\solution $\because H(s) = 1$, $G(s)H(s) = G(s)$. Also, $G(s)$ has a pole at $s = 0$, hence $N = 0$.
\item Use the Nyquist Stability criterion to determine if the system in \eqref{eq:ee18btech11007_system_ang} is stable.
\begin{table}[!ht]
\centering
\begin{enumerate}[label=\thesection.\arabic*.,ref=\thesection.\theenumi]
\numberwithin{equation}{enumi}
\item The open loop transfer function of a unity feedback system is given by
\begin{align}
\label{eq:ee18btech11007_system}
 G(s)=\frac{\pi e^{-0.25s}}{s}
\end{align}
\item Find $\text{Re} \cbrak{G(\j \omega)}$ and $\text{Im} \cbrak{G(\j \omega)}$.
\\
\solution From \eqref{eq:ee18btech11007_system},
%
\begin{align}
G(j\omega)&=\frac{\pi}{\omega}(-\sin{0.25\omega}-j\cos{0.25\omega})
\\
\implies  \text{Re} \cbrak{G(\j \omega)}&=\frac{\pi}{\omega}(-\sin{0.25\omega}) 
\\
 \text{Im} \cbrak{G(\j \omega)}&=\frac{\pi}{\omega}(-j\cos{0.25\omega}) 
\end{align}
%
\item Sketch the Nyquist plot.
\\
\solution The Nyquist plot is a graph of $\text{Re} \cbrak{G(\j \omega)}$  vs $\text{Im} \cbrak{G(\j \omega)}$.
The following python code generates the Nyquist plot in Fig.  \ref{fig:ee18btech11007}
%
\begin{figure}[!h]
  \includegraphics[width=\columnwidth]{./figs/ee18btech11007.eps}
  \caption{}
  \label{fig:ee18btech11007}
\end{figure}
%
\item Find the point at which the Nyquist plot of G(s) passes through the negative real axis
\\
\solution  Nyquist plot cuts the negative real axis at $\omega $ for which 
\begin{align}
\angle G(\j\omega)=-\pi
\label{eq:ee18btech11007_system_neg_real}
\end{align}
From \eqref{eq:ee18btech11007_system},
\begin{align}
 G(\j\omega)&=\frac{\pi e^{-\frac{\j\omega}{4}}}{\j\omega} = \frac{\pi e^{-\j\brak{\frac{\omega}{4}+\frac{\pi}{2}}}}{\omega}
\\
\implies \angle{ G(\j\omega)} &= -\brak{\frac{\omega}{4}+\frac{\pi}{2}}
\label{eq:ee18btech11007_system_ang}
\end{align}
From \eqref{eq:ee18btech11007_system_ang} and \eqref{eq:ee18btech11007_system_neg_real}, 
\begin{align}
\frac{\omega}{4}+\frac{\pi}{2} &= \pi
\\
\implies \omega = 2\pi
\end{align}
Also, from \eqref{eq:ee18btech11007_system},
\begin{align}
\label{eq:ee18btech11007_system_mod}
\abs{ G(\j\omega)}&=\frac{\pi }{\abs{\omega}}
\\
\implies \abs{ G(\j2\pi)} &= \frac{1}{2}
\end{align}
%
%\item Find the value of $P$ defined in Table \ref{table:ee18btech11007} from Fig.  \ref{fig:ee18btech11007}.

%
%\solution $P = 0$.
%\item Find the value of $N$ defined in Table \ref{table:ee18btech11007} from  \eqref{eq:ee18btech11007_system}
%\\
%\solution $\because H(s) = 1$, $G(s)H(s) = G(s)$. Also, $G(s)$ has a pole at $s = 0$, hence $N = 0$.
\item Use the Nyquist Stability criterion to determine if the system in \eqref{eq:ee18btech11007_system_ang} is stable.
\begin{table}[!ht]
\centering
\begin{enumerate}[label=\thesection.\arabic*.,ref=\thesection.\theenumi]
\numberwithin{equation}{enumi}
\item The open loop transfer function of a unity feedback system is given by
\begin{align}
\label{eq:ee18btech11007_system}
 G(s)=\frac{\pi e^{-0.25s}}{s}
\end{align}
\item Find $\text{Re} \cbrak{G(\j \omega)}$ and $\text{Im} \cbrak{G(\j \omega)}$.
\\
\solution From \eqref{eq:ee18btech11007_system},
%
\begin{align}
G(j\omega)&=\frac{\pi}{\omega}(-\sin{0.25\omega}-j\cos{0.25\omega})
\\
\implies  \text{Re} \cbrak{G(\j \omega)}&=\frac{\pi}{\omega}(-\sin{0.25\omega}) 
\\
 \text{Im} \cbrak{G(\j \omega)}&=\frac{\pi}{\omega}(-j\cos{0.25\omega}) 
\end{align}
%
\item Sketch the Nyquist plot.
\\
\solution The Nyquist plot is a graph of $\text{Re} \cbrak{G(\j \omega)}$  vs $\text{Im} \cbrak{G(\j \omega)}$.
The following python code generates the Nyquist plot in Fig.  \ref{fig:ee18btech11007}
%
\begin{figure}[!h]
  \includegraphics[width=\columnwidth]{./figs/ee18btech11007.eps}
  \caption{}
  \label{fig:ee18btech11007}
\end{figure}
%
\item Find the point at which the Nyquist plot of G(s) passes through the negative real axis
\\
\solution  Nyquist plot cuts the negative real axis at $\omega $ for which 
\begin{align}
\angle G(\j\omega)=-\pi
\label{eq:ee18btech11007_system_neg_real}
\end{align}
From \eqref{eq:ee18btech11007_system},
\begin{align}
 G(\j\omega)&=\frac{\pi e^{-\frac{\j\omega}{4}}}{\j\omega} = \frac{\pi e^{-\j\brak{\frac{\omega}{4}+\frac{\pi}{2}}}}{\omega}
\\
\implies \angle{ G(\j\omega)} &= -\brak{\frac{\omega}{4}+\frac{\pi}{2}}
\label{eq:ee18btech11007_system_ang}
\end{align}
From \eqref{eq:ee18btech11007_system_ang} and \eqref{eq:ee18btech11007_system_neg_real}, 
\begin{align}
\frac{\omega}{4}+\frac{\pi}{2} &= \pi
\\
\implies \omega = 2\pi
\end{align}
Also, from \eqref{eq:ee18btech11007_system},
\begin{align}
\label{eq:ee18btech11007_system_mod}
\abs{ G(\j\omega)}&=\frac{\pi }{\abs{\omega}}
\\
\implies \abs{ G(\j2\pi)} &= \frac{1}{2}
\end{align}
%
%\item Find the value of $P$ defined in Table \ref{table:ee18btech11007} from Fig.  \ref{fig:ee18btech11007}.

%
%\solution $P = 0$.
%\item Find the value of $N$ defined in Table \ref{table:ee18btech11007} from  \eqref{eq:ee18btech11007_system}
%\\
%\solution $\because H(s) = 1$, $G(s)H(s) = G(s)$. Also, $G(s)$ has a pole at $s = 0$, hence $N = 0$.
\item Use the Nyquist Stability criterion to determine if the system in \eqref{eq:ee18btech11007_system_ang} is stable.
\begin{table}[!ht]
\centering
\begin{enumerate}[label=\thesection.\arabic*.,ref=\thesection.\theenumi]
\numberwithin{equation}{enumi}
\item The open loop transfer function of a unity feedback system is given by
\begin{align}
\label{eq:ee18btech11007_system}
 G(s)=\frac{\pi e^{-0.25s}}{s}
\end{align}
\item Find $\text{Re} \cbrak{G(\j \omega)}$ and $\text{Im} \cbrak{G(\j \omega)}$.
\\
\solution From \eqref{eq:ee18btech11007_system},
%
\begin{align}
G(j\omega)&=\frac{\pi}{\omega}(-\sin{0.25\omega}-j\cos{0.25\omega})
\\
\implies  \text{Re} \cbrak{G(\j \omega)}&=\frac{\pi}{\omega}(-\sin{0.25\omega}) 
\\
 \text{Im} \cbrak{G(\j \omega)}&=\frac{\pi}{\omega}(-j\cos{0.25\omega}) 
\end{align}
%
\item Sketch the Nyquist plot.
\\
\solution The Nyquist plot is a graph of $\text{Re} \cbrak{G(\j \omega)}$  vs $\text{Im} \cbrak{G(\j \omega)}$.
The following python code generates the Nyquist plot in Fig.  \ref{fig:ee18btech11007}
%
\begin{figure}[!h]
  \includegraphics[width=\columnwidth]{./figs/ee18btech11007.eps}
  \caption{}
  \label{fig:ee18btech11007}
\end{figure}
%
\item Find the point at which the Nyquist plot of G(s) passes through the negative real axis
\\
\solution  Nyquist plot cuts the negative real axis at $\omega $ for which 
\begin{align}
\angle G(\j\omega)=-\pi
\label{eq:ee18btech11007_system_neg_real}
\end{align}
From \eqref{eq:ee18btech11007_system},
\begin{align}
 G(\j\omega)&=\frac{\pi e^{-\frac{\j\omega}{4}}}{\j\omega} = \frac{\pi e^{-\j\brak{\frac{\omega}{4}+\frac{\pi}{2}}}}{\omega}
\\
\implies \angle{ G(\j\omega)} &= -\brak{\frac{\omega}{4}+\frac{\pi}{2}}
\label{eq:ee18btech11007_system_ang}
\end{align}
From \eqref{eq:ee18btech11007_system_ang} and \eqref{eq:ee18btech11007_system_neg_real}, 
\begin{align}
\frac{\omega}{4}+\frac{\pi}{2} &= \pi
\\
\implies \omega = 2\pi
\end{align}
Also, from \eqref{eq:ee18btech11007_system},
\begin{align}
\label{eq:ee18btech11007_system_mod}
\abs{ G(\j\omega)}&=\frac{\pi }{\abs{\omega}}
\\
\implies \abs{ G(\j2\pi)} &= \frac{1}{2}
\end{align}
%
%\item Find the value of $P$ defined in Table \ref{table:ee18btech11007} from Fig.  \ref{fig:ee18btech11007}.

%
%\solution $P = 0$.
%\item Find the value of $N$ defined in Table \ref{table:ee18btech11007} from  \eqref{eq:ee18btech11007_system}
%\\
%\solution $\because H(s) = 1$, $G(s)H(s) = G(s)$. Also, $G(s)$ has a pole at $s = 0$, hence $N = 0$.
\item Use the Nyquist Stability criterion to determine if the system in \eqref{eq:ee18btech11007_system_ang} is stable.
\begin{table}[!ht]
\centering
\input{./tables/ee18btech11007.tex}
\caption{}
\label{table:ee18btech11007}
\end{table}
\\
\solution Consider Table \ref{table:ee18btech11007}.  According to the Nyquist stability criterion, 
\begin{enumerate}
\item If the open-loop transfer function $G(s)$ has a zero pole of multiplicity $l$, then the Nyquist plot has a discontinuity at $\omega$ =0. During further analysis it should be assumed that the phasor travels l times clock-wise along a semicircle of infinite radius. After applying this rule, the zero poles should be neglected, i.e. if there are no other unstable poles, then the open-loop transfer function $G(s)$ should be considered stable.
\item If the open-loop transfer function $G(s)$ is stable, then the closed-loop system is unstable for any encirclement of the point -1.
If the open-loop transfer function $G(s)$ is unstable, then there must be one counter clock-wise encirclement of -1 for each pole of $G(s)$ in the right-half of the complex plane.
\label{them:ee18btech11007_nyquist3}
\item The number of surplus encirclements (N + P greater than 0) is exactly the number of unstable poles of the closed-loop system.
\item However, if the graph happens to pass through the point $-1+\j0$, then deciding upon even the marginal stability of the system becomes difficult and the only conclusion that can be drawn from the graph is that there exist zeros on the $\j \omega$  axis.
\end{enumerate}
From \eqref{eq:ee18btech11007_system}, $G(s)$ is stable since it has a single pole at $s = 0$.  Further,  from Fig.  \ref{fig:ee18btech11007}, the Nyquist plot doesnot encircle $s =  -1$.  From  Theorem \ref{them:ee18btech11007_nyquist3}, we may conclude that the system is stable.


\end{enumerate}

\caption{}
\label{table:ee18btech11007}
\end{table}
\\
\solution Consider Table \ref{table:ee18btech11007}.  According to the Nyquist stability criterion, 
\begin{enumerate}
\item If the open-loop transfer function $G(s)$ has a zero pole of multiplicity $l$, then the Nyquist plot has a discontinuity at $\omega$ =0. During further analysis it should be assumed that the phasor travels l times clock-wise along a semicircle of infinite radius. After applying this rule, the zero poles should be neglected, i.e. if there are no other unstable poles, then the open-loop transfer function $G(s)$ should be considered stable.
\item If the open-loop transfer function $G(s)$ is stable, then the closed-loop system is unstable for any encirclement of the point -1.
If the open-loop transfer function $G(s)$ is unstable, then there must be one counter clock-wise encirclement of -1 for each pole of $G(s)$ in the right-half of the complex plane.
\label{them:ee18btech11007_nyquist3}
\item The number of surplus encirclements (N + P greater than 0) is exactly the number of unstable poles of the closed-loop system.
\item However, if the graph happens to pass through the point $-1+\j0$, then deciding upon even the marginal stability of the system becomes difficult and the only conclusion that can be drawn from the graph is that there exist zeros on the $\j \omega$  axis.
\end{enumerate}
From \eqref{eq:ee18btech11007_system}, $G(s)$ is stable since it has a single pole at $s = 0$.  Further,  from Fig.  \ref{fig:ee18btech11007}, the Nyquist plot doesnot encircle $s =  -1$.  From  Theorem \ref{them:ee18btech11007_nyquist3}, we may conclude that the system is stable.


\end{enumerate}

\caption{}
\label{table:ee18btech11007}
\end{table}
\\
\solution Consider Table \ref{table:ee18btech11007}.  According to the Nyquist stability criterion, 
\begin{enumerate}
\item If the open-loop transfer function $G(s)$ has a zero pole of multiplicity $l$, then the Nyquist plot has a discontinuity at $\omega$ =0. During further analysis it should be assumed that the phasor travels l times clock-wise along a semicircle of infinite radius. After applying this rule, the zero poles should be neglected, i.e. if there are no other unstable poles, then the open-loop transfer function $G(s)$ should be considered stable.
\item If the open-loop transfer function $G(s)$ is stable, then the closed-loop system is unstable for any encirclement of the point -1.
If the open-loop transfer function $G(s)$ is unstable, then there must be one counter clock-wise encirclement of -1 for each pole of $G(s)$ in the right-half of the complex plane.
\label{them:ee18btech11007_nyquist3}
\item The number of surplus encirclements (N + P greater than 0) is exactly the number of unstable poles of the closed-loop system.
\item However, if the graph happens to pass through the point $-1+\j0$, then deciding upon even the marginal stability of the system becomes difficult and the only conclusion that can be drawn from the graph is that there exist zeros on the $\j \omega$  axis.
\end{enumerate}
From \eqref{eq:ee18btech11007_system}, $G(s)$ is stable since it has a single pole at $s = 0$.  Further,  from Fig.  \ref{fig:ee18btech11007}, the Nyquist plot doesnot encircle $s =  -1$.  From  Theorem \ref{them:ee18btech11007_nyquist3}, we may conclude that the system is stable.


\end{enumerate}

\caption{}
\label{table:ee18btech11007}
\end{table}
\\
\solution Consider Table \ref{table:ee18btech11007}.  According to the Nyquist stability criterion, 
\begin{enumerate}
\item If the open-loop transfer function $G(s)$ has a zero pole of multiplicity $l$, then the Nyquist plot has a discontinuity at $\omega$ =0. During further analysis it should be assumed that the phasor travels l times clock-wise along a semicircle of infinite radius. After applying this rule, the zero poles should be neglected, i.e. if there are no other unstable poles, then the open-loop transfer function $G(s)$ should be considered stable.
\item If the open-loop transfer function $G(s)$ is stable, then the closed-loop system is unstable for any encirclement of the point -1.
If the open-loop transfer function $G(s)$ is unstable, then there must be one counter clock-wise encirclement of -1 for each pole of $G(s)$ in the right-half of the complex plane.
\label{them:ee18btech11007_nyquist3}
\item The number of surplus encirclements (N + P greater than 0) is exactly the number of unstable poles of the closed-loop system.
\item However, if the graph happens to pass through the point $-1+\j0$, then deciding upon even the marginal stability of the system becomes difficult and the only conclusion that can be drawn from the graph is that there exist zeros on the $\j \omega$  axis.
\end{enumerate}
From \eqref{eq:ee18btech11007_system}, $G(s)$ is stable since it has a single pole at $s = 0$.  Further,  from Fig.  \ref{fig:ee18btech11007}, the Nyquist plot doesnot encircle $s =  -1$.  From  Theorem \ref{them:ee18btech11007_nyquist3}, we may conclude that the system is stable.


\end{enumerate}
