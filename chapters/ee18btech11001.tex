\begin{enumerate}[label=\thesection.\arabic*.,ref=\thesection.\theenumi]
\numberwithin{equation}{enumi}

\item For an LTI system, the Bode plot for its gain defined as
\begin{align}
	G(s) = 20\log\abs{H(s)}
	\label{eq:ee18btech11001_gain}
\end{align}
is as illustrated in the Fig. \ref{fig:ee18btech11001_bode}. Express $G(f)$ in terms of $f$.\\
\begin{figure}[ht!]
    \includegraphics[width=\columnwidth]{./figs/ee18btech11001/ee18btech11001.eps}
    \caption{}
    \label{fig:ee18btech11001_bode}
\end{figure}\\

\solution
\begin{align}
 G(f) = 
 \begin{cases} 
        100 & 0 < f < 10^{1} \\
      120-20\log(f) & 10 < f < 10^{2} \\
      200-60\log(f) & 10^2 < f < 10^{3} \\
      140-40\log(f) & 10^{3} < f < 10^{4} \\
       -20 & 10^{4} < f < 10^{5} \\
      180-40\log(f) & 10^{5} < f < 10^{6} \\
      300-60\log(f) & 10^{6} < f < 10^{7}   
 \end{cases}
\end{align}

%-----------------------------------------------------------------------%

\item Express the slope of $G(f)$ in terms of $f$.
\\
\solution The desired slope is 
\begin{align}
\nabla G(f) &= \dfrac{d(G(f))}{d(\log(f))}
\end{align}

\begin{align}
 \nabla G(f) = 
 \begin{cases} 
        0 & 0 < f < 10^{1} \\
      -20 & 10 < f < 10^{2} \\
      -60 & 10^{2} < f < 10^{3} \\
      -40 & 10^{3} < f < 10^{4} \\
       0 & 10^{4} < f < 10^{5} \\
      -40 & 10^{5} < f < 10^{6} \\
      -60 & 10^{6} < f < 10^{7}   
 \end{cases}
\end{align}

%-----------------------------------------------------------------------%

\item Express the change of slope of $G(f)$ in terms of $f$.
\\
\solution\\
$\Delta(\nabla G(f))$  = Change of slope G(f) at f

\begin{align}
 \Delta(\nabla G(f)) = 
 \begin{cases} 
      -20 &  f = 10^{1} \\
      -40 &  f = 10^{2} \\
      +20 &  f = 10^{3} \\
      +40 &  f = 10^{4} \\
      -40 &  f = 10^{5} \\
      -20 &  f = 10^{6} 
 \end{cases}
\label{eq:ee18btech11001_slope_diff}
\end{align}

%-----------------------------------------------------------------------%

\item Tabulate the poles and zeros of $H(s)$ using \eqref{eq:ee18btech11001_slope_diff}.
\\
\solution Table \ref{table:ee18btech11001} provides the details.  
%
\begin{table}[!ht]
\centering
\begin{enumerate}[label=\thesection.\arabic*.,ref=\thesection.\theenumi]
\numberwithin{equation}{enumi}

\item For an LTI system, the Bode plot for its gain defined as
\begin{align}
	G(s) = 20\log\abs{H(s)}
	\label{eq:ee18btech11001_gain}
\end{align}
is as illustrated in the Fig. \ref{fig:ee18btech11001_bode}. Express $G(f)$ in terms of $f$.\\
\begin{figure}[ht!]
    \includegraphics[width=\columnwidth]{./figs/ee18btech11001/ee18btech11001.eps}
    \caption{}
    \label{fig:ee18btech11001_bode}
\end{figure}\\

\solution
\begin{align}
 G(f) = 
 \begin{cases} 
        100 & 0 < f < 10^{1} \\
      120-20\log(f) & 10 < f < 10^{2} \\
      200-60\log(f) & 10^2 < f < 10^{3} \\
      140-40\log(f) & 10^{3} < f < 10^{4} \\
       -20 & 10^{4} < f < 10^{5} \\
      180-40\log(f) & 10^{5} < f < 10^{6} \\
      300-60\log(f) & 10^{6} < f < 10^{7}   
 \end{cases}
\end{align}

%-----------------------------------------------------------------------%

\item Express the slope of $G(f)$ in terms of $f$.
\\
\solution The desired slope is 
\begin{align}
\nabla G(f) &= \dfrac{d(G(f))}{d(\log(f))}
\end{align}

\begin{align}
 \nabla G(f) = 
 \begin{cases} 
        0 & 0 < f < 10^{1} \\
      -20 & 10 < f < 10^{2} \\
      -60 & 10^{2} < f < 10^{3} \\
      -40 & 10^{3} < f < 10^{4} \\
       0 & 10^{4} < f < 10^{5} \\
      -40 & 10^{5} < f < 10^{6} \\
      -60 & 10^{6} < f < 10^{7}   
 \end{cases}
\end{align}

%-----------------------------------------------------------------------%

\item Express the change of slope of $G(f)$ in terms of $f$.
\\
\solution\\
$\Delta(\nabla G(f))$  = Change of slope G(f) at f

\begin{align}
 \Delta(\nabla G(f)) = 
 \begin{cases} 
      -20 &  f = 10^{1} \\
      -40 &  f = 10^{2} \\
      +20 &  f = 10^{3} \\
      +40 &  f = 10^{4} \\
      -40 &  f = 10^{5} \\
      -20 &  f = 10^{6} 
 \end{cases}
\label{eq:ee18btech11001_slope_diff}
\end{align}

%-----------------------------------------------------------------------%

\item Tabulate the poles and zeros of $H(s)$ using \eqref{eq:ee18btech11001_slope_diff}.
\\
\solution Table \ref{table:ee18btech11001} provides the details.  
%
\begin{table}[!ht]
\centering
\begin{enumerate}[label=\thesection.\arabic*.,ref=\thesection.\theenumi]
\numberwithin{equation}{enumi}

\item For an LTI system, the Bode plot for its gain defined as
\begin{align}
	G(s) = 20\log\abs{H(s)}
	\label{eq:ee18btech11001_gain}
\end{align}
is as illustrated in the Fig. \ref{fig:ee18btech11001_bode}. Express $G(f)$ in terms of $f$.\\
\begin{figure}[ht!]
    \includegraphics[width=\columnwidth]{./figs/ee18btech11001/ee18btech11001.eps}
    \caption{}
    \label{fig:ee18btech11001_bode}
\end{figure}\\

\solution
\begin{align}
 G(f) = 
 \begin{cases} 
        100 & 0 < f < 10^{1} \\
      120-20\log(f) & 10 < f < 10^{2} \\
      200-60\log(f) & 10^2 < f < 10^{3} \\
      140-40\log(f) & 10^{3} < f < 10^{4} \\
       -20 & 10^{4} < f < 10^{5} \\
      180-40\log(f) & 10^{5} < f < 10^{6} \\
      300-60\log(f) & 10^{6} < f < 10^{7}   
 \end{cases}
\end{align}

%-----------------------------------------------------------------------%

\item Express the slope of $G(f)$ in terms of $f$.
\\
\solution The desired slope is 
\begin{align}
\nabla G(f) &= \dfrac{d(G(f))}{d(\log(f))}
\end{align}

\begin{align}
 \nabla G(f) = 
 \begin{cases} 
        0 & 0 < f < 10^{1} \\
      -20 & 10 < f < 10^{2} \\
      -60 & 10^{2} < f < 10^{3} \\
      -40 & 10^{3} < f < 10^{4} \\
       0 & 10^{4} < f < 10^{5} \\
      -40 & 10^{5} < f < 10^{6} \\
      -60 & 10^{6} < f < 10^{7}   
 \end{cases}
\end{align}

%-----------------------------------------------------------------------%

\item Express the change of slope of $G(f)$ in terms of $f$.
\\
\solution\\
$\Delta(\nabla G(f))$  = Change of slope G(f) at f

\begin{align}
 \Delta(\nabla G(f)) = 
 \begin{cases} 
      -20 &  f = 10^{1} \\
      -40 &  f = 10^{2} \\
      +20 &  f = 10^{3} \\
      +40 &  f = 10^{4} \\
      -40 &  f = 10^{5} \\
      -20 &  f = 10^{6} 
 \end{cases}
\label{eq:ee18btech11001_slope_diff}
\end{align}

%-----------------------------------------------------------------------%

\item Tabulate the poles and zeros of $H(s)$ using \eqref{eq:ee18btech11001_slope_diff}.
\\
\solution Table \ref{table:ee18btech11001} provides the details.  
%
\begin{table}[!ht]
\centering
\begin{enumerate}[label=\thesection.\arabic*.,ref=\thesection.\theenumi]
\numberwithin{equation}{enumi}

\item For an LTI system, the Bode plot for its gain defined as
\begin{align}
	G(s) = 20\log\abs{H(s)}
	\label{eq:ee18btech11001_gain}
\end{align}
is as illustrated in the Fig. \ref{fig:ee18btech11001_bode}. Express $G(f)$ in terms of $f$.\\
\begin{figure}[ht!]
    \includegraphics[width=\columnwidth]{./figs/ee18btech11001/ee18btech11001.eps}
    \caption{}
    \label{fig:ee18btech11001_bode}
\end{figure}\\

\solution
\begin{align}
 G(f) = 
 \begin{cases} 
        100 & 0 < f < 10^{1} \\
      120-20\log(f) & 10 < f < 10^{2} \\
      200-60\log(f) & 10^2 < f < 10^{3} \\
      140-40\log(f) & 10^{3} < f < 10^{4} \\
       -20 & 10^{4} < f < 10^{5} \\
      180-40\log(f) & 10^{5} < f < 10^{6} \\
      300-60\log(f) & 10^{6} < f < 10^{7}   
 \end{cases}
\end{align}

%-----------------------------------------------------------------------%

\item Express the slope of $G(f)$ in terms of $f$.
\\
\solution The desired slope is 
\begin{align}
\nabla G(f) &= \dfrac{d(G(f))}{d(\log(f))}
\end{align}

\begin{align}
 \nabla G(f) = 
 \begin{cases} 
        0 & 0 < f < 10^{1} \\
      -20 & 10 < f < 10^{2} \\
      -60 & 10^{2} < f < 10^{3} \\
      -40 & 10^{3} < f < 10^{4} \\
       0 & 10^{4} < f < 10^{5} \\
      -40 & 10^{5} < f < 10^{6} \\
      -60 & 10^{6} < f < 10^{7}   
 \end{cases}
\end{align}

%-----------------------------------------------------------------------%

\item Express the change of slope of $G(f)$ in terms of $f$.
\\
\solution\\
$\Delta(\nabla G(f))$  = Change of slope G(f) at f

\begin{align}
 \Delta(\nabla G(f)) = 
 \begin{cases} 
      -20 &  f = 10^{1} \\
      -40 &  f = 10^{2} \\
      +20 &  f = 10^{3} \\
      +40 &  f = 10^{4} \\
      -40 &  f = 10^{5} \\
      -20 &  f = 10^{6} 
 \end{cases}
\label{eq:ee18btech11001_slope_diff}
\end{align}

%-----------------------------------------------------------------------%

\item Tabulate the poles and zeros of $H(s)$ using \eqref{eq:ee18btech11001_slope_diff}.
\\
\solution Table \ref{table:ee18btech11001} provides the details.  
%
\begin{table}[!ht]
\centering
\input{./tables/ee18btech11001.tex}
\caption{}
\label{table:ee18btech11001}
\end{table}


%-----------------------------------------------------------------------%

\item Obtain the transfer function of $H(s)$.
\\
\solution From Table \ref{table:ee18btech11001},
{\footnotesize
\begin{align}
\label{eq:ee18btech11001_system}
	H(s) = \frac{K(s+j2\pi 10^{3})(s+j2\pi 10^{4})^{2}}{(s+j2\pi 10^{1})(s+j2\pi 10^{2})^{2}(s+j2\pi 10^{5})^{2}(s+j2\pi 10^{6})}
\end{align}
}
\item Justify the above results.
\\
\solution
Let us consider a generalized transfer gain
\begin{align}
	H(s) = k \dfrac{(s-z_{1})(s-z_{2})...(s-z_{m-1})(s-z_{m})}{(s-p_{1})(s-p_{2})....(s-p_{n-1})(s-p_{n})}
\end{align}
The gain
\begin{multline}
	G(f) = 20\log\abs{H(s)} 
\\
= 20\log \abs{k} + 20\log \abs{s-z_{1}} 
	    \\
	    + 20\log \abs{s-z_{2}} + \dots + 20\log \abs{s-z_{m}} 
\\
- 20\log \abs{s-p_{1}} 
	    - 20\log \abs{s-p_{2}} 
	    \\
- \dots - 20\log \abs{s-z_{n}} 
\end{multline}
%
Substituting $s = \j \omega$, for real $z_1$
\begin{align}
	20\log \abs{s-z_{1}} &= 20\log \abs{\sqrt{\omega^{2} + z_{1}^{2}}}
\\
&= 
\begin{cases}
20 \log \abs{z_{1}}, & \omega \ll z_{1}
\\
20 \log \abs{\omega}, & \omega \gg z_{1}
\end{cases}
\end{align}
%
Taking the derivative, 
\begin{align}
	\frac{d\brak{20\log \abs{s-z_{1}}}}{d\brak{\log \abs{\omega}}} 
= 
\begin{cases}
0, & \omega \ll z_{1}
\\
20, & \omega \gg z_{1}
\end{cases}
\end{align}
%
Thus, when a zero is encountered, the gradient of $H(\j\omega)$ jumps by +20 in the log scale.  When a pole is encountered, the gradient falls by -20. Note that this is a very loose justification, but works well in practice.

%-----------------------------------------------------------------------%

\item Obtain the Bode plot and the slope plot for $H(s)$ and verify with  Fig. \ref{fig:ee18btech11001_bode}
\\
\solution Bode Plot of obtained Transfer Function is 
\begin{figure}[htp]
    \centering
    \includegraphics[width=\columnwidth]{./figs/ee18btech11001/ee18btech11001_2.eps}
    \caption{}
    \label{fig:ee18btech11001_2}
\end{figure}
%
Fig. \ref{fig:ee18btech11001}, obtained from  \eqref{eq:ee18btech11001_system},
is a close reconstruction of Fig. \ref{fig:ee18btech11001}.
\end{enumerate}



\caption{}
\label{table:ee18btech11001}
\end{table}


%-----------------------------------------------------------------------%

\item Obtain the transfer function of $H(s)$.
\\
\solution From Table \ref{table:ee18btech11001},
{\footnotesize
\begin{align}
\label{eq:ee18btech11001_system}
	H(s) = \frac{K(s+j2\pi 10^{3})(s+j2\pi 10^{4})^{2}}{(s+j2\pi 10^{1})(s+j2\pi 10^{2})^{2}(s+j2\pi 10^{5})^{2}(s+j2\pi 10^{6})}
\end{align}
}
\item Justify the above results.
\\
\solution
Let us consider a generalized transfer gain
\begin{align}
	H(s) = k \dfrac{(s-z_{1})(s-z_{2})...(s-z_{m-1})(s-z_{m})}{(s-p_{1})(s-p_{2})....(s-p_{n-1})(s-p_{n})}
\end{align}
The gain
\begin{multline}
	G(f) = 20\log\abs{H(s)} 
\\
= 20\log \abs{k} + 20\log \abs{s-z_{1}} 
	    \\
	    + 20\log \abs{s-z_{2}} + \dots + 20\log \abs{s-z_{m}} 
\\
- 20\log \abs{s-p_{1}} 
	    - 20\log \abs{s-p_{2}} 
	    \\
- \dots - 20\log \abs{s-z_{n}} 
\end{multline}
%
Substituting $s = \j \omega$, for real $z_1$
\begin{align}
	20\log \abs{s-z_{1}} &= 20\log \abs{\sqrt{\omega^{2} + z_{1}^{2}}}
\\
&= 
\begin{cases}
20 \log \abs{z_{1}}, & \omega \ll z_{1}
\\
20 \log \abs{\omega}, & \omega \gg z_{1}
\end{cases}
\end{align}
%
Taking the derivative, 
\begin{align}
	\frac{d\brak{20\log \abs{s-z_{1}}}}{d\brak{\log \abs{\omega}}} 
= 
\begin{cases}
0, & \omega \ll z_{1}
\\
20, & \omega \gg z_{1}
\end{cases}
\end{align}
%
Thus, when a zero is encountered, the gradient of $H(\j\omega)$ jumps by +20 in the log scale.  When a pole is encountered, the gradient falls by -20. Note that this is a very loose justification, but works well in practice.

%-----------------------------------------------------------------------%

\item Obtain the Bode plot and the slope plot for $H(s)$ and verify with  Fig. \ref{fig:ee18btech11001_bode}
\\
\solution Bode Plot of obtained Transfer Function is 
\begin{figure}[htp]
    \centering
    \includegraphics[width=\columnwidth]{./figs/ee18btech11001/ee18btech11001_2.eps}
    \caption{}
    \label{fig:ee18btech11001_2}
\end{figure}
%
Fig. \ref{fig:ee18btech11001}, obtained from  \eqref{eq:ee18btech11001_system},
is a close reconstruction of Fig. \ref{fig:ee18btech11001}.
\end{enumerate}



\caption{}
\label{table:ee18btech11001}
\end{table}


%-----------------------------------------------------------------------%

\item Obtain the transfer function of $H(s)$.
\\
\solution From Table \ref{table:ee18btech11001},
{\footnotesize
\begin{align}
\label{eq:ee18btech11001_system}
	H(s) = \frac{K(s+j2\pi 10^{3})(s+j2\pi 10^{4})^{2}}{(s+j2\pi 10^{1})(s+j2\pi 10^{2})^{2}(s+j2\pi 10^{5})^{2}(s+j2\pi 10^{6})}
\end{align}
}
\item Justify the above results.
\\
\solution
Let us consider a generalized transfer gain
\begin{align}
	H(s) = k \dfrac{(s-z_{1})(s-z_{2})...(s-z_{m-1})(s-z_{m})}{(s-p_{1})(s-p_{2})....(s-p_{n-1})(s-p_{n})}
\end{align}
The gain
\begin{multline}
	G(f) = 20\log\abs{H(s)} 
\\
= 20\log \abs{k} + 20\log \abs{s-z_{1}} 
	    \\
	    + 20\log \abs{s-z_{2}} + \dots + 20\log \abs{s-z_{m}} 
\\
- 20\log \abs{s-p_{1}} 
	    - 20\log \abs{s-p_{2}} 
	    \\
- \dots - 20\log \abs{s-z_{n}} 
\end{multline}
%
Substituting $s = \j \omega$, for real $z_1$
\begin{align}
	20\log \abs{s-z_{1}} &= 20\log \abs{\sqrt{\omega^{2} + z_{1}^{2}}}
\\
&= 
\begin{cases}
20 \log \abs{z_{1}}, & \omega \ll z_{1}
\\
20 \log \abs{\omega}, & \omega \gg z_{1}
\end{cases}
\end{align}
%
Taking the derivative, 
\begin{align}
	\frac{d\brak{20\log \abs{s-z_{1}}}}{d\brak{\log \abs{\omega}}} 
= 
\begin{cases}
0, & \omega \ll z_{1}
\\
20, & \omega \gg z_{1}
\end{cases}
\end{align}
%
Thus, when a zero is encountered, the gradient of $H(\j\omega)$ jumps by +20 in the log scale.  When a pole is encountered, the gradient falls by -20. Note that this is a very loose justification, but works well in practice.

%-----------------------------------------------------------------------%

\item Obtain the Bode plot and the slope plot for $H(s)$ and verify with  Fig. \ref{fig:ee18btech11001_bode}
\\
\solution Bode Plot of obtained Transfer Function is 
\begin{figure}[htp]
    \centering
    \includegraphics[width=\columnwidth]{./figs/ee18btech11001/ee18btech11001_2.eps}
    \caption{}
    \label{fig:ee18btech11001_2}
\end{figure}
%
Fig. \ref{fig:ee18btech11001}, obtained from  \eqref{eq:ee18btech11001_system},
is a close reconstruction of Fig. \ref{fig:ee18btech11001}.
\end{enumerate}



\caption{}
\label{table:ee18btech11001}
\end{table}


%-----------------------------------------------------------------------%

\item Obtain the transfer function of $H(s)$.
\\
\solution From Table \ref{table:ee18btech11001},
{\footnotesize
\begin{align}
\label{eq:ee18btech11001_system}
	H(s) = \frac{K(s+j2\pi 10^{3})(s+j2\pi 10^{4})^{2}}{(s+j2\pi 10^{1})(s+j2\pi 10^{2})^{2}(s+j2\pi 10^{5})^{2}(s+j2\pi 10^{6})}
\end{align}
}
\item Justify the above results.
\\
\solution
Let us consider a generalized transfer gain
\begin{align}
	H(s) = k \dfrac{(s-z_{1})(s-z_{2})...(s-z_{m-1})(s-z_{m})}{(s-p_{1})(s-p_{2})....(s-p_{n-1})(s-p_{n})}
\end{align}
The gain
\begin{multline}
	G(f) = 20\log\abs{H(s)} 
\\
= 20\log \abs{k} + 20\log \abs{s-z_{1}} 
	    \\
	    + 20\log \abs{s-z_{2}} + \dots + 20\log \abs{s-z_{m}} 
\\
- 20\log \abs{s-p_{1}} 
	    - 20\log \abs{s-p_{2}} 
	    \\
- \dots - 20\log \abs{s-z_{n}} 
\end{multline}
%
Substituting $s = \j \omega$, for real $z_1$
\begin{align}
	20\log \abs{s-z_{1}} &= 20\log \abs{\sqrt{\omega^{2} + z_{1}^{2}}}
\\
&= 
\begin{cases}
20 \log \abs{z_{1}}, & \omega \ll z_{1}
\\
20 \log \abs{\omega}, & \omega \gg z_{1}
\end{cases}
\end{align}
%
Taking the derivative, 
\begin{align}
	\frac{d\brak{20\log \abs{s-z_{1}}}}{d\brak{\log \abs{\omega}}} 
= 
\begin{cases}
0, & \omega \ll z_{1}
\\
20, & \omega \gg z_{1}
\end{cases}
\end{align}
%
Thus, when a zero is encountered, the gradient of $H(\j\omega)$ jumps by +20 in the log scale.  When a pole is encountered, the gradient falls by -20. Note that this is a very loose justification, but works well in practice.

%-----------------------------------------------------------------------%

\item Obtain the Bode plot and the slope plot for $H(s)$ and verify with  Fig. \ref{fig:ee18btech11001_bode}
\\
\solution Bode Plot of obtained Transfer Function is 
\begin{figure}[htp]
    \centering
    \includegraphics[width=\columnwidth]{./figs/ee18btech11001/ee18btech11001_2.eps}
    \caption{}
    \label{fig:ee18btech11001_2}
\end{figure}
%
Fig. \ref{fig:ee18btech11001}, obtained from  \eqref{eq:ee18btech11001_system},
is a close reconstruction of Fig. \ref{fig:ee18btech11001}.
\end{enumerate}


