\begin{enumerate}[label=\thesection.\arabic*.,ref=\thesection.\theenumi]
\numberwithin{equation}{enumi}

\item The Block diagram of a system is illustrated in the figure shown, where $X(s)$ is the input and $Y(s)$ is the output.  Draw the equivalent signal flow graph. 
\renewcommand{\thefigure}{\theenumi.\arabic{figure}}
%
\begin{figure}[!ht]
    \begin{center}
		
		\resizebox{\columnwidth}{!}{\input{./figs/ee18btech11003/block_diagram.tex}}
	\end{center}
\caption{Block Diagram}
\label{fig:ee18btech11003_block_diagram}
\end{figure}
\\
\solution The signal flow graph of the block diagram in Fig. \ref{fig:ee18btech11003_block_diagram} is available in Fig. \ref{fig:ee18btech11003_signal_flow}
%
\begin{figure}[!ht]
\begin{center}
		
		\resizebox{\columnwidth}{!}{\input{./figs/ee18btech11003/signal_flow.tex}}
	\end{center}
\caption{Signal Flow Graph}
\label{fig:ee18btech11003_signal_flow}
\end{figure}
%
\renewcommand{\thefigure}{\theenumi}
\item Draw all the forward paths in Fig. \ref{fig:ee18btech11003_signal_flow}
and compute the respective gains.
\renewcommand{\thefigure}{\theenumi.\arabic{figure}}
\\
\solution The forward paths are available in Figs. \ref{fig:ee18btech11003_P1}
 and \ref{fig:ee18btech11003_P2}.  The respective gains are
\begin{align}
P_1&=s \brak{\frac{1}{s}}=1
\\
P_2&=(1/s)(1/s)=1/s^2
\end{align}
%
\begin{figure}[!ht]
\begin{center}
		
		\resizebox{\columnwidth}{!}{\input{./figs/ee18btech11003/P1.tex}}
	\end{center}
\caption{$P_1$}
\label{fig:ee18btech11003_P1}
\end{figure}
%
\begin{figure}[!ht]
\begin{center}
		
		\resizebox{\columnwidth}{!}{\input{./figs/ee18btech11003/P2.tex}}
	\end{center}
\caption{$P_2$}
\label{fig:ee18btech11003_P2}
\end{figure}
\renewcommand{\thefigure}{\theenumi}
%
\item Draw all the loops in Fig. \ref{fig:ee18btech11003_signal_flow} and calculate the respective gains.
\renewcommand{\thefigure}{\theenumi.\arabic{figure}}
\\
\solution The loops are available in Figs. \ref{fig:ee18btech11003_L1}-\ref{fig:ee18btech11003_L4}
and the corresponding gains are
%
\begin{align}
L_1&=(-1)(s)=-s
\\
L_2&=s\brak{\frac{1}{s}}\brak{-1}=-1
\\
L_3&=\brak{\frac{1}{s}}(-1)=-\frac{1}{s}
\\
L_4&=\brak{\frac{1}{s}}\brak{\frac{1}{s}}(-1)=-\frac{1}{s^2}
\end{align}

\begin{figure}[!ht]
\begin{center}
		
		\resizebox{\columnwidth}{!}{\input{./figs/ee18btech11003/L1.tex}}
	\end{center}
\caption{$L_1$}
\label{fig:ee18btech11003_L1}
\end{figure}



\begin{figure}[!ht]
\begin{center}
		
		\resizebox{\columnwidth}{!}{\input{./figs/ee18btech11003/L2.tex}}
	\end{center}
\caption{$L_2$}
\label{fig:ee18btech11003_L2}
\end{figure}



\begin{figure}[!ht]
\begin{center}
		
		\resizebox{\columnwidth}{!}{\input{./figs/ee18btech11003/L3.tex}}
	\end{center}
\caption{$L_3$}
\label{fig:ee18btech11003_L3}
\end{figure}



\begin{figure}[!ht]
\begin{center}
		
		\resizebox{\columnwidth}{!}{\input{./figs/ee18btech11003/L4.tex}}
	\end{center}
\caption{$L_4$}
\label{fig:ee18btech11003_L4}
\end{figure}

\renewcommand{\thefigure}{\theenumi}

\item State Mason's Gain formula and explain the parameters through a table.
\\
\solution 
According to Mason's Gain Formula,
\begin{align}
T &= \frac{Y(s)}{X(s)} 
\\
 &= \frac{\sum_{i=1}^{N} P_i\Delta_i}{\Delta}
\label{eq:ee18btech11003_mason}
\end{align}
%
where the parameters are described in Table \ref{table:ee18btech11003}
\begin{table}[!ht]
\centering
\begin{enumerate}[label=\thesection.\arabic*.,ref=\thesection.\theenumi]
\numberwithin{equation}{enumi}

\item The Block diagram of a system is illustrated in the figure shown, where $X(s)$ is the input and $Y(s)$ is the output.  Draw the equivalent signal flow graph. 
\renewcommand{\thefigure}{\theenumi.\arabic{figure}}
%
\begin{figure}[!ht]
    \begin{center}
		
		\resizebox{\columnwidth}{!}{\input{./figs/ee18btech11003/block_diagram.tex}}
	\end{center}
\caption{Block Diagram}
\label{fig:ee18btech11003_block_diagram}
\end{figure}
\\
\solution The signal flow graph of the block diagram in Fig. \ref{fig:ee18btech11003_block_diagram} is available in Fig. \ref{fig:ee18btech11003_signal_flow}
%
\begin{figure}[!ht]
\begin{center}
		
		\resizebox{\columnwidth}{!}{\input{./figs/ee18btech11003/signal_flow.tex}}
	\end{center}
\caption{Signal Flow Graph}
\label{fig:ee18btech11003_signal_flow}
\end{figure}
%
\renewcommand{\thefigure}{\theenumi}
\item Draw all the forward paths in Fig. \ref{fig:ee18btech11003_signal_flow}
and compute the respective gains.
\renewcommand{\thefigure}{\theenumi.\arabic{figure}}
\\
\solution The forward paths are available in Figs. \ref{fig:ee18btech11003_P1}
 and \ref{fig:ee18btech11003_P2}.  The respective gains are
\begin{align}
P_1&=s \brak{\frac{1}{s}}=1
\\
P_2&=(1/s)(1/s)=1/s^2
\end{align}
%
\begin{figure}[!ht]
\begin{center}
		
		\resizebox{\columnwidth}{!}{\input{./figs/ee18btech11003/P1.tex}}
	\end{center}
\caption{$P_1$}
\label{fig:ee18btech11003_P1}
\end{figure}
%
\begin{figure}[!ht]
\begin{center}
		
		\resizebox{\columnwidth}{!}{\input{./figs/ee18btech11003/P2.tex}}
	\end{center}
\caption{$P_2$}
\label{fig:ee18btech11003_P2}
\end{figure}
\renewcommand{\thefigure}{\theenumi}
%
\item Draw all the loops in Fig. \ref{fig:ee18btech11003_signal_flow} and calculate the respective gains.
\renewcommand{\thefigure}{\theenumi.\arabic{figure}}
\\
\solution The loops are available in Figs. \ref{fig:ee18btech11003_L1}-\ref{fig:ee18btech11003_L4}
and the corresponding gains are
%
\begin{align}
L_1&=(-1)(s)=-s
\\
L_2&=s\brak{\frac{1}{s}}\brak{-1}=-1
\\
L_3&=\brak{\frac{1}{s}}(-1)=-\frac{1}{s}
\\
L_4&=\brak{\frac{1}{s}}\brak{\frac{1}{s}}(-1)=-\frac{1}{s^2}
\end{align}

\begin{figure}[!ht]
\begin{center}
		
		\resizebox{\columnwidth}{!}{\input{./figs/ee18btech11003/L1.tex}}
	\end{center}
\caption{$L_1$}
\label{fig:ee18btech11003_L1}
\end{figure}



\begin{figure}[!ht]
\begin{center}
		
		\resizebox{\columnwidth}{!}{\input{./figs/ee18btech11003/L2.tex}}
	\end{center}
\caption{$L_2$}
\label{fig:ee18btech11003_L2}
\end{figure}



\begin{figure}[!ht]
\begin{center}
		
		\resizebox{\columnwidth}{!}{\input{./figs/ee18btech11003/L3.tex}}
	\end{center}
\caption{$L_3$}
\label{fig:ee18btech11003_L3}
\end{figure}



\begin{figure}[!ht]
\begin{center}
		
		\resizebox{\columnwidth}{!}{\input{./figs/ee18btech11003/L4.tex}}
	\end{center}
\caption{$L_4$}
\label{fig:ee18btech11003_L4}
\end{figure}

\renewcommand{\thefigure}{\theenumi}

\item State Mason's Gain formula and explain the parameters through a table.
\\
\solution 
According to Mason's Gain Formula,
\begin{align}
T &= \frac{Y(s)}{X(s)} 
\\
 &= \frac{\sum_{i=1}^{N} P_i\Delta_i}{\Delta}
\label{eq:ee18btech11003_mason}
\end{align}
%
where the parameters are described in Table \ref{table:ee18btech11003}
\begin{table}[!ht]
\centering
\begin{enumerate}[label=\thesection.\arabic*.,ref=\thesection.\theenumi]
\numberwithin{equation}{enumi}

\item The Block diagram of a system is illustrated in the figure shown, where $X(s)$ is the input and $Y(s)$ is the output.  Draw the equivalent signal flow graph. 
\renewcommand{\thefigure}{\theenumi.\arabic{figure}}
%
\begin{figure}[!ht]
    \begin{center}
		
		\resizebox{\columnwidth}{!}{\input{./figs/ee18btech11003/block_diagram.tex}}
	\end{center}
\caption{Block Diagram}
\label{fig:ee18btech11003_block_diagram}
\end{figure}
\\
\solution The signal flow graph of the block diagram in Fig. \ref{fig:ee18btech11003_block_diagram} is available in Fig. \ref{fig:ee18btech11003_signal_flow}
%
\begin{figure}[!ht]
\begin{center}
		
		\resizebox{\columnwidth}{!}{\input{./figs/ee18btech11003/signal_flow.tex}}
	\end{center}
\caption{Signal Flow Graph}
\label{fig:ee18btech11003_signal_flow}
\end{figure}
%
\renewcommand{\thefigure}{\theenumi}
\item Draw all the forward paths in Fig. \ref{fig:ee18btech11003_signal_flow}
and compute the respective gains.
\renewcommand{\thefigure}{\theenumi.\arabic{figure}}
\\
\solution The forward paths are available in Figs. \ref{fig:ee18btech11003_P1}
 and \ref{fig:ee18btech11003_P2}.  The respective gains are
\begin{align}
P_1&=s \brak{\frac{1}{s}}=1
\\
P_2&=(1/s)(1/s)=1/s^2
\end{align}
%
\begin{figure}[!ht]
\begin{center}
		
		\resizebox{\columnwidth}{!}{\input{./figs/ee18btech11003/P1.tex}}
	\end{center}
\caption{$P_1$}
\label{fig:ee18btech11003_P1}
\end{figure}
%
\begin{figure}[!ht]
\begin{center}
		
		\resizebox{\columnwidth}{!}{\input{./figs/ee18btech11003/P2.tex}}
	\end{center}
\caption{$P_2$}
\label{fig:ee18btech11003_P2}
\end{figure}
\renewcommand{\thefigure}{\theenumi}
%
\item Draw all the loops in Fig. \ref{fig:ee18btech11003_signal_flow} and calculate the respective gains.
\renewcommand{\thefigure}{\theenumi.\arabic{figure}}
\\
\solution The loops are available in Figs. \ref{fig:ee18btech11003_L1}-\ref{fig:ee18btech11003_L4}
and the corresponding gains are
%
\begin{align}
L_1&=(-1)(s)=-s
\\
L_2&=s\brak{\frac{1}{s}}\brak{-1}=-1
\\
L_3&=\brak{\frac{1}{s}}(-1)=-\frac{1}{s}
\\
L_4&=\brak{\frac{1}{s}}\brak{\frac{1}{s}}(-1)=-\frac{1}{s^2}
\end{align}

\begin{figure}[!ht]
\begin{center}
		
		\resizebox{\columnwidth}{!}{\input{./figs/ee18btech11003/L1.tex}}
	\end{center}
\caption{$L_1$}
\label{fig:ee18btech11003_L1}
\end{figure}



\begin{figure}[!ht]
\begin{center}
		
		\resizebox{\columnwidth}{!}{\input{./figs/ee18btech11003/L2.tex}}
	\end{center}
\caption{$L_2$}
\label{fig:ee18btech11003_L2}
\end{figure}



\begin{figure}[!ht]
\begin{center}
		
		\resizebox{\columnwidth}{!}{\input{./figs/ee18btech11003/L3.tex}}
	\end{center}
\caption{$L_3$}
\label{fig:ee18btech11003_L3}
\end{figure}



\begin{figure}[!ht]
\begin{center}
		
		\resizebox{\columnwidth}{!}{\input{./figs/ee18btech11003/L4.tex}}
	\end{center}
\caption{$L_4$}
\label{fig:ee18btech11003_L4}
\end{figure}

\renewcommand{\thefigure}{\theenumi}

\item State Mason's Gain formula and explain the parameters through a table.
\\
\solution 
According to Mason's Gain Formula,
\begin{align}
T &= \frac{Y(s)}{X(s)} 
\\
 &= \frac{\sum_{i=1}^{N} P_i\Delta_i}{\Delta}
\label{eq:ee18btech11003_mason}
\end{align}
%
where the parameters are described in Table \ref{table:ee18btech11003}
\begin{table}[!ht]
\centering
\begin{enumerate}[label=\thesection.\arabic*.,ref=\thesection.\theenumi]
\numberwithin{equation}{enumi}

\item The Block diagram of a system is illustrated in the figure shown, where $X(s)$ is the input and $Y(s)$ is the output.  Draw the equivalent signal flow graph. 
\renewcommand{\thefigure}{\theenumi.\arabic{figure}}
%
\begin{figure}[!ht]
    \begin{center}
		
		\resizebox{\columnwidth}{!}{\input{./figs/ee18btech11003/block_diagram.tex}}
	\end{center}
\caption{Block Diagram}
\label{fig:ee18btech11003_block_diagram}
\end{figure}
\\
\solution The signal flow graph of the block diagram in Fig. \ref{fig:ee18btech11003_block_diagram} is available in Fig. \ref{fig:ee18btech11003_signal_flow}
%
\begin{figure}[!ht]
\begin{center}
		
		\resizebox{\columnwidth}{!}{\input{./figs/ee18btech11003/signal_flow.tex}}
	\end{center}
\caption{Signal Flow Graph}
\label{fig:ee18btech11003_signal_flow}
\end{figure}
%
\renewcommand{\thefigure}{\theenumi}
\item Draw all the forward paths in Fig. \ref{fig:ee18btech11003_signal_flow}
and compute the respective gains.
\renewcommand{\thefigure}{\theenumi.\arabic{figure}}
\\
\solution The forward paths are available in Figs. \ref{fig:ee18btech11003_P1}
 and \ref{fig:ee18btech11003_P2}.  The respective gains are
\begin{align}
P_1&=s \brak{\frac{1}{s}}=1
\\
P_2&=(1/s)(1/s)=1/s^2
\end{align}
%
\begin{figure}[!ht]
\begin{center}
		
		\resizebox{\columnwidth}{!}{\input{./figs/ee18btech11003/P1.tex}}
	\end{center}
\caption{$P_1$}
\label{fig:ee18btech11003_P1}
\end{figure}
%
\begin{figure}[!ht]
\begin{center}
		
		\resizebox{\columnwidth}{!}{\input{./figs/ee18btech11003/P2.tex}}
	\end{center}
\caption{$P_2$}
\label{fig:ee18btech11003_P2}
\end{figure}
\renewcommand{\thefigure}{\theenumi}
%
\item Draw all the loops in Fig. \ref{fig:ee18btech11003_signal_flow} and calculate the respective gains.
\renewcommand{\thefigure}{\theenumi.\arabic{figure}}
\\
\solution The loops are available in Figs. \ref{fig:ee18btech11003_L1}-\ref{fig:ee18btech11003_L4}
and the corresponding gains are
%
\begin{align}
L_1&=(-1)(s)=-s
\\
L_2&=s\brak{\frac{1}{s}}\brak{-1}=-1
\\
L_3&=\brak{\frac{1}{s}}(-1)=-\frac{1}{s}
\\
L_4&=\brak{\frac{1}{s}}\brak{\frac{1}{s}}(-1)=-\frac{1}{s^2}
\end{align}

\begin{figure}[!ht]
\begin{center}
		
		\resizebox{\columnwidth}{!}{\input{./figs/ee18btech11003/L1.tex}}
	\end{center}
\caption{$L_1$}
\label{fig:ee18btech11003_L1}
\end{figure}



\begin{figure}[!ht]
\begin{center}
		
		\resizebox{\columnwidth}{!}{\input{./figs/ee18btech11003/L2.tex}}
	\end{center}
\caption{$L_2$}
\label{fig:ee18btech11003_L2}
\end{figure}



\begin{figure}[!ht]
\begin{center}
		
		\resizebox{\columnwidth}{!}{\input{./figs/ee18btech11003/L3.tex}}
	\end{center}
\caption{$L_3$}
\label{fig:ee18btech11003_L3}
\end{figure}



\begin{figure}[!ht]
\begin{center}
		
		\resizebox{\columnwidth}{!}{\input{./figs/ee18btech11003/L4.tex}}
	\end{center}
\caption{$L_4$}
\label{fig:ee18btech11003_L4}
\end{figure}

\renewcommand{\thefigure}{\theenumi}

\item State Mason's Gain formula and explain the parameters through a table.
\\
\solution 
According to Mason's Gain Formula,
\begin{align}
T &= \frac{Y(s)}{X(s)} 
\\
 &= \frac{\sum_{i=1}^{N} P_i\Delta_i}{\Delta}
\label{eq:ee18btech11003_mason}
\end{align}
%
where the parameters are described in Table \ref{table:ee18btech11003}
\begin{table}[!ht]
\centering
\input{./tables/ee18btech11003.tex}
\caption{}
\label{table:ee18btech11003}
\end{table}
\item List the parameters in Table \ref{table:ee18btech11003}
for Fig. \ref{fig:ee18btech11003_signal_flow}.
\\
\solution The parameters are available in Table \ref{table:ee18btech11003_ex}

\begin{table}[!ht]
\centering
\input{./tables/ee18btech11003_ex.tex}
\caption{}
\label{table:ee18btech11003_ex}
\end{table}

\item  Find the transfer function using Mason's Gain Formula.
\renewcommand{\thefigure}{\theenumi.\arabic{figure}}
%
\\
\solution From \eqref{eq:ee18btech11003_mason} and \ref{table:ee18btech11003_ex},
\begin{align}
T(s)&=\frac{P_1 \Delta_1+P_2 \Delta_2}{\Delta}
\\
&=\frac{1 +\frac{1}{s^2}}{1-(-s-1-\frac{1}{s}-\frac{1}{s^2})}
\\
&=\frac{s^2+1}{s^3+2s^2+s+1}
\end{align}
%
after simplification.
\renewcommand{\thefigure}{\theenumi}
\item Write a program to compute Mason's gain formula, given the branch nodes and gains for each path.
\end{enumerate}

\caption{}
\label{table:ee18btech11003}
\end{table}
\item List the parameters in Table \ref{table:ee18btech11003}
for Fig. \ref{fig:ee18btech11003_signal_flow}.
\\
\solution The parameters are available in Table \ref{table:ee18btech11003_ex}

\begin{table}[!ht]
\centering
\input{./tables/ee18btech11003_ex.tex}
\caption{}
\label{table:ee18btech11003_ex}
\end{table}

\item  Find the transfer function using Mason's Gain Formula.
\renewcommand{\thefigure}{\theenumi.\arabic{figure}}
%
\\
\solution From \eqref{eq:ee18btech11003_mason} and \ref{table:ee18btech11003_ex},
\begin{align}
T(s)&=\frac{P_1 \Delta_1+P_2 \Delta_2}{\Delta}
\\
&=\frac{1 +\frac{1}{s^2}}{1-(-s-1-\frac{1}{s}-\frac{1}{s^2})}
\\
&=\frac{s^2+1}{s^3+2s^2+s+1}
\end{align}
%
after simplification.
\renewcommand{\thefigure}{\theenumi}
\item Write a program to compute Mason's gain formula, given the branch nodes and gains for each path.
\end{enumerate}

\caption{}
\label{table:ee18btech11003}
\end{table}
\item List the parameters in Table \ref{table:ee18btech11003}
for Fig. \ref{fig:ee18btech11003_signal_flow}.
\\
\solution The parameters are available in Table \ref{table:ee18btech11003_ex}

\begin{table}[!ht]
\centering
\input{./tables/ee18btech11003_ex.tex}
\caption{}
\label{table:ee18btech11003_ex}
\end{table}

\item  Find the transfer function using Mason's Gain Formula.
\renewcommand{\thefigure}{\theenumi.\arabic{figure}}
%
\\
\solution From \eqref{eq:ee18btech11003_mason} and \ref{table:ee18btech11003_ex},
\begin{align}
T(s)&=\frac{P_1 \Delta_1+P_2 \Delta_2}{\Delta}
\\
&=\frac{1 +\frac{1}{s^2}}{1-(-s-1-\frac{1}{s}-\frac{1}{s^2})}
\\
&=\frac{s^2+1}{s^3+2s^2+s+1}
\end{align}
%
after simplification.
\renewcommand{\thefigure}{\theenumi}
\item Write a program to compute Mason's gain formula, given the branch nodes and gains for each path.
\end{enumerate}

\caption{}
\label{table:ee18btech11003}
\end{table}
\item List the parameters in Table \ref{table:ee18btech11003}
for Fig. \ref{fig:ee18btech11003_signal_flow}.
\\
\solution The parameters are available in Table \ref{table:ee18btech11003_ex}

\begin{table}[!ht]
\centering
\input{./tables/ee18btech11003_ex.tex}
\caption{}
\label{table:ee18btech11003_ex}
\end{table}

\item  Find the transfer function using Mason's Gain Formula.
\renewcommand{\thefigure}{\theenumi.\arabic{figure}}
%
\\
\solution From \eqref{eq:ee18btech11003_mason} and \ref{table:ee18btech11003_ex},
\begin{align}
T(s)&=\frac{P_1 \Delta_1+P_2 \Delta_2}{\Delta}
\\
&=\frac{1 +\frac{1}{s^2}}{1-(-s-1-\frac{1}{s}-\frac{1}{s^2})}
\\
&=\frac{s^2+1}{s^3+2s^2+s+1}
\end{align}
%
after simplification.
\renewcommand{\thefigure}{\theenumi}
\item Write a program to compute Mason's gain formula, given the branch nodes and gains for each path.
\end{enumerate}
