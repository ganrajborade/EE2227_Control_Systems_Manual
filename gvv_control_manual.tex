\documentclass[journal,12pt,twocolumn]{IEEEtran}
%
\usepackage{setspace}
\usepackage{gensymb}
%\doublespacing
\singlespacing

%\usepackage{graphicx}
%\usepackage{amssymb}
%\usepackage{relsize}
\usepackage[cmex10]{amsmath}
%\usepackage{amsthm}
%\interdisplaylinepenalty=2500
%\savesymbol{iint}
%\usepackage{txfonts}
%\restoresymbol{TXF}{iint}
%\usepackage{wasysym}
\usepackage{amsthm}
%\usepackage{iithtlc}
\usepackage{mathrsfs}
\usepackage{txfonts}
\usepackage{stfloats}
\usepackage{bm}
\usepackage{cite}
\usepackage{cases}
\usepackage{subfig}
%\usepackage{xtab}
\usepackage{longtable}
\usepackage{multirow}
%\usepackage{algorithm}
%\usepackage{algpseudocode}
\usepackage{enumitem}
\usepackage{mathtools}
\usepackage{tikz}
\usepackage{circuitikz}
\usepackage{verbatim}
\usepackage{tfrupee}
\usepackage[breaklinks=true]{hyperref}
%\usepackage{stmaryrd}
\usepackage{tkz-euclide} % loads  TikZ and tkz-base
\usetkzobj{all}
\usetikzlibrary{decorations.markings}
\newif\iflabrev
\usepackage{listings}
    \usepackage{color}                                            %%
    \usepackage{array}                                            %%
    \usepackage{longtable}                                        %%
    \usepackage{calc}                                             %%
    \usepackage{multirow}                                         %%
    \usepackage{hhline}                                           %%
    \usepackage{ifthen}                                           %%
  %optionally (for landscape tables embedded in another document): %%
    \usepackage{lscape}     
\usepackage{multicol}
\usepackage{chngcntr}
%\usepackage{enumerate}

%\usepackage{wasysym}
%\newcounter{MYtempeqncnt}
\DeclareMathOperator*{\Res}{Res}
%\renewcommand{\baselinestretch}{2}
\renewcommand\thesection{\arabic{section}}
\renewcommand\thesubsection{\thesection.\arabic{subsection}}
\renewcommand\thesubsubsection{\thesubsection.\arabic{subsubsection}}

\renewcommand\thesectiondis{\arabic{section}}
\renewcommand\thesubsectiondis{\thesectiondis.\arabic{subsection}}
\renewcommand\thesubsubsectiondis{\thesubsectiondis.\arabic{subsubsection}}

% correct bad hyphenation here
\hyphenation{op-tical net-works semi-conduc-tor}
\def\inputGnumericTable{}                                 %%

\lstset{
%language=C,
frame=single, 
breaklines=true,
columns=fullflexible
}
%\lstset{
%language=tex,
%frame=single, 
%breaklines=true
%}

\begin{document}
%


\newtheorem{theorem}{Theorem}[section]
\newtheorem{problem}{Problem}
\newtheorem{proposition}{Proposition}[section]
\newtheorem{lemma}{Lemma}[section]
\newtheorem{corollary}[theorem]{Corollary}
\newtheorem{example}{Example}[section]
\newtheorem{definition}[problem]{Definition}
%\newtheorem{thm}{Theorem}[section] 
%\newtheorem{defn}[thm]{Definition}
%\newtheorem{algorithm}{Algorithm}[section]
%\newtheorem{cor}{Corollary}
\newcommand{\BEQA}{\begin{eqnarray}}
\newcommand{\EEQA}{\end{eqnarray}}
\newcommand{\define}{\stackrel{\triangle}{=}}

\bibliographystyle{IEEEtran}
%\bibliographystyle{ieeetr}


\providecommand{\mbf}{\mathbf}
\providecommand{\pr}[1]{\ensuremath{\Pr\left(#1\right)}}
\providecommand{\qfunc}[1]{\ensuremath{Q\left(#1\right)}}
\providecommand{\sbrak}[1]{\ensuremath{{}\left[#1\right]}}
\providecommand{\lsbrak}[1]{\ensuremath{{}\left[#1\right.}}
\providecommand{\rsbrak}[1]{\ensuremath{{}\left.#1\right]}}
\providecommand{\brak}[1]{\ensuremath{\left(#1\right)}}
\providecommand{\lbrak}[1]{\ensuremath{\left(#1\right.}}
\providecommand{\rbrak}[1]{\ensuremath{\left.#1\right)}}
\providecommand{\cbrak}[1]{\ensuremath{\left\{#1\right\}}}
\providecommand{\lcbrak}[1]{\ensuremath{\left\{#1\right.}}
\providecommand{\rcbrak}[1]{\ensuremath{\left.#1\right\}}}
\theoremstyle{remark}
\newtheorem{rem}{Remark}
\newcommand{\sgn}{\mathop{\mathrm{sgn}}}
\providecommand{\abs}[1]{\left\vert#1\right\vert}
\providecommand{\res}[1]{\Res\displaylimits_{#1}} 
\providecommand{\norm}[1]{\left\lVert#1\right\rVert}
%\providecommand{\norm}[1]{\lVert#1\rVert}
\providecommand{\mtx}[1]{\mathbf{#1}}
\providecommand{\mean}[1]{E\left[ #1 \right]}
\providecommand{\fourier}{\overset{\mathcal{F}}{ \rightleftharpoons}}
%\providecommand{\hilbert}{\overset{\mathcal{H}}{ \rightleftharpoons}}
\providecommand{\system}{\overset{\mathcal{H}}{ \longleftrightarrow}}
	%\newcommand{\solution}[2]{\textbf{Solution:}{#1}}
\newcommand{\solution}{\noindent \textbf{Solution: }}
\newcommand{\cosec}{\,\text{cosec}\,}
\providecommand{\dec}[2]{\ensuremath{\overset{#1}{\underset{#2}{\gtrless}}}}
\newcommand{\myvec}[1]{\ensuremath{\begin{pmatrix}#1\end{pmatrix}}}
\newcommand{\mydet}[1]{\ensuremath{\begin{vmatrix}#1\end{vmatrix}}}
%\numberwithin{equation}{section}
\numberwithin{equation}{subsection}
%\numberwithin{problem}{section}
%\numberwithin{definition}{section}
\makeatletter
\@addtoreset{figure}{problem}
\makeatother

\let\StandardTheFigure\thefigure
\let\vec\mathbf
%\renewcommand{\thefigure}{\theproblem.\arabic{figure}}
\renewcommand{\thefigure}{\theproblem}
%\setlist[enumerate,1]{before=\renewcommand\theequation{\theenumi.\arabic{equation}}
%\counterwithin{equation}{enumi}


%\renewcommand{\theequation}{\arabic{subsection}.\arabic{equation}}

\def\putbox#1#2#3{\makebox[0in][l]{\makebox[#1][l]{}\raisebox{\baselineskip}[0in][0in]{\raisebox{#2}[0in][0in]{#3}}}}
     \def\rightbox#1{\makebox[0in][r]{#1}}
     \def\centbox#1{\makebox[0in]{#1}}
     \def\topbox#1{\raisebox{-\baselineskip}[0in][0in]{#1}}
     \def\midbox#1{\raisebox{-0.5\baselineskip}[0in][0in]{#1}}

\vspace{3cm}

\title{
%	\logo{
Control Systems
%	}
}
\author{ G V V Sharma$^{*}$% <-this % stops a space
	\thanks{*The author is with the Department
		of Electrical Engineering, Indian Institute of Technology, Hyderabad
		502285 India e-mail:  gadepall@iith.ac.in. All content in this manual is released under GNU GPL.  Free and open source.}
	
}	
%\title{
%	\logo{Matrix Analysis through Octave}{\begin{center}\includegraphics[scale=.24]{tlc}\end{center}}{}{HAMDSP}
%}


% paper title
% can use linebreaks \\ within to get better formatting as desired
%\title{Matrix Analysis through Octave}
%
%
% author names and IEEE memberships
% note positions of commas and nonbreaking spaces ( ~ ) LaTeX will not break
% a structure at a ~ so this keeps an author's name from being broken across
% two lines.
% use \thanks{} to gain access to the first footnote area
% a separate \thanks must be used for each paragraph as LaTeX2e's \thanks
% was not built to handle multiple paragraphs
%

%\author{<-this % stops a space
%\thanks{}}
%}
% note the % following the last \IEEEmembership and also \thanks - 
% these prevent an unwanted space from occurring between the last author name
% and the end of the author line. i.e., if you had this:
% 
% \author{....lastname \thanks{...} \thanks{...} }
%                     ^------------^------------^----Do not want these spaces!
%
% a space would be appended to the last name and could cause every name on that
% line to be shifted left slightly. This is one of those "LaTeX things". For
% instance, "\textbf{A} \textbf{B}" will typeset as "A B" not "AB". To get
% "AB" then you have to do: "\textbf{A}\textbf{B}"
% \thanks is no different in this regard, so shield the last } of each \thanks
% that ends a line with a % and do not let a space in before the next \thanks.
% Spaces after \IEEEmembership other than the last one are OK (and needed) as
% you are supposed to have spaces between the names. For what it is worth,
% this is a minor point as most people would not even notice if the said evil
% space somehow managed to creep in.



% The paper headers
%\markboth{Journal of \LaTeX\ Class Files,~Vol.~6, No.~1, January~2007}%
%{Shell \MakeLowercase{\textit{et al.}}: Bare Demo of IEEEtran.cls for Journals}
% The only time the second header will appear is for the odd numbered pages
% after the title page when using the twoside option.
% 
% *** Note that you probably will NOT want to include the author's ***
% *** name in the headers of peer review papers.                   ***
% You can use \ifCLASSOPTIONpeerreview for conditional compilation here if
% you desire.




% If you want to put a publisher's ID mark on the page you can do it like
% this:
%\IEEEpubid{0000--0000/00\$00.00~\copyright~2007 IEEE}
% Remember, if you use this you must call \IEEEpubidadjcol in the second
% column for its text to clear the IEEEpubid mark.



% make the title area
\maketitle

\newpage

\tableofcontents

\bigskip

\renewcommand{\thefigure}{\theenumi}
\renewcommand{\thetable}{\theenumi}
%\renewcommand{\theequation}{\theenumi}

%\begin{abstract}
%%\boldmath
%In this letter, an algorithm for evaluating the exact analytical bit error rate  (BER)  for the piecewise linear (PL) combiner for  multiple relays is presented. Previous results were available only for upto three relays. The algorithm is unique in the sense that  the actual mathematical expressions, that are prohibitively large, need not be explicitly obtained. The diversity gain due to multiple relays is shown through plots of the analytical BER, well supported by simulations. 
%
%\end{abstract}
% IEEEtran.cls defaults to using nonbold math in the Abstract.
% This preserves the distinction between vectors and scalars. However,
% if the journal you are submitting to favors bold math in the abstract,
% then you can use LaTeX's standard command \boldmath at the very start
% of the abstract to achieve this. Many IEEE journals frown on math
% in the abstract anyway.

% Note that keywords are not normally used for peerreview papers.
%\begin{IEEEkeywords}
%Cooperative diversity, decode and forward, piecewise linear
%\end{IEEEkeywords}



% For peer review papers, you can put extra information on the cover
% page as needed:
% \ifCLASSOPTIONpeerreview
% \begin{center} \bfseries EDICS Category: 3-BBND \end{center}
% \fi
%
% For peerreview papers, this IEEEtran command inserts a page break and
% creates the second title. It will be ignored for other modes.
%\IEEEpeerreviewmaketitle

\begin{abstract}
This manual is an introduction to control systems based on GATE problems.Links to sample Python codes are available in the text.  
\end{abstract}
Download python codes using 
\begin{lstlisting}
svn co https://github.com/gadepall/school/trunk/control/codes
\end{lstlisting}

\section{Mason's Gain Formula}
\begin{enumerate}[label=\thesection.\arabic*.,ref=\thesection.\theenumi]
\numberwithin{equation}{enumi}

\item The Block diagram of a system is illustrated in the figure shown, where $X(s)$ is the input and $Y(s)$ is the output.  Draw the equivalent signal flow graph. 
\renewcommand{\thefigure}{\theenumi.\arabic{figure}}
%
\begin{figure}[!ht]
    \begin{center}
		
		\resizebox{\columnwidth}{!}{\input{./figs/ee18btech11003/block_diagram.tex}}
	\end{center}
\caption{Block Diagram}
\label{fig:ee18btech11003_block_diagram}
\end{figure}
\\
\solution The signal flow graph of the block diagram in Fig. \ref{fig:ee18btech11003_block_diagram} is available in Fig. \ref{fig:ee18btech11003_signal_flow}
%
\begin{figure}[!ht]
\begin{center}
		
		\resizebox{\columnwidth}{!}{\input{./figs/ee18btech11003/signal_flow.tex}}
	\end{center}
\caption{Signal Flow Graph}
\label{fig:ee18btech11003_signal_flow}
\end{figure}
%
\renewcommand{\thefigure}{\theenumi}
\item Draw all the forward paths in Fig. \ref{fig:ee18btech11003_signal_flow}
and compute the respective gains.
\renewcommand{\thefigure}{\theenumi.\arabic{figure}}
\\
\solution The forward paths are available in Figs. \ref{fig:ee18btech11003_P1}
 and \ref{fig:ee18btech11003_P2}.  The respective gains are
\begin{align}
P_1&=s \brak{\frac{1}{s}}=1
\\
P_2&=(1/s)(1/s)=1/s^2
\end{align}
%
\begin{figure}[!ht]
\begin{center}
		
		\resizebox{\columnwidth}{!}{\input{./figs/ee18btech11003/P1.tex}}
	\end{center}
\caption{$P_1$}
\label{fig:ee18btech11003_P1}
\end{figure}
%
\begin{figure}[!ht]
\begin{center}
		
		\resizebox{\columnwidth}{!}{\input{./figs/ee18btech11003/P2.tex}}
	\end{center}
\caption{$P_2$}
\label{fig:ee18btech11003_P2}
\end{figure}
\renewcommand{\thefigure}{\theenumi}
%
\item Draw all the loops in Fig. \ref{fig:ee18btech11003_signal_flow} and calculate the respective gains.
\renewcommand{\thefigure}{\theenumi.\arabic{figure}}
\\
\solution The loops are available in Figs. \ref{fig:ee18btech11003_L1}-\ref{fig:ee18btech11003_L4}
and the corresponding gains are
%
\begin{align}
L_1&=(-1)(s)=-s
\\
L_2&=s\brak{\frac{1}{s}}\brak{-1}=-1
\\
L_3&=\brak{\frac{1}{s}}(-1)=-\frac{1}{s}
\\
L_4&=\brak{\frac{1}{s}}\brak{\frac{1}{s}}(-1)=-\frac{1}{s^2}
\end{align}

\begin{figure}[!ht]
\begin{center}
		
		\resizebox{\columnwidth}{!}{\input{./figs/ee18btech11003/L1.tex}}
	\end{center}
\caption{$L_1$}
\label{fig:ee18btech11003_L1}
\end{figure}



\begin{figure}[!ht]
\begin{center}
		
		\resizebox{\columnwidth}{!}{\input{./figs/ee18btech11003/L2.tex}}
	\end{center}
\caption{$L_2$}
\label{fig:ee18btech11003_L2}
\end{figure}



\begin{figure}[!ht]
\begin{center}
		
		\resizebox{\columnwidth}{!}{\input{./figs/ee18btech11003/L3.tex}}
	\end{center}
\caption{$L_3$}
\label{fig:ee18btech11003_L3}
\end{figure}



\begin{figure}[!ht]
\begin{center}
		
		\resizebox{\columnwidth}{!}{\input{./figs/ee18btech11003/L4.tex}}
	\end{center}
\caption{$L_4$}
\label{fig:ee18btech11003_L4}
\end{figure}

\renewcommand{\thefigure}{\theenumi}

\item State Mason's Gain formula and explain the parameters through a table.
\\
\solution 
According to Mason's Gain Formula,
\begin{align}
T &= \frac{Y(s)}{X(s)} 
\\
 &= \frac{\sum_{i=1}^{N} P_i\Delta_i}{\Delta}
\label{eq:ee18btech11003_mason}
\end{align}
%
where the parameters are described in Table \ref{table:ee18btech11003}
\begin{table}[!ht]
\centering
\begin{enumerate}[label=\thesection.\arabic*.,ref=\thesection.\theenumi]
\numberwithin{equation}{enumi}

\item The Block diagram of a system is illustrated in the figure shown, where $X(s)$ is the input and $Y(s)$ is the output.  Draw the equivalent signal flow graph. 
\renewcommand{\thefigure}{\theenumi.\arabic{figure}}
%
\begin{figure}[!ht]
    \begin{center}
		
		\resizebox{\columnwidth}{!}{\input{./figs/ee18btech11003/block_diagram.tex}}
	\end{center}
\caption{Block Diagram}
\label{fig:ee18btech11003_block_diagram}
\end{figure}
\\
\solution The signal flow graph of the block diagram in Fig. \ref{fig:ee18btech11003_block_diagram} is available in Fig. \ref{fig:ee18btech11003_signal_flow}
%
\begin{figure}[!ht]
\begin{center}
		
		\resizebox{\columnwidth}{!}{\input{./figs/ee18btech11003/signal_flow.tex}}
	\end{center}
\caption{Signal Flow Graph}
\label{fig:ee18btech11003_signal_flow}
\end{figure}
%
\renewcommand{\thefigure}{\theenumi}
\item Draw all the forward paths in Fig. \ref{fig:ee18btech11003_signal_flow}
and compute the respective gains.
\renewcommand{\thefigure}{\theenumi.\arabic{figure}}
\\
\solution The forward paths are available in Figs. \ref{fig:ee18btech11003_P1}
 and \ref{fig:ee18btech11003_P2}.  The respective gains are
\begin{align}
P_1&=s \brak{\frac{1}{s}}=1
\\
P_2&=(1/s)(1/s)=1/s^2
\end{align}
%
\begin{figure}[!ht]
\begin{center}
		
		\resizebox{\columnwidth}{!}{\input{./figs/ee18btech11003/P1.tex}}
	\end{center}
\caption{$P_1$}
\label{fig:ee18btech11003_P1}
\end{figure}
%
\begin{figure}[!ht]
\begin{center}
		
		\resizebox{\columnwidth}{!}{\input{./figs/ee18btech11003/P2.tex}}
	\end{center}
\caption{$P_2$}
\label{fig:ee18btech11003_P2}
\end{figure}
\renewcommand{\thefigure}{\theenumi}
%
\item Draw all the loops in Fig. \ref{fig:ee18btech11003_signal_flow} and calculate the respective gains.
\renewcommand{\thefigure}{\theenumi.\arabic{figure}}
\\
\solution The loops are available in Figs. \ref{fig:ee18btech11003_L1}-\ref{fig:ee18btech11003_L4}
and the corresponding gains are
%
\begin{align}
L_1&=(-1)(s)=-s
\\
L_2&=s\brak{\frac{1}{s}}\brak{-1}=-1
\\
L_3&=\brak{\frac{1}{s}}(-1)=-\frac{1}{s}
\\
L_4&=\brak{\frac{1}{s}}\brak{\frac{1}{s}}(-1)=-\frac{1}{s^2}
\end{align}

\begin{figure}[!ht]
\begin{center}
		
		\resizebox{\columnwidth}{!}{\input{./figs/ee18btech11003/L1.tex}}
	\end{center}
\caption{$L_1$}
\label{fig:ee18btech11003_L1}
\end{figure}



\begin{figure}[!ht]
\begin{center}
		
		\resizebox{\columnwidth}{!}{\input{./figs/ee18btech11003/L2.tex}}
	\end{center}
\caption{$L_2$}
\label{fig:ee18btech11003_L2}
\end{figure}



\begin{figure}[!ht]
\begin{center}
		
		\resizebox{\columnwidth}{!}{\input{./figs/ee18btech11003/L3.tex}}
	\end{center}
\caption{$L_3$}
\label{fig:ee18btech11003_L3}
\end{figure}



\begin{figure}[!ht]
\begin{center}
		
		\resizebox{\columnwidth}{!}{\input{./figs/ee18btech11003/L4.tex}}
	\end{center}
\caption{$L_4$}
\label{fig:ee18btech11003_L4}
\end{figure}

\renewcommand{\thefigure}{\theenumi}

\item State Mason's Gain formula and explain the parameters through a table.
\\
\solution 
According to Mason's Gain Formula,
\begin{align}
T &= \frac{Y(s)}{X(s)} 
\\
 &= \frac{\sum_{i=1}^{N} P_i\Delta_i}{\Delta}
\label{eq:ee18btech11003_mason}
\end{align}
%
where the parameters are described in Table \ref{table:ee18btech11003}
\begin{table}[!ht]
\centering
\begin{enumerate}[label=\thesection.\arabic*.,ref=\thesection.\theenumi]
\numberwithin{equation}{enumi}

\item The Block diagram of a system is illustrated in the figure shown, where $X(s)$ is the input and $Y(s)$ is the output.  Draw the equivalent signal flow graph. 
\renewcommand{\thefigure}{\theenumi.\arabic{figure}}
%
\begin{figure}[!ht]
    \begin{center}
		
		\resizebox{\columnwidth}{!}{\input{./figs/ee18btech11003/block_diagram.tex}}
	\end{center}
\caption{Block Diagram}
\label{fig:ee18btech11003_block_diagram}
\end{figure}
\\
\solution The signal flow graph of the block diagram in Fig. \ref{fig:ee18btech11003_block_diagram} is available in Fig. \ref{fig:ee18btech11003_signal_flow}
%
\begin{figure}[!ht]
\begin{center}
		
		\resizebox{\columnwidth}{!}{\input{./figs/ee18btech11003/signal_flow.tex}}
	\end{center}
\caption{Signal Flow Graph}
\label{fig:ee18btech11003_signal_flow}
\end{figure}
%
\renewcommand{\thefigure}{\theenumi}
\item Draw all the forward paths in Fig. \ref{fig:ee18btech11003_signal_flow}
and compute the respective gains.
\renewcommand{\thefigure}{\theenumi.\arabic{figure}}
\\
\solution The forward paths are available in Figs. \ref{fig:ee18btech11003_P1}
 and \ref{fig:ee18btech11003_P2}.  The respective gains are
\begin{align}
P_1&=s \brak{\frac{1}{s}}=1
\\
P_2&=(1/s)(1/s)=1/s^2
\end{align}
%
\begin{figure}[!ht]
\begin{center}
		
		\resizebox{\columnwidth}{!}{\input{./figs/ee18btech11003/P1.tex}}
	\end{center}
\caption{$P_1$}
\label{fig:ee18btech11003_P1}
\end{figure}
%
\begin{figure}[!ht]
\begin{center}
		
		\resizebox{\columnwidth}{!}{\input{./figs/ee18btech11003/P2.tex}}
	\end{center}
\caption{$P_2$}
\label{fig:ee18btech11003_P2}
\end{figure}
\renewcommand{\thefigure}{\theenumi}
%
\item Draw all the loops in Fig. \ref{fig:ee18btech11003_signal_flow} and calculate the respective gains.
\renewcommand{\thefigure}{\theenumi.\arabic{figure}}
\\
\solution The loops are available in Figs. \ref{fig:ee18btech11003_L1}-\ref{fig:ee18btech11003_L4}
and the corresponding gains are
%
\begin{align}
L_1&=(-1)(s)=-s
\\
L_2&=s\brak{\frac{1}{s}}\brak{-1}=-1
\\
L_3&=\brak{\frac{1}{s}}(-1)=-\frac{1}{s}
\\
L_4&=\brak{\frac{1}{s}}\brak{\frac{1}{s}}(-1)=-\frac{1}{s^2}
\end{align}

\begin{figure}[!ht]
\begin{center}
		
		\resizebox{\columnwidth}{!}{\input{./figs/ee18btech11003/L1.tex}}
	\end{center}
\caption{$L_1$}
\label{fig:ee18btech11003_L1}
\end{figure}



\begin{figure}[!ht]
\begin{center}
		
		\resizebox{\columnwidth}{!}{\input{./figs/ee18btech11003/L2.tex}}
	\end{center}
\caption{$L_2$}
\label{fig:ee18btech11003_L2}
\end{figure}



\begin{figure}[!ht]
\begin{center}
		
		\resizebox{\columnwidth}{!}{\input{./figs/ee18btech11003/L3.tex}}
	\end{center}
\caption{$L_3$}
\label{fig:ee18btech11003_L3}
\end{figure}



\begin{figure}[!ht]
\begin{center}
		
		\resizebox{\columnwidth}{!}{\input{./figs/ee18btech11003/L4.tex}}
	\end{center}
\caption{$L_4$}
\label{fig:ee18btech11003_L4}
\end{figure}

\renewcommand{\thefigure}{\theenumi}

\item State Mason's Gain formula and explain the parameters through a table.
\\
\solution 
According to Mason's Gain Formula,
\begin{align}
T &= \frac{Y(s)}{X(s)} 
\\
 &= \frac{\sum_{i=1}^{N} P_i\Delta_i}{\Delta}
\label{eq:ee18btech11003_mason}
\end{align}
%
where the parameters are described in Table \ref{table:ee18btech11003}
\begin{table}[!ht]
\centering
\input{./tables/ee18btech11003.tex}
\caption{}
\label{table:ee18btech11003}
\end{table}
\item List the parameters in Table \ref{table:ee18btech11003}
for Fig. \ref{fig:ee18btech11003_signal_flow}.
\\
\solution The parameters are available in Table \ref{table:ee18btech11003_ex}

\begin{table}[!ht]
\centering
\input{./tables/ee18btech11003_ex.tex}
\caption{}
\label{table:ee18btech11003_ex}
\end{table}

\item  Find the transfer function using Mason's Gain Formula.
\renewcommand{\thefigure}{\theenumi.\arabic{figure}}
%
\\
\solution From \eqref{eq:ee18btech11003_mason} and \ref{table:ee18btech11003_ex},
\begin{align}
T(s)&=\frac{P_1 \Delta_1+P_2 \Delta_2}{\Delta}
\\
&=\frac{1 +\frac{1}{s^2}}{1-(-s-1-\frac{1}{s}-\frac{1}{s^2})}
\\
&=\frac{s^2+1}{s^3+2s^2+s+1}
\end{align}
%
after simplification.
\renewcommand{\thefigure}{\theenumi}
\item Write a program to compute Mason's gain formula, given the branch nodes and gains for each path.
\end{enumerate}

\caption{}
\label{table:ee18btech11003}
\end{table}
\item List the parameters in Table \ref{table:ee18btech11003}
for Fig. \ref{fig:ee18btech11003_signal_flow}.
\\
\solution The parameters are available in Table \ref{table:ee18btech11003_ex}

\begin{table}[!ht]
\centering
\input{./tables/ee18btech11003_ex.tex}
\caption{}
\label{table:ee18btech11003_ex}
\end{table}

\item  Find the transfer function using Mason's Gain Formula.
\renewcommand{\thefigure}{\theenumi.\arabic{figure}}
%
\\
\solution From \eqref{eq:ee18btech11003_mason} and \ref{table:ee18btech11003_ex},
\begin{align}
T(s)&=\frac{P_1 \Delta_1+P_2 \Delta_2}{\Delta}
\\
&=\frac{1 +\frac{1}{s^2}}{1-(-s-1-\frac{1}{s}-\frac{1}{s^2})}
\\
&=\frac{s^2+1}{s^3+2s^2+s+1}
\end{align}
%
after simplification.
\renewcommand{\thefigure}{\theenumi}
\item Write a program to compute Mason's gain formula, given the branch nodes and gains for each path.
\end{enumerate}

\caption{}
\label{table:ee18btech11003}
\end{table}
\item List the parameters in Table \ref{table:ee18btech11003}
for Fig. \ref{fig:ee18btech11003_signal_flow}.
\\
\solution The parameters are available in Table \ref{table:ee18btech11003_ex}

\begin{table}[!ht]
\centering
\input{./tables/ee18btech11003_ex.tex}
\caption{}
\label{table:ee18btech11003_ex}
\end{table}

\item  Find the transfer function using Mason's Gain Formula.
\renewcommand{\thefigure}{\theenumi.\arabic{figure}}
%
\\
\solution From \eqref{eq:ee18btech11003_mason} and \ref{table:ee18btech11003_ex},
\begin{align}
T(s)&=\frac{P_1 \Delta_1+P_2 \Delta_2}{\Delta}
\\
&=\frac{1 +\frac{1}{s^2}}{1-(-s-1-\frac{1}{s}-\frac{1}{s^2})}
\\
&=\frac{s^2+1}{s^3+2s^2+s+1}
\end{align}
%
after simplification.
\renewcommand{\thefigure}{\theenumi}
\item Write a program to compute Mason's gain formula, given the branch nodes and gains for each path.
\end{enumerate}

\section{Bode Plot}
\subsection{Introduction}
\begin{enumerate}[label=\thesection.\arabic*.,ref=\thesection.\theenumi]
\numberwithin{equation}{enumi}

\item For an LTI system, the Bode plot for its gain defined as
\begin{align}
	G(s) = 20\log\abs{H(s)}
	\label{eq:ee18btech11001_gain}
\end{align}
is as illustrated in the Fig. \ref{fig:ee18btech11001_bode}. Express $G(f)$ in terms of $f$.\\
\begin{figure}[ht!]
    \includegraphics[width=\columnwidth]{./figs/ee18btech11001/ee18btech11001.eps}
    \caption{}
    \label{fig:ee18btech11001_bode}
\end{figure}\\

\solution
\begin{align}
 G(f) = 
 \begin{cases} 
        100 & 0 < f < 10^{1} \\
      120-20\log(f) & 10 < f < 10^{2} \\
      200-60\log(f) & 10^2 < f < 10^{3} \\
      140-40\log(f) & 10^{3} < f < 10^{4} \\
       -20 & 10^{4} < f < 10^{5} \\
      180-40\log(f) & 10^{5} < f < 10^{6} \\
      300-60\log(f) & 10^{6} < f < 10^{7}   
 \end{cases}
\end{align}

%-----------------------------------------------------------------------%

\item Express the slope of $G(f)$ in terms of $f$.
\\
\solution The desired slope is 
\begin{align}
\nabla G(f) &= \dfrac{d(G(f))}{d(\log(f))}
\end{align}

\begin{align}
 \nabla G(f) = 
 \begin{cases} 
        0 & 0 < f < 10^{1} \\
      -20 & 10 < f < 10^{2} \\
      -60 & 10^{2} < f < 10^{3} \\
      -40 & 10^{3} < f < 10^{4} \\
       0 & 10^{4} < f < 10^{5} \\
      -40 & 10^{5} < f < 10^{6} \\
      -60 & 10^{6} < f < 10^{7}   
 \end{cases}
\end{align}

%-----------------------------------------------------------------------%

\item Express the change of slope of $G(f)$ in terms of $f$.
\\
\solution\\
$\Delta(\nabla G(f))$  = Change of slope G(f) at f

\begin{align}
 \Delta(\nabla G(f)) = 
 \begin{cases} 
      -20 &  f = 10^{1} \\
      -40 &  f = 10^{2} \\
      +20 &  f = 10^{3} \\
      +40 &  f = 10^{4} \\
      -40 &  f = 10^{5} \\
      -20 &  f = 10^{6} 
 \end{cases}
\label{eq:ee18btech11001_slope_diff}
\end{align}

%-----------------------------------------------------------------------%

\item Tabulate the poles and zeros of $H(s)$ using \eqref{eq:ee18btech11001_slope_diff}.
\\
\solution Table \ref{table:ee18btech11001} provides the details.  
%
\begin{table}[!ht]
\centering
\begin{enumerate}[label=\thesection.\arabic*.,ref=\thesection.\theenumi]
\numberwithin{equation}{enumi}

\item For an LTI system, the Bode plot for its gain defined as
\begin{align}
	G(s) = 20\log\abs{H(s)}
	\label{eq:ee18btech11001_gain}
\end{align}
is as illustrated in the Fig. \ref{fig:ee18btech11001_bode}. Express $G(f)$ in terms of $f$.\\
\begin{figure}[ht!]
    \includegraphics[width=\columnwidth]{./figs/ee18btech11001/ee18btech11001.eps}
    \caption{}
    \label{fig:ee18btech11001_bode}
\end{figure}\\

\solution
\begin{align}
 G(f) = 
 \begin{cases} 
        100 & 0 < f < 10^{1} \\
      120-20\log(f) & 10 < f < 10^{2} \\
      200-60\log(f) & 10^2 < f < 10^{3} \\
      140-40\log(f) & 10^{3} < f < 10^{4} \\
       -20 & 10^{4} < f < 10^{5} \\
      180-40\log(f) & 10^{5} < f < 10^{6} \\
      300-60\log(f) & 10^{6} < f < 10^{7}   
 \end{cases}
\end{align}

%-----------------------------------------------------------------------%

\item Express the slope of $G(f)$ in terms of $f$.
\\
\solution The desired slope is 
\begin{align}
\nabla G(f) &= \dfrac{d(G(f))}{d(\log(f))}
\end{align}

\begin{align}
 \nabla G(f) = 
 \begin{cases} 
        0 & 0 < f < 10^{1} \\
      -20 & 10 < f < 10^{2} \\
      -60 & 10^{2} < f < 10^{3} \\
      -40 & 10^{3} < f < 10^{4} \\
       0 & 10^{4} < f < 10^{5} \\
      -40 & 10^{5} < f < 10^{6} \\
      -60 & 10^{6} < f < 10^{7}   
 \end{cases}
\end{align}

%-----------------------------------------------------------------------%

\item Express the change of slope of $G(f)$ in terms of $f$.
\\
\solution\\
$\Delta(\nabla G(f))$  = Change of slope G(f) at f

\begin{align}
 \Delta(\nabla G(f)) = 
 \begin{cases} 
      -20 &  f = 10^{1} \\
      -40 &  f = 10^{2} \\
      +20 &  f = 10^{3} \\
      +40 &  f = 10^{4} \\
      -40 &  f = 10^{5} \\
      -20 &  f = 10^{6} 
 \end{cases}
\label{eq:ee18btech11001_slope_diff}
\end{align}

%-----------------------------------------------------------------------%

\item Tabulate the poles and zeros of $H(s)$ using \eqref{eq:ee18btech11001_slope_diff}.
\\
\solution Table \ref{table:ee18btech11001} provides the details.  
%
\begin{table}[!ht]
\centering
\begin{enumerate}[label=\thesection.\arabic*.,ref=\thesection.\theenumi]
\numberwithin{equation}{enumi}

\item For an LTI system, the Bode plot for its gain defined as
\begin{align}
	G(s) = 20\log\abs{H(s)}
	\label{eq:ee18btech11001_gain}
\end{align}
is as illustrated in the Fig. \ref{fig:ee18btech11001_bode}. Express $G(f)$ in terms of $f$.\\
\begin{figure}[ht!]
    \includegraphics[width=\columnwidth]{./figs/ee18btech11001/ee18btech11001.eps}
    \caption{}
    \label{fig:ee18btech11001_bode}
\end{figure}\\

\solution
\begin{align}
 G(f) = 
 \begin{cases} 
        100 & 0 < f < 10^{1} \\
      120-20\log(f) & 10 < f < 10^{2} \\
      200-60\log(f) & 10^2 < f < 10^{3} \\
      140-40\log(f) & 10^{3} < f < 10^{4} \\
       -20 & 10^{4} < f < 10^{5} \\
      180-40\log(f) & 10^{5} < f < 10^{6} \\
      300-60\log(f) & 10^{6} < f < 10^{7}   
 \end{cases}
\end{align}

%-----------------------------------------------------------------------%

\item Express the slope of $G(f)$ in terms of $f$.
\\
\solution The desired slope is 
\begin{align}
\nabla G(f) &= \dfrac{d(G(f))}{d(\log(f))}
\end{align}

\begin{align}
 \nabla G(f) = 
 \begin{cases} 
        0 & 0 < f < 10^{1} \\
      -20 & 10 < f < 10^{2} \\
      -60 & 10^{2} < f < 10^{3} \\
      -40 & 10^{3} < f < 10^{4} \\
       0 & 10^{4} < f < 10^{5} \\
      -40 & 10^{5} < f < 10^{6} \\
      -60 & 10^{6} < f < 10^{7}   
 \end{cases}
\end{align}

%-----------------------------------------------------------------------%

\item Express the change of slope of $G(f)$ in terms of $f$.
\\
\solution\\
$\Delta(\nabla G(f))$  = Change of slope G(f) at f

\begin{align}
 \Delta(\nabla G(f)) = 
 \begin{cases} 
      -20 &  f = 10^{1} \\
      -40 &  f = 10^{2} \\
      +20 &  f = 10^{3} \\
      +40 &  f = 10^{4} \\
      -40 &  f = 10^{5} \\
      -20 &  f = 10^{6} 
 \end{cases}
\label{eq:ee18btech11001_slope_diff}
\end{align}

%-----------------------------------------------------------------------%

\item Tabulate the poles and zeros of $H(s)$ using \eqref{eq:ee18btech11001_slope_diff}.
\\
\solution Table \ref{table:ee18btech11001} provides the details.  
%
\begin{table}[!ht]
\centering
\input{./tables/ee18btech11001.tex}
\caption{}
\label{table:ee18btech11001}
\end{table}


%-----------------------------------------------------------------------%

\item Obtain the transfer function of $H(s)$.
\\
\solution From Table \ref{table:ee18btech11001},
{\footnotesize
\begin{align}
\label{eq:ee18btech11001_system}
	H(s) = \frac{K(s+j2\pi 10^{3})(s+j2\pi 10^{4})^{2}}{(s+j2\pi 10^{1})(s+j2\pi 10^{2})^{2}(s+j2\pi 10^{5})^{2}(s+j2\pi 10^{6})}
\end{align}
}
\item Justify the above results.
\\
\solution
Let us consider a generalized transfer gain
\begin{align}
	H(s) = k \dfrac{(s-z_{1})(s-z_{2})...(s-z_{m-1})(s-z_{m})}{(s-p_{1})(s-p_{2})....(s-p_{n-1})(s-p_{n})}
\end{align}
The gain
\begin{multline}
	G(f) = 20\log\abs{H(s)} 
\\
= 20\log \abs{k} + 20\log \abs{s-z_{1}} 
	    \\
	    + 20\log \abs{s-z_{2}} + \dots + 20\log \abs{s-z_{m}} 
\\
- 20\log \abs{s-p_{1}} 
	    - 20\log \abs{s-p_{2}} 
	    \\
- \dots - 20\log \abs{s-z_{n}} 
\end{multline}
%
Substituting $s = \j \omega$, for real $z_1$
\begin{align}
	20\log \abs{s-z_{1}} &= 20\log \abs{\sqrt{\omega^{2} + z_{1}^{2}}}
\\
&= 
\begin{cases}
20 \log \abs{z_{1}}, & \omega \ll z_{1}
\\
20 \log \abs{\omega}, & \omega \gg z_{1}
\end{cases}
\end{align}
%
Taking the derivative, 
\begin{align}
	\frac{d\brak{20\log \abs{s-z_{1}}}}{d\brak{\log \abs{\omega}}} 
= 
\begin{cases}
0, & \omega \ll z_{1}
\\
20, & \omega \gg z_{1}
\end{cases}
\end{align}
%
Thus, when a zero is encountered, the gradient of $H(\j\omega)$ jumps by +20 in the log scale.  When a pole is encountered, the gradient falls by -20. Note that this is a very loose justification, but works well in practice.

%-----------------------------------------------------------------------%

\item Obtain the Bode plot and the slope plot for $H(s)$ and verify with  Fig. \ref{fig:ee18btech11001_bode}
\\
\solution Bode Plot of obtained Transfer Function is 
\begin{figure}[htp]
    \centering
    \includegraphics[width=\columnwidth]{./figs/ee18btech11001/ee18btech11001_2.eps}
    \caption{}
    \label{fig:ee18btech11001_2}
\end{figure}
%
Fig. \ref{fig:ee18btech11001}, obtained from  \eqref{eq:ee18btech11001_system},
is a close reconstruction of Fig. \ref{fig:ee18btech11001}.
\end{enumerate}



\caption{}
\label{table:ee18btech11001}
\end{table}


%-----------------------------------------------------------------------%

\item Obtain the transfer function of $H(s)$.
\\
\solution From Table \ref{table:ee18btech11001},
{\footnotesize
\begin{align}
\label{eq:ee18btech11001_system}
	H(s) = \frac{K(s+j2\pi 10^{3})(s+j2\pi 10^{4})^{2}}{(s+j2\pi 10^{1})(s+j2\pi 10^{2})^{2}(s+j2\pi 10^{5})^{2}(s+j2\pi 10^{6})}
\end{align}
}
\item Justify the above results.
\\
\solution
Let us consider a generalized transfer gain
\begin{align}
	H(s) = k \dfrac{(s-z_{1})(s-z_{2})...(s-z_{m-1})(s-z_{m})}{(s-p_{1})(s-p_{2})....(s-p_{n-1})(s-p_{n})}
\end{align}
The gain
\begin{multline}
	G(f) = 20\log\abs{H(s)} 
\\
= 20\log \abs{k} + 20\log \abs{s-z_{1}} 
	    \\
	    + 20\log \abs{s-z_{2}} + \dots + 20\log \abs{s-z_{m}} 
\\
- 20\log \abs{s-p_{1}} 
	    - 20\log \abs{s-p_{2}} 
	    \\
- \dots - 20\log \abs{s-z_{n}} 
\end{multline}
%
Substituting $s = \j \omega$, for real $z_1$
\begin{align}
	20\log \abs{s-z_{1}} &= 20\log \abs{\sqrt{\omega^{2} + z_{1}^{2}}}
\\
&= 
\begin{cases}
20 \log \abs{z_{1}}, & \omega \ll z_{1}
\\
20 \log \abs{\omega}, & \omega \gg z_{1}
\end{cases}
\end{align}
%
Taking the derivative, 
\begin{align}
	\frac{d\brak{20\log \abs{s-z_{1}}}}{d\brak{\log \abs{\omega}}} 
= 
\begin{cases}
0, & \omega \ll z_{1}
\\
20, & \omega \gg z_{1}
\end{cases}
\end{align}
%
Thus, when a zero is encountered, the gradient of $H(\j\omega)$ jumps by +20 in the log scale.  When a pole is encountered, the gradient falls by -20. Note that this is a very loose justification, but works well in practice.

%-----------------------------------------------------------------------%

\item Obtain the Bode plot and the slope plot for $H(s)$ and verify with  Fig. \ref{fig:ee18btech11001_bode}
\\
\solution Bode Plot of obtained Transfer Function is 
\begin{figure}[htp]
    \centering
    \includegraphics[width=\columnwidth]{./figs/ee18btech11001/ee18btech11001_2.eps}
    \caption{}
    \label{fig:ee18btech11001_2}
\end{figure}
%
Fig. \ref{fig:ee18btech11001}, obtained from  \eqref{eq:ee18btech11001_system},
is a close reconstruction of Fig. \ref{fig:ee18btech11001}.
\end{enumerate}



\caption{}
\label{table:ee18btech11001}
\end{table}


%-----------------------------------------------------------------------%

\item Obtain the transfer function of $H(s)$.
\\
\solution From Table \ref{table:ee18btech11001},
{\footnotesize
\begin{align}
\label{eq:ee18btech11001_system}
	H(s) = \frac{K(s+j2\pi 10^{3})(s+j2\pi 10^{4})^{2}}{(s+j2\pi 10^{1})(s+j2\pi 10^{2})^{2}(s+j2\pi 10^{5})^{2}(s+j2\pi 10^{6})}
\end{align}
}
\item Justify the above results.
\\
\solution
Let us consider a generalized transfer gain
\begin{align}
	H(s) = k \dfrac{(s-z_{1})(s-z_{2})...(s-z_{m-1})(s-z_{m})}{(s-p_{1})(s-p_{2})....(s-p_{n-1})(s-p_{n})}
\end{align}
The gain
\begin{multline}
	G(f) = 20\log\abs{H(s)} 
\\
= 20\log \abs{k} + 20\log \abs{s-z_{1}} 
	    \\
	    + 20\log \abs{s-z_{2}} + \dots + 20\log \abs{s-z_{m}} 
\\
- 20\log \abs{s-p_{1}} 
	    - 20\log \abs{s-p_{2}} 
	    \\
- \dots - 20\log \abs{s-z_{n}} 
\end{multline}
%
Substituting $s = \j \omega$, for real $z_1$
\begin{align}
	20\log \abs{s-z_{1}} &= 20\log \abs{\sqrt{\omega^{2} + z_{1}^{2}}}
\\
&= 
\begin{cases}
20 \log \abs{z_{1}}, & \omega \ll z_{1}
\\
20 \log \abs{\omega}, & \omega \gg z_{1}
\end{cases}
\end{align}
%
Taking the derivative, 
\begin{align}
	\frac{d\brak{20\log \abs{s-z_{1}}}}{d\brak{\log \abs{\omega}}} 
= 
\begin{cases}
0, & \omega \ll z_{1}
\\
20, & \omega \gg z_{1}
\end{cases}
\end{align}
%
Thus, when a zero is encountered, the gradient of $H(\j\omega)$ jumps by +20 in the log scale.  When a pole is encountered, the gradient falls by -20. Note that this is a very loose justification, but works well in practice.

%-----------------------------------------------------------------------%

\item Obtain the Bode plot and the slope plot for $H(s)$ and verify with  Fig. \ref{fig:ee18btech11001_bode}
\\
\solution Bode Plot of obtained Transfer Function is 
\begin{figure}[htp]
    \centering
    \includegraphics[width=\columnwidth]{./figs/ee18btech11001/ee18btech11001_2.eps}
    \caption{}
    \label{fig:ee18btech11001_2}
\end{figure}
%
Fig. \ref{fig:ee18btech11001}, obtained from  \eqref{eq:ee18btech11001_system},
is a close reconstruction of Fig. \ref{fig:ee18btech11001}.
\end{enumerate}



\subsection{Example}
\begin{enumerate}[label=\thesubsection.\arabic*.,ref=\thesubsection.\theenumi]
\numberwithin{equation}{enumi} 
\item 
The asymptotic Bode magnitude plot of  minimum phase transfer function
G(s) is show below.
\begin{figure}[htp]
	\centering
	\includegraphics[width=1 \columnwidth]{./figs/ee18btech11009/pppp.eps}
	\caption{}
	\label{fig:ee18btech11009_bode}
\end{figure} 
%
Express $20\log\abs{G(\j\omega)}$ as a function of $\omega$ using Fig. \ref{fig:ee18btech11009_bode}.
\label{prob:ee18btech11009_bode}
\item Express the slope of $20\log\abs{G(\j\omega)}$ as a function of $\omega$. 
\item Express the change of slope of $20\log\abs{G(\j\omega)}$ as a function of $\omega$. 
\item Find the poles and zeros of $G(s)$.
\item Find $G(s)$
\item Obtain the Bode plot of $G(s)$ through a python code and compare with the line plot of the expression that you obtained in Problem \ref{prob:ee18btech11009_bode}
\item  Verify if at very high frequency $(\omega \to \infty)$, the phase angle $ \angle G(j\omega)=-3\pi/2$
\solution

Since, each pole corresponds to -20 dB/decade  
and each zero corresponds to +20 dB/decade.\\
Therefore, from the given Bode plot we can get the Transfer equation,
\begin{align}
G(s) = \frac{k}{s(1+s)(20+s)}
\end{align}

Now, from the Transfer equation we can conclude that,
there are three poles (0, -1 and -20 ) and no zeros.\\

$\therefore$ Statement 1 is false  ..........(1)\\ \\

%------------------------------------------------

\textbf{Calculating phase:}\\ \\
Since we know that,\\
phase $ \phi $ is the sum of all the phases corresponding to each pole and zero.\\
phase corresponding to pole is =  
\begin{align}
-tan^{-1}( \frac{imaginary}{real})
\end{align} 
phase corresponding to zero is =
\begin{align}
 tan^{-1}( \frac{imaginary}{real})
 \end{align} 
%------------------------------------------------
Now take,
\begin{align}
 s = j\omega
  \end{align} 
  \begin{align}
 \Rightarrow  G(j\omega) =  \frac{k}{j\omega(1+j\omega)(20+j\omega)}
 \end{align} 
Therefore, 
\begin{align}
 \phi =  -tan^{-1}( {\frac{\omega}{0}}) - tan^{-1}(\omega) - tan^{-1}( \frac{\omega}{20})
 \end{align} 
 \begin{align}
  \phi =  - 90^\circ - tan^{-1}(\omega) - tan^{-1}( \frac{\omega}{20})
  \end{align} 
  \begin{align}
  \because \omega \to \infty
 \end{align} 
 \begin{align}
   \phi =   - 90^\circ - 90^\circ - 90^\circ
   \end{align} 
   \begin{align}
 \phi = -270^\circ
 \end{align} 
 \begin{align}
 \phi = -3\pi/2 
 \end{align} 
 $\therefore$ Statement 2 is true ........(2)\\
 thus, from (1) and (2) option (B) is correct.
 \\
 \\
 \item
%\begin{flushleft}

 \end{enumerate}

\section{Second order System}
\subsection{Damping}
\begin{enumerate}[label=\thesubsection.\arabic*.,ref=\thesubsection.\theenumi]
\numberwithin{equation}{enumi}
\item List the different kinds of damping for a second order system defined by 
\begin{align}
\label{eq:ee18btech11012_second}
H(s)=\frac{\omega^2}{s^2+2\zeta\omega+\omega^2}
\end{align}
where $\omega$ is  the natural   frequency  and $ \zeta $  is the  damping factor.
%
\\
\solution The details are available in Table \ref{table:ee18btech11012}
\begin{table}[!ht]
\centering
\begin{enumerate}[label=\thesubsection.\arabic*.,ref=\thesubsection.\theenumi]
\numberwithin{equation}{enumi}
\item List the different kinds of damping for a second order system defined by 
\begin{align}
\label{eq:ee18btech11012_second}
H(s)=\frac{\omega^2}{s^2+2\zeta\omega+\omega^2}
\end{align}
where $\omega$ is  the natural   frequency  and $ \zeta $  is the  damping factor.
%
\\
\solution The details are available in Table \ref{table:ee18btech11012}
\begin{table}[!ht]
\centering
\begin{enumerate}[label=\thesubsection.\arabic*.,ref=\thesubsection.\theenumi]
\numberwithin{equation}{enumi}
\item List the different kinds of damping for a second order system defined by 
\begin{align}
\label{eq:ee18btech11012_second}
H(s)=\frac{\omega^2}{s^2+2\zeta\omega+\omega^2}
\end{align}
where $\omega$ is  the natural   frequency  and $ \zeta $  is the  damping factor.
%
\\
\solution The details are available in Table \ref{table:ee18btech11012}
\begin{table}[!ht]
\centering
\input{./tables/ee18btech11012.tex}
\caption{}
\label{table:ee18btech11012}
\end{table}

\item Classify the following second-order systems according to damping.
\label{prob:ee18btech11012_damp}
\begin{enumerate}
\item $H(s) = \frac{15}{{s^2+5s+15}}$ 
\item $H(s) = \frac{25}{{s^2+10s+25}}$
\item $H(s) =\frac{35}{{s^2+18s+35}}$ 
\end{enumerate}
\solution For 
\begin{align}
H(s) &= \frac{25}{{s^2+10s+25}},
\\
     \omega^2 &= 25,   2\zeta\omega =10\\
\implies  \omega &=1,  {\zeta} = 1
\end{align}
and the system is critically damped.  Similarly, the damping factors for other systems in Problem \ref{prob:ee18btech11012_damp} are calculated and listed in Table \ref{table:ee18btech11012_damp}
%
\begin{table}[!ht]
\centering
\input{./tables/ee18btech11012_damp.tex}
\caption{}
\label{table:ee18btech11012_damp}
\end{table}

\item By choosing an appropriate input, illustrate the effect of damping using a Python code to sketch the response.  
\end{enumerate}

\caption{}
\label{table:ee18btech11012}
\end{table}

\item Classify the following second-order systems according to damping.
\label{prob:ee18btech11012_damp}
\begin{enumerate}
\item $H(s) = \frac{15}{{s^2+5s+15}}$ 
\item $H(s) = \frac{25}{{s^2+10s+25}}$
\item $H(s) =\frac{35}{{s^2+18s+35}}$ 
\end{enumerate}
\solution For 
\begin{align}
H(s) &= \frac{25}{{s^2+10s+25}},
\\
     \omega^2 &= 25,   2\zeta\omega =10\\
\implies  \omega &=1,  {\zeta} = 1
\end{align}
and the system is critically damped.  Similarly, the damping factors for other systems in Problem \ref{prob:ee18btech11012_damp} are calculated and listed in Table \ref{table:ee18btech11012_damp}
%
\begin{table}[!ht]
\centering
\input{./tables/ee18btech11012_damp.tex}
\caption{}
\label{table:ee18btech11012_damp}
\end{table}

\item By choosing an appropriate input, illustrate the effect of damping using a Python code to sketch the response.  
\end{enumerate}

\caption{}
\label{table:ee18btech11012}
\end{table}

\item Classify the following second-order systems according to damping.
\label{prob:ee18btech11012_damp}
\begin{enumerate}
\item $H(s) = \frac{15}{{s^2+5s+15}}$ 
\item $H(s) = \frac{25}{{s^2+10s+25}}$
\item $H(s) =\frac{35}{{s^2+18s+35}}$ 
\end{enumerate}
\solution For 
\begin{align}
H(s) &= \frac{25}{{s^2+10s+25}},
\\
     \omega^2 &= 25,   2\zeta\omega =10\\
\implies  \omega &=1,  {\zeta} = 1
\end{align}
and the system is critically damped.  Similarly, the damping factors for other systems in Problem \ref{prob:ee18btech11012_damp} are calculated and listed in Table \ref{table:ee18btech11012_damp}
%
\begin{table}[!ht]
\centering
\input{./tables/ee18btech11012_damp.tex}
\caption{}
\label{table:ee18btech11012_damp}
\end{table}

\item By choosing an appropriate input, illustrate the effect of damping using a Python code to sketch the response.  
\end{enumerate}

\subsection{Example}
\begin{enumerate}[label=\thesection.\arabic*.,ref=\thesection.\theenumi]
\numberwithin{equation}{enumi}
\item
Consider the following second order system with the transfer function
\begin{align}
G(s) = \frac{1}{1+2s+s^2}
\end{align}
Is the system stable? 
\\
\solution The poles of 
\begin{align}
G(s) = \frac{1}{1+2s+s^2}
\end{align}
are at 
\begin{align}
s = -1
\end{align}
i.e.,  the left half of s-plane.  Hence the system is stable.
\item Find and sketch the step response $c(t)$ of the system.
\\
\solution 
For step-response, we take input as unit-step function u(t)
\begin{align}
C(s) &= U(s).G(s) = \sbrak{\frac{1}{s}} \sbrak{\frac{1}{1+2s+s^2}}
\\
&= \frac{1}{s(1+s)^2}
\\
&= \frac{1}{s} - \frac{1}{(1+s)} - \frac{1}{(1+s)^2}
\end{align}
%
Taking the inverse Laplace transform,
%
\begin{align}
c(t) &= L^{-1} \sbrak{ \frac{1}{s}} - L^{-1}\sbrak{\frac{1}{1+s}} - L^{-1}\sbrak{\frac{1}{(1+s)^2}} 
\\
&= \brak{1 - e^{-t} - te^{-t}}  u(t)
\label{eq:sec_order_op}
\end{align}
%
The following code plots $c(t)$ in Fig. \ref{fig:sec_order}
\begin{lstlisting}
codes/ee18btech11002/plot.py
\end{lstlisting}
\begin{figure}
\centering
\includegraphics[width=\columnwidth]{./figs/ee18btech11002.eps}
\caption{}
\label{fig:sec_order}
\end{figure}
\item Find the steady state response of the system using the final value theorem.  Verify using 
\ref{eq:sec_order_op}
\\
\solution 
To know the steady response value of c(t), using final value theorem,
\begin{align}
\lim_{t\to\infty} c(t) = \lim_{s\to 0} sC(s) 
\end{align}
We get
\begin{align}
\lim_{s\to 0} s \brak{\frac{1}{s}}\brak{\frac{1}{1+s+s^2}} = \frac{1}{1+0+0} = 1
\end{align}
Using \ref{eq:sec_order_op}, 
\begin{align}
\lim_{t\to\infty} c(t) &=\lim_{t\to\infty}\brak{1 - e^{-t} - te^{-t}}  u(t) 
\\
&=(1-0-0) = 1 
\end{align}
%
\item Find the time taken for the system output c(t) to reach 94\% of its steady state value.
\\
\solution 
Now, 94\% of 1 is 0.94, so we should now solve for a positive t such that
\begin{align}
1 - e^{-t} - te^{-t} = 0.94
\end{align}
The following code 
%
\begin{lstlisting}
codes/ee18btech11002/solution.py
\end{lstlisting}
%\lstinputlisting{./}
provides the necessary solution as 
%the attached code gives us the solution for the equation
%and t turns out to be
\begin{align}
 t = 4.5228
\end{align}

\end{enumerate}

\section{Routh Hurwitz Criterion}
\subsection{Routh Array}
\begin{enumerate}[label=\thesubsection.\arabic*.,ref=\thesubsection.\theenumi]
\numberwithin{equation}{enumi}

\item Generate the Routh array for the 
polynomial, 
\begin{align}
f(s)=s^{7}+s^{6}+7 s^{5}+14 s^{4}+31s^{3}+73 s^{2}+25 s+ 200
\label{eq:ee18btech11014_routh_poly}
\end{align}
\solution 
\begin{align}
\mydet{s^7\\s^6\\s^5}
\mydet{1 & 7 & 31 & 25 \\ 1 & 14 & 73 & 200 \\ -7 & -42 & -175 & 0}
\end{align}\\
\begin{align}
\mydet{s^7\\s^6\\s^5\\s^4}
\mydet{1 & 7 & 31 & 25 \\ 1 & 14 & 73 & 200 \\ -7 & -42 & -175 & 0 \\ 8 & 48 & 200 & 0}
\end{align}\\
\begin{align}
\mydet{s^7\\s^6\\s^5\\s^4\\s^3}
\mydet{1 & 7 & 31 & 25 \\ 1 & 14 & 73 & 200 \\ -7 & -42 & -175 & 0 \\ 8 & 48 & 200 & 0 \\ 0 & 0 & 0 &  }
\end{align}\\

When such a case is encountered, we take the derivative of the expression formed the the coefficients above it i.e derivative of $8s^4 + 48s^2 +200$.
\begin{center}
    $\frac{d}{dx}(8s^4 + 48s^2 +200) = 32s^3 + 96s$
\end{center}

The coefficients of obtained expression are placed in the table.

\begin{align}
\mydet{s^7\\s^6\\s^5\\s^4\\s^3}
\mydet{1 & 7 & 31 & 25 \\ 1 & 14 & 73 & 200 \\ -7 & -42 & -175 & 0 \\ 8 & 48 & 200 & 0 \\ 32 & 96 & 0 &  }
\end{align}\\
\begin{align}
\mydet{s^7\\s^6\\s^5\\s^4\\s^3\\s^2}
\mydet{1 & 7 & 31 & 25 \\ 1 & 14 & 73 & 200 \\ -7 & -42 & -175 & 0 \\ 8 & 48 & 200 & 0 \\ 32 & 96 & 0 &  \\ 24 & 200 & 0 &  }
\end{align}\\
\begin{align}
\mydet{s^7\\s^6\\s^5\\s^4\\s^3\\s^2\\s^1}
\mydet{1 & 7 & 31 & 25 \\ 1 & 14 & 73 & 200 \\ -7 & -42 & -175 & 0 \\ 8 & 48 & 200 & 0 \\ 32 & 96 & 0 &  \\ 24 & 200 & 0 &  \\ -170.67 & 0 &  & }
\end{align}\\
\begin{align}
\mydet{s^7\\s^6\\s^5\\s^4\\s^3\\s^2\\s^1\\s^0}
\mydet{1 & 7 & 31 & 25 \\ 1 & 14 & 73 & 200 \\ -7 & -42 & -175 & 0 \\ 8 & 48 & 200 & 0 \\ 32 & 96 & 0 &  \\ 24 & 200 & 0 &  \\ -170.67 & 0 &  & \\200 &   &   & }
\label{eq:ee18btech11014_routh_final}
\end{align}

So, the above one is the Routh-Hurwitz Table.

\item Find the number of roots of the polynomial in the right half of the $s$-plane.
\\
\solution The number of roots of the polynomial that are in the right half-plane is equal to
the number of sign changes in the first column. From \ref{eq:ee18btech11014_routh_final},
the polynomial in \eqref{eq:ee18btech11014_routh_poly}
has 4 roots lie on right-side of Imaginary Axis.

\item Write a Python code for generating each stage of the Routh Table.
\\
\solution The following code 
\begin{lstlisting}
codes/ee18btech11014/ee18btech11014.py
\end{lstlisting}
%
generates the various stages.
%The below image is the location of Roots of given polynomial.\\
%%\includegraphics[scale=0.15]{figs/Roots.eps}\\
%
%The below image is finding out the roots of given polynomial by substituting $z=x+iy$ and obtaining the Real and Imaginary Parts of f(z). The values of $z$ where Real and Imaginary parts f(z) becomes zero simultaneously are roots of f(z).\\
%%\includegraphics[scale=0.15]{figs/f(z).png}

\item Find the roots of the polynomial in in \eqref{eq:ee18btech11014_routh_poly} and verify that 4 roots are in the right half $s$-plane.
\\
\solution The following code generates the necessary roots.
\begin{lstlisting}
codes/ee18btech11014/Roots.py
\end{lstlisting}
\end{enumerate}

\subsection{Marginal Stability}
\begin{enumerate}[label=\thesubsection.\arabic*.,ref=\thesubsection.\theenumi]
\numberwithin{equation}{enumi}

\item
Consider a unity feedback system as shown in Fig.  \ref{fig:ee18btech11005}, with an integral compensator $\frac{k}{s}$ and open-loop transfer function
\begin{align}
G(s) = \frac{1}{s^2+3s+2}
\end{align}
where k greater than 0. 
%
Find its closed loop transfer function.
\begin{figure}[!ht]
	\begin{center}
		
		\resizebox{\columnwidth}{!}{\begin{enumerate}[label=\thesubsection.\arabic*.,ref=\thesubsection.\theenumi]
\numberwithin{equation}{enumi}

\item
Consider a unity feedback system as shown in Fig.  \ref{fig:ee18btech11005}, with an integral compensator $\frac{k}{s}$ and open-loop transfer function
\begin{align}
G(s) = \frac{1}{s^2+3s+2}
\end{align}
where k greater than 0. 
%
Find its closed loop transfer function.
\begin{figure}[!ht]
	\begin{center}
		
		\resizebox{\columnwidth}{!}{\begin{enumerate}[label=\thesubsection.\arabic*.,ref=\thesubsection.\theenumi]
\numberwithin{equation}{enumi}

\item
Consider a unity feedback system as shown in Fig.  \ref{fig:ee18btech11005}, with an integral compensator $\frac{k}{s}$ and open-loop transfer function
\begin{align}
G(s) = \frac{1}{s^2+3s+2}
\end{align}
where k greater than 0. 
%
Find its closed loop transfer function.
\begin{figure}[!ht]
	\begin{center}
		
		\resizebox{\columnwidth}{!}{\input{./figs/ee18btech11005.tex}}
	\end{center}
\caption{}
\label{fig:ee18btech11005}
\end{figure}

\solution $\because H(s) = 1$ in Fig.  \ref{fig:ee18btech11005}, due to unity feedback,   the transfer function is given by
\begin{align}
\frac{Y(s)}{X(s)} &= \frac{G(s)}{1+G(s)H(s)}
\\
\implies T(s) &= \frac{k}{s^3+3s^2+2s}
\end{align}
%
\item Find the {\em characteristic} equation for $G(s)$.
\\
\solution The characteristic equation is
\begin{align}
\label{eq:routh_char_eq}
 1 + G(s)H(s) &= 0 
\\
\implies 1 + \sbrak{\frac{k}{s^3+3s^2+2s}} &= 0
\\
\text{or, } s^3+3s^2+2s+k &= 0
\end{align}
\item Using the tabular method for the Routh hurwitz criterion, find $k > 0$ for which there are two poles of unity feedback system on j${\omega}$ axis.
%
\\
\solution 
This criterion is based on arranging the coefficients of characteristic equation into an array called Routh array.
For any characteristic equation 
\begin{multline}
q(s) = a_0s^n+a_1s^{n-1}+.....+a_{n-1}s+a_n = 0
\end{multline}
the Routh array can be constructed as 
 
\begin{align}
\mydet{s^n\\s^{n-1}\\s^{n-2} \\ \vdots}
 \mydet{a_0 & a_2 & a_4 & \cdots \\
a_1 & a_3 & a_5 & \cdots  \\
b_1 & b_2 & b_3 & \cdots \\
\vdots & \vdots & \vdots & \ddots &\vdots 
 \cdots \\}
\end{align}
%
 where
 \begin{align}
 b_1 =\frac{ a_1a_2-a_0a_3}{a_1}  
 \\
 b_2 =\frac{ a_1a_4-a_0a_5}{a_1} 
 \\
 c_1=\frac{ b_1a_3-a_1b_2}{b_1} 
\\
 c_2=\frac{ b_1a_5-a_1b_3}{b_1}  
\end{align}
For poles to lie on imaginary axis any one entire row of hurwitz matrix should be zero.
Constructing the routh array for the characteristic equation obtained in \ref{eq:routh_char_eq},
%
\begin{align}
 s^3+3s^2+2s+k = 0
\end{align}
%
\begin{align}
\mydet{s^3\\s^2\\s^1 \\ s^0}
\mydet{1 & 2 \\ 3 & k \\  \frac{6-k}{3} & 0\\ k & 0}
\end{align}
For poles on $\j \omega$ axis any one of the row should be zero.
%
\begin{align}
\therefore \frac{6-k}{3} &= 0 \text{ or } k = 0
\\
\implies k &= 6 \quad \because k > 0
\end{align}
\item Repeat the above using the determinant method.
\\
\solution The {\em Routh matrix} can be expressed as
\begin{align}
\vec{R} = \myvec{a_0 & a_2 & a_4 & \cdots \\
a_1 & a_3 & a_5 & \cdots  \\
 0 & a_0 & a_2\cdots \\
 0 & a_1 & a_3 \cdots\\
\vdots & \vdots & \vdots & \ddots &\vdots 
\cdots \\}
\end{align}
and the corresponding Routh determinants are
\begin{align}
D_1 &= |a_0|
\\
D_2 &= 
\mydet{
a_0 & a_2 
\\ 
a_1 & a_3
} 
\\
D_3 &=\mydet{
a_0 & a_2 & a_4 
\\ a_1 & a_3 & a_5 
\\ 0 & a_0 & a_2}
\\
\dots
\end{align}
If at least any one of the Determinents are zero then the poles lie on imaginary axes.  From \eqref{eq:routh_char_eq},
%
\begin{align}
D_1 &= 1 \ne 0
\\
D2 &= \mydet{
1 & 2 \\ 3 & k } 
&= k-6 =0 \implies k = 6
\end{align}
%
\item Verify your answer using a python code for both the determinant method as well as the tabular method.
\label{prob:ee18btech11005_python}
\\
\solution 
The following code verifies the stability using the tabular method 
%
\begin{lstlisting}
codes/ee18btech11005_2.py
\end{lstlisting}
and the following one verifies using the determinant method.
\begin{lstlisting}
codes/ee18btech11005.py
\end{lstlisting}
%
provides the necessary soution.
\begin{itemize}
\item  For the system to be stable all coefficients should lie on left half of s-plane. Because if any pole is in right half of s-plane then there will be a component in output that increases without bound,causing system to be unstable.
All the coefficients in the characteristic equation should be positive.This is necessary condition but not sufficient.Because it may have poles on right half of s plane.
Poles are the roots of the characteristic equation.
    \item A system is stable if all of its characteristic modes go to finite value as t goes to infinity.It is possible only if all the poles are on the left half of s plane.
    The characteristic equation should have negative roots only. So the first column should always be greater than zero.That means no sign changes.
    \item A system is unstable if its characteristic modes are not bounded. Then the characteristic equation will also have roots in the right side of s-plane.That means it has sign changes.
    \end{itemize}

\end{enumerate}


}
	\end{center}
\caption{}
\label{fig:ee18btech11005}
\end{figure}

\solution $\because H(s) = 1$ in Fig.  \ref{fig:ee18btech11005}, due to unity feedback,   the transfer function is given by
\begin{align}
\frac{Y(s)}{X(s)} &= \frac{G(s)}{1+G(s)H(s)}
\\
\implies T(s) &= \frac{k}{s^3+3s^2+2s}
\end{align}
%
\item Find the {\em characteristic} equation for $G(s)$.
\\
\solution The characteristic equation is
\begin{align}
\label{eq:routh_char_eq}
 1 + G(s)H(s) &= 0 
\\
\implies 1 + \sbrak{\frac{k}{s^3+3s^2+2s}} &= 0
\\
\text{or, } s^3+3s^2+2s+k &= 0
\end{align}
\item Using the tabular method for the Routh hurwitz criterion, find $k > 0$ for which there are two poles of unity feedback system on j${\omega}$ axis.
%
\\
\solution 
This criterion is based on arranging the coefficients of characteristic equation into an array called Routh array.
For any characteristic equation 
\begin{multline}
q(s) = a_0s^n+a_1s^{n-1}+.....+a_{n-1}s+a_n = 0
\end{multline}
the Routh array can be constructed as 
 
\begin{align}
\mydet{s^n\\s^{n-1}\\s^{n-2} \\ \vdots}
 \mydet{a_0 & a_2 & a_4 & \cdots \\
a_1 & a_3 & a_5 & \cdots  \\
b_1 & b_2 & b_3 & \cdots \\
\vdots & \vdots & \vdots & \ddots &\vdots 
 \cdots \\}
\end{align}
%
 where
 \begin{align}
 b_1 =\frac{ a_1a_2-a_0a_3}{a_1}  
 \\
 b_2 =\frac{ a_1a_4-a_0a_5}{a_1} 
 \\
 c_1=\frac{ b_1a_3-a_1b_2}{b_1} 
\\
 c_2=\frac{ b_1a_5-a_1b_3}{b_1}  
\end{align}
For poles to lie on imaginary axis any one entire row of hurwitz matrix should be zero.
Constructing the routh array for the characteristic equation obtained in \ref{eq:routh_char_eq},
%
\begin{align}
 s^3+3s^2+2s+k = 0
\end{align}
%
\begin{align}
\mydet{s^3\\s^2\\s^1 \\ s^0}
\mydet{1 & 2 \\ 3 & k \\  \frac{6-k}{3} & 0\\ k & 0}
\end{align}
For poles on $\j \omega$ axis any one of the row should be zero.
%
\begin{align}
\therefore \frac{6-k}{3} &= 0 \text{ or } k = 0
\\
\implies k &= 6 \quad \because k > 0
\end{align}
\item Repeat the above using the determinant method.
\\
\solution The {\em Routh matrix} can be expressed as
\begin{align}
\vec{R} = \myvec{a_0 & a_2 & a_4 & \cdots \\
a_1 & a_3 & a_5 & \cdots  \\
 0 & a_0 & a_2\cdots \\
 0 & a_1 & a_3 \cdots\\
\vdots & \vdots & \vdots & \ddots &\vdots 
\cdots \\}
\end{align}
and the corresponding Routh determinants are
\begin{align}
D_1 &= |a_0|
\\
D_2 &= 
\mydet{
a_0 & a_2 
\\ 
a_1 & a_3
} 
\\
D_3 &=\mydet{
a_0 & a_2 & a_4 
\\ a_1 & a_3 & a_5 
\\ 0 & a_0 & a_2}
\\
\dots
\end{align}
If at least any one of the Determinents are zero then the poles lie on imaginary axes.  From \eqref{eq:routh_char_eq},
%
\begin{align}
D_1 &= 1 \ne 0
\\
D2 &= \mydet{
1 & 2 \\ 3 & k } 
&= k-6 =0 \implies k = 6
\end{align}
%
\item Verify your answer using a python code for both the determinant method as well as the tabular method.
\label{prob:ee18btech11005_python}
\\
\solution 
The following code verifies the stability using the tabular method 
%
\begin{lstlisting}
codes/ee18btech11005_2.py
\end{lstlisting}
and the following one verifies using the determinant method.
\begin{lstlisting}
codes/ee18btech11005.py
\end{lstlisting}
%
provides the necessary soution.
\begin{itemize}
\item  For the system to be stable all coefficients should lie on left half of s-plane. Because if any pole is in right half of s-plane then there will be a component in output that increases without bound,causing system to be unstable.
All the coefficients in the characteristic equation should be positive.This is necessary condition but not sufficient.Because it may have poles on right half of s plane.
Poles are the roots of the characteristic equation.
    \item A system is stable if all of its characteristic modes go to finite value as t goes to infinity.It is possible only if all the poles are on the left half of s plane.
    The characteristic equation should have negative roots only. So the first column should always be greater than zero.That means no sign changes.
    \item A system is unstable if its characteristic modes are not bounded. Then the characteristic equation will also have roots in the right side of s-plane.That means it has sign changes.
    \end{itemize}

\end{enumerate}


}
	\end{center}
\caption{}
\label{fig:ee18btech11005}
\end{figure}

\solution $\because H(s) = 1$ in Fig.  \ref{fig:ee18btech11005}, due to unity feedback,   the transfer function is given by
\begin{align}
\frac{Y(s)}{X(s)} &= \frac{G(s)}{1+G(s)H(s)}
\\
\implies T(s) &= \frac{k}{s^3+3s^2+2s}
\end{align}
%
\item Find the {\em characteristic} equation for $G(s)$.
\\
\solution The characteristic equation is
\begin{align}
\label{eq:routh_char_eq}
 1 + G(s)H(s) &= 0 
\\
\implies 1 + \sbrak{\frac{k}{s^3+3s^2+2s}} &= 0
\\
\text{or, } s^3+3s^2+2s+k &= 0
\end{align}
\item Using the tabular method for the Routh hurwitz criterion, find $k > 0$ for which there are two poles of unity feedback system on j${\omega}$ axis.
%
\\
\solution 
This criterion is based on arranging the coefficients of characteristic equation into an array called Routh array.
For any characteristic equation 
\begin{multline}
q(s) = a_0s^n+a_1s^{n-1}+.....+a_{n-1}s+a_n = 0
\end{multline}
the Routh array can be constructed as 
 
\begin{align}
\mydet{s^n\\s^{n-1}\\s^{n-2} \\ \vdots}
 \mydet{a_0 & a_2 & a_4 & \cdots \\
a_1 & a_3 & a_5 & \cdots  \\
b_1 & b_2 & b_3 & \cdots \\
\vdots & \vdots & \vdots & \ddots &\vdots 
 \cdots \\}
\end{align}
%
 where
 \begin{align}
 b_1 =\frac{ a_1a_2-a_0a_3}{a_1}  
 \\
 b_2 =\frac{ a_1a_4-a_0a_5}{a_1} 
 \\
 c_1=\frac{ b_1a_3-a_1b_2}{b_1} 
\\
 c_2=\frac{ b_1a_5-a_1b_3}{b_1}  
\end{align}
For poles to lie on imaginary axis any one entire row of hurwitz matrix should be zero.
Constructing the routh array for the characteristic equation obtained in \ref{eq:routh_char_eq},
%
\begin{align}
 s^3+3s^2+2s+k = 0
\end{align}
%
\begin{align}
\mydet{s^3\\s^2\\s^1 \\ s^0}
\mydet{1 & 2 \\ 3 & k \\  \frac{6-k}{3} & 0\\ k & 0}
\end{align}
For poles on $\j \omega$ axis any one of the row should be zero.
%
\begin{align}
\therefore \frac{6-k}{3} &= 0 \text{ or } k = 0
\\
\implies k &= 6 \quad \because k > 0
\end{align}
\item Repeat the above using the determinant method.
\\
\solution The {\em Routh matrix} can be expressed as
\begin{align}
\vec{R} = \myvec{a_0 & a_2 & a_4 & \cdots \\
a_1 & a_3 & a_5 & \cdots  \\
 0 & a_0 & a_2\cdots \\
 0 & a_1 & a_3 \cdots\\
\vdots & \vdots & \vdots & \ddots &\vdots 
\cdots \\}
\end{align}
and the corresponding Routh determinants are
\begin{align}
D_1 &= |a_0|
\\
D_2 &= 
\mydet{
a_0 & a_2 
\\ 
a_1 & a_3
} 
\\
D_3 &=\mydet{
a_0 & a_2 & a_4 
\\ a_1 & a_3 & a_5 
\\ 0 & a_0 & a_2}
\\
\dots
\end{align}
If at least any one of the Determinents are zero then the poles lie on imaginary axes.  From \eqref{eq:routh_char_eq},
%
\begin{align}
D_1 &= 1 \ne 0
\\
D2 &= \mydet{
1 & 2 \\ 3 & k } 
&= k-6 =0 \implies k = 6
\end{align}
%
\item Verify your answer using a python code for both the determinant method as well as the tabular method.
\label{prob:ee18btech11005_python}
\\
\solution 
The following code verifies the stability using the tabular method 
%
\begin{lstlisting}
codes/ee18btech11005_2.py
\end{lstlisting}
and the following one verifies using the determinant method.
\begin{lstlisting}
codes/ee18btech11005.py
\end{lstlisting}
%
provides the necessary soution.
\begin{itemize}
\item  For the system to be stable all coefficients should lie on left half of s-plane. Because if any pole is in right half of s-plane then there will be a component in output that increases without bound,causing system to be unstable.
All the coefficients in the characteristic equation should be positive.This is necessary condition but not sufficient.Because it may have poles on right half of s plane.
Poles are the roots of the characteristic equation.
    \item A system is stable if all of its characteristic modes go to finite value as t goes to infinity.It is possible only if all the poles are on the left half of s plane.
    The characteristic equation should have negative roots only. So the first column should always be greater than zero.That means no sign changes.
    \item A system is unstable if its characteristic modes are not bounded. Then the characteristic equation will also have roots in the right side of s-plane.That means it has sign changes.
    \end{itemize}

\end{enumerate}



\subsection{Stability}
\begin{enumerate}[label=\thesubsection.\arabic*.,ref=\thesubsection.\theenumi]
\numberwithin{equation}{enumi}
\item 
The characteristic equation of linear time invariant system is given by
\begin{align} 
\nabla(s)=s^4+3s^3+3s^2+s+k=0
\end{align}
Find the condition for the system to be BIBO stable using the Routh Array.

\textbf{solution}
\begin{align}
\nabla(s)=s^4+3s^3+3s^2+s+k=0
\end{align}

The Routh hurwitz criterion:-
\bigskip
\begin{align}
\mydet{s^4\\s^3\\s^2\\s^1 \\ s^0}
\mydet{1 & 3 & k \\ 3 & 1 & 0\\  \frac{8}{3}& k & 0\\ \frac{\frac{8}{3}-3k}{\frac{8}{3}} & 0 & 0\\k & 0 & 0} 
\end{align}
From the above array, the given system is stable if
\begin{align}
\begin{split}
k&>0 
\\
  \frac{\frac{8}{3}-3k}{\frac{8}{3}}&>0    
\end{split}
\\
\implies 0<k<\frac{8}{9}
\end{align}
%
\item Modify the Python code in Problem \ref{prob:ee18btech11005_python} to verify your solution by choosing two different values of $k$.
\label{prob:ee18btech11008_python}
\\
\solution 
The following code 
%
\begin{lstlisting}
codes/ee18btech11008.py
\end{lstlisting}
%
provides the necessary soution for $k = 0.5, 3$.
\begin{itemize}
  \item $k=0.5 < \frac{8}{9}$ has no sign changes in first column of its routh array.So the system is stable.
  \item $k=3 > \frac{8}{9}$ has 2 sign changes in first column of its routh array.So the system is unstable.
  \end{itemize}

\end{enumerate}

\section{State-Space Model}
\subsection{Controllability and Observabiity}
%\begin{enumerate}[label=\thesubsection.\arabic*.,ref=\thesubsection.\theenumi]
\numberwithin{equation}{enumi}
\item State the general model of a state space system specifying the dimensions of the matrices and vectors.
\\
\solution The model is given by 
\begin{align}
\label{eq:ee18btech11004_state}
\dot{\vec{x}}(t)&=\vec{A}\vec{x}(t)+\vec{B}\vec{u}(t) \\
 \vec{y}(t)&=\vec{C}\vec{x}(t)+\vec{D} \vec{u}(t)
\end{align}
%
with parameters listed in Table \ref{table:ee18btech11004}.
%
\begin{table}[!ht]
\centering
\begin{enumerate}[label=\thesubsection.\arabic*.,ref=\thesubsection.\theenumi]
\numberwithin{equation}{enumi}
\item State the general model of a state space system specifying the dimensions of the matrices and vectors.
\\
\solution The model is given by 
\begin{align}
\label{eq:ee18btech11004_state}
\dot{\vec{x}}(t)&=\vec{A}\vec{x}(t)+\vec{B}\vec{u}(t) \\
 \vec{y}(t)&=\vec{C}\vec{x}(t)+\vec{D} \vec{u}(t)
\end{align}
%
with parameters listed in Table \ref{table:ee18btech11004}.
%
\begin{table}[!ht]
\centering
\begin{enumerate}[label=\thesubsection.\arabic*.,ref=\thesubsection.\theenumi]
\numberwithin{equation}{enumi}
\item State the general model of a state space system specifying the dimensions of the matrices and vectors.
\\
\solution The model is given by 
\begin{align}
\label{eq:ee18btech11004_state}
\dot{\vec{x}}(t)&=\vec{A}\vec{x}(t)+\vec{B}\vec{u}(t) \\
 \vec{y}(t)&=\vec{C}\vec{x}(t)+\vec{D} \vec{u}(t)
\end{align}
%
with parameters listed in Table \ref{table:ee18btech11004}.
%
\begin{table}[!ht]
\centering
\input{./tables/ee18btech11004.tex}
\caption{}
\label{table:ee18btech11004}
\end{table}
\item Find the transfer function $\vec{H}(s)$ for the general system.
\\
\solution 
Taking Laplace transform on both sides we have the following equations
\begin{align}
 s\vec{IX}(s)-\vec{x}(0)&= \vec{AX}(s)+ \vec{BU}(s)\\
(s\vec{I}-\vec{A})\vec{X}(s)&= \vec{BU}(s)+ \vec{x}(0)\\
\vec{X}(s)&={(s\vec{I}-\vec{A})^{-1}}\vec{B U}(s)\\
& +(s\vec{I}-\vec{A})^{-1}\vec{x}(0)
\label{eq:x_init}
\end{align}
and
\begin{align}
\vec{Y}(s)&= \vec{CX}(s)+D\vec{IU}(s)
\end{align}
Substituting from \eqref{eq:x_init} in the above,
%
\begin{multline}
\vec{Y}(s)=( \vec{C}{(s\vec{I}-\vec{A})^{-1}}\vec{B}+D\vec{I}) \vec{U}(s) 
\\
+ \vec{C}(s\vec{I}-\vec{A})^{-1}\vec{x}(0)
\end{multline}
%
%
\item Find $H(s)$ for a SISO (single input single output) system.
\\
\solution
\begin{align}
\label{eq:ee18btech11004_siso}
H(s)= {\frac{Y(s)}{U(s)}}= C{(sI-A)^{-1}}B+DI
\end{align}

\item Given 
\begin{align}
H(s)&=\frac{1}{s^3+3s^2+2s+1}
\\
D&=0
\\
\vec{B}&= \myvec{0\\0\\1}
\end{align}
%
 find $\vec{A}$ and $\vec{C}$ such that the state-space realization is in {\em controllable canonical form}.
\\
\solution 
\begin{align} 
\because {\frac{Y(s)}{U(s)}}= \frac{Y(s)}{V(s)} \times \frac{V(s)}{U(s)},
\end{align}
letting
\begin{align}
 {\frac{Y(s)}{V(s)}}= 1, 
\end{align}
results in 
\begin{align}
{\frac{U(s)}{V(s)}}={s^3 + 3s^2+2s + 1}
\end{align}

giving
\begin{align}
U(s)= s^3 V(s) + 3s^2 V(s)+2sV(s) + V(s)
\end{align}

so equation 0.1.13 can be written as
\begin{align}
\myvec{sV(s)\\s^2V(s)\\s^3V(s)}
=
\myvec{0&1&0\\0&0&1\\-1&-2&-3}\myvec{V(s)\\s(s)\\s^2V(s)}
+
\myvec{0\\0\\1}  U
\end{align}
So 
\begin{align}
\vec{A}=\myvec{0&1&0\\0&0&1\\-1&-2&-3}
\end{align}

\begin{align}
Y=X_{1}(s)
=\myvec{1&0&0} \myvec{V(s)\\sV(s)\\s^2V(s)}
\end{align}
\begin{align}
\vec{C}=\myvec{1&0&0}
\end{align}

\item Obtain $\vec{A}$ and $\vec{C}$ so that the state-space realization in in {\em observable canonical form}.
\\
\solution  Given that
\begin{align}
H(s)&=\frac{1}{s^3+3s^2+2s+1}
\end{align}
\begin{align}
\frac{Y(s)}{U(s)}=\frac{1}{s^3+3s^2+2s+1} \\
Y(s) \times (s^3+3s^2+2s+1) = U(s)
\end{align}
\begin{align}
s^3Y(s)+3s^2Y(s)+2sY(s)+Y(s)=U(s)\\
s^3Y(s)=U(s)-3s^2Y(s)-2sY(s)-Y(s)\\
Y(s)=-3s^{-1}Y(s)-2s^{-2}Y(s)+s^{-3}(U(s)-Y(s))
\end{align}
\\ let $Y=aU+X_{1}$
\\ by comparing with equation 1.5.6 we get a=0 and
\begin{align}
Y=X_{1}
\end{align}
inverse laplace transform of above equation is 
\begin{align}
y=x_{1}
\end{align}
so from above equation 1.5.6 and 1.5.7
\begin{align}
X_{1}=-3s^{-1}Y(s)-2s^{-2}Y(s)+s^{-3}(U(s)-Y(s))\\
sX_{1}=-3Y(s)-2s^{-1}Y(s)+s^{-2}(U(s)-Y(s)) 
\end{align}
inverse laplace transform of above equation 
\begin{align}
\dot{x_{1}}=-3y+x_{2}
\end{align} 
where
\begin{align}
X_{2}=-2s^{-1}Y(s)+s^{-2}(U(s)-Y(s))\\
sX_{2}=-2Y(s)+s^{-1}(U(s)-Y(s))
\end{align} 
inverse laplace transform of above equation 
\begin{align}
\dot{x_{2}}=-2y+x_{3}
\end{align}
where
\begin{align}
X_{3}=s^{-1}(U(s)-Y(s))\\
sX_{3}=U(s)-Y(s)
\end{align} 
inverse laplace transform of above equation 
\begin{align}
\dot{x_{3}}=u-y
\end{align}
so we get four equations which are
\begin{align}
y=x_{1}\\
\dot{x_{1}}=-3y+x_{2}\\
\dot{x_{2}}=-2y+x_{3}\\
\dot{x_{3}}=u-y
\end{align} 
sub $ y=x_{1}$ in 1.5.19,1.5.20,1.5.21 we get
\begin{align}
 y=x_{1}\\
\dot{x_{1}}=-3x_{1}+x_{2}\\
\dot{x_{2}}=-2x_{1}+x_{3}\\
\dot{x_{3}}=u-x_{1}
\end{align} 
so above equations can be written as
\begin{align}
\myvec{\dot{x_{1}}\\\dot{x_{2}}\\\dot{x_{3}})}
=
\myvec{-3&1&0\\-2&0&1\\-1&0&0}\myvec{x_{1}\\x_{2}\\x_{3}}
+
\myvec{0\\0\\1}  U
\end{align}
So 
\begin{align}
\vec{A}=\myvec{-3&1&0\\-2&0&1\\-1&0&0}
\end{align}
\begin{align}
y=x_{1}
=\myvec{1&0&0} \myvec{x_{1}\\x_{2}\\x_{3}}
\end{align}
\begin{align}
\vec{C}=\myvec{1&0&0}
\end{align}


\item Find the eigenvaues of $\vec{A}$ and the poles of $H(s)$ using a python code.
\\
\solution The following code 
%
\begin{lstlisting}
codes/ee18btech11004.py
\end{lstlisting}
gives the necessary values.  The roots are the same as the eigenvalues.
%
\item Theoretically, show that eigenvaues of $\vec{A}$ are the poles of  $H(s)$.
\solution 
\\ as we know tthat  the characteristic equation is det(sI-A) 
\\\begin{align}
\vec{sI-A}=
\myvec{s&0&0\\0&s&0\\0&0&s}
-
\myvec{0&1&0\\0&0&1\\-1&-2&-3}
=\myvec{s&-1&0\\0&s&-1\\1&2&s+3}
\end{align}
\\therfore
\begin{align}
det(sI-A)=s(s^2+3s+2)+1(1)=s^3+3s^2+2s+1
\end{align} 
\\so from equation 1.6.2 we can see that charcteristic equation is equal to the denominator of the transefer function
\end{enumerate}


\caption{}
\label{table:ee18btech11004}
\end{table}
\item Find the transfer function $\vec{H}(s)$ for the general system.
\\
\solution 
Taking Laplace transform on both sides we have the following equations
\begin{align}
 s\vec{IX}(s)-\vec{x}(0)&= \vec{AX}(s)+ \vec{BU}(s)\\
(s\vec{I}-\vec{A})\vec{X}(s)&= \vec{BU}(s)+ \vec{x}(0)\\
\vec{X}(s)&={(s\vec{I}-\vec{A})^{-1}}\vec{B U}(s)\\
& +(s\vec{I}-\vec{A})^{-1}\vec{x}(0)
\label{eq:x_init}
\end{align}
and
\begin{align}
\vec{Y}(s)&= \vec{CX}(s)+D\vec{IU}(s)
\end{align}
Substituting from \eqref{eq:x_init} in the above,
%
\begin{multline}
\vec{Y}(s)=( \vec{C}{(s\vec{I}-\vec{A})^{-1}}\vec{B}+D\vec{I}) \vec{U}(s) 
\\
+ \vec{C}(s\vec{I}-\vec{A})^{-1}\vec{x}(0)
\end{multline}
%
%
\item Find $H(s)$ for a SISO (single input single output) system.
\\
\solution
\begin{align}
\label{eq:ee18btech11004_siso}
H(s)= {\frac{Y(s)}{U(s)}}= C{(sI-A)^{-1}}B+DI
\end{align}

\item Given 
\begin{align}
H(s)&=\frac{1}{s^3+3s^2+2s+1}
\\
D&=0
\\
\vec{B}&= \myvec{0\\0\\1}
\end{align}
%
 find $\vec{A}$ and $\vec{C}$ such that the state-space realization is in {\em controllable canonical form}.
\\
\solution 
\begin{align} 
\because {\frac{Y(s)}{U(s)}}= \frac{Y(s)}{V(s)} \times \frac{V(s)}{U(s)},
\end{align}
letting
\begin{align}
 {\frac{Y(s)}{V(s)}}= 1, 
\end{align}
results in 
\begin{align}
{\frac{U(s)}{V(s)}}={s^3 + 3s^2+2s + 1}
\end{align}

giving
\begin{align}
U(s)= s^3 V(s) + 3s^2 V(s)+2sV(s) + V(s)
\end{align}

so equation 0.1.13 can be written as
\begin{align}
\myvec{sV(s)\\s^2V(s)\\s^3V(s)}
=
\myvec{0&1&0\\0&0&1\\-1&-2&-3}\myvec{V(s)\\s(s)\\s^2V(s)}
+
\myvec{0\\0\\1}  U
\end{align}
So 
\begin{align}
\vec{A}=\myvec{0&1&0\\0&0&1\\-1&-2&-3}
\end{align}

\begin{align}
Y=X_{1}(s)
=\myvec{1&0&0} \myvec{V(s)\\sV(s)\\s^2V(s)}
\end{align}
\begin{align}
\vec{C}=\myvec{1&0&0}
\end{align}

\item Obtain $\vec{A}$ and $\vec{C}$ so that the state-space realization in in {\em observable canonical form}.
\\
\solution  Given that
\begin{align}
H(s)&=\frac{1}{s^3+3s^2+2s+1}
\end{align}
\begin{align}
\frac{Y(s)}{U(s)}=\frac{1}{s^3+3s^2+2s+1} \\
Y(s) \times (s^3+3s^2+2s+1) = U(s)
\end{align}
\begin{align}
s^3Y(s)+3s^2Y(s)+2sY(s)+Y(s)=U(s)\\
s^3Y(s)=U(s)-3s^2Y(s)-2sY(s)-Y(s)\\
Y(s)=-3s^{-1}Y(s)-2s^{-2}Y(s)+s^{-3}(U(s)-Y(s))
\end{align}
\\ let $Y=aU+X_{1}$
\\ by comparing with equation 1.5.6 we get a=0 and
\begin{align}
Y=X_{1}
\end{align}
inverse laplace transform of above equation is 
\begin{align}
y=x_{1}
\end{align}
so from above equation 1.5.6 and 1.5.7
\begin{align}
X_{1}=-3s^{-1}Y(s)-2s^{-2}Y(s)+s^{-3}(U(s)-Y(s))\\
sX_{1}=-3Y(s)-2s^{-1}Y(s)+s^{-2}(U(s)-Y(s)) 
\end{align}
inverse laplace transform of above equation 
\begin{align}
\dot{x_{1}}=-3y+x_{2}
\end{align} 
where
\begin{align}
X_{2}=-2s^{-1}Y(s)+s^{-2}(U(s)-Y(s))\\
sX_{2}=-2Y(s)+s^{-1}(U(s)-Y(s))
\end{align} 
inverse laplace transform of above equation 
\begin{align}
\dot{x_{2}}=-2y+x_{3}
\end{align}
where
\begin{align}
X_{3}=s^{-1}(U(s)-Y(s))\\
sX_{3}=U(s)-Y(s)
\end{align} 
inverse laplace transform of above equation 
\begin{align}
\dot{x_{3}}=u-y
\end{align}
so we get four equations which are
\begin{align}
y=x_{1}\\
\dot{x_{1}}=-3y+x_{2}\\
\dot{x_{2}}=-2y+x_{3}\\
\dot{x_{3}}=u-y
\end{align} 
sub $ y=x_{1}$ in 1.5.19,1.5.20,1.5.21 we get
\begin{align}
 y=x_{1}\\
\dot{x_{1}}=-3x_{1}+x_{2}\\
\dot{x_{2}}=-2x_{1}+x_{3}\\
\dot{x_{3}}=u-x_{1}
\end{align} 
so above equations can be written as
\begin{align}
\myvec{\dot{x_{1}}\\\dot{x_{2}}\\\dot{x_{3}})}
=
\myvec{-3&1&0\\-2&0&1\\-1&0&0}\myvec{x_{1}\\x_{2}\\x_{3}}
+
\myvec{0\\0\\1}  U
\end{align}
So 
\begin{align}
\vec{A}=\myvec{-3&1&0\\-2&0&1\\-1&0&0}
\end{align}
\begin{align}
y=x_{1}
=\myvec{1&0&0} \myvec{x_{1}\\x_{2}\\x_{3}}
\end{align}
\begin{align}
\vec{C}=\myvec{1&0&0}
\end{align}


\item Find the eigenvaues of $\vec{A}$ and the poles of $H(s)$ using a python code.
\\
\solution The following code 
%
\begin{lstlisting}
codes/ee18btech11004.py
\end{lstlisting}
gives the necessary values.  The roots are the same as the eigenvalues.
%
\item Theoretically, show that eigenvaues of $\vec{A}$ are the poles of  $H(s)$.
\solution 
\\ as we know tthat  the characteristic equation is det(sI-A) 
\\\begin{align}
\vec{sI-A}=
\myvec{s&0&0\\0&s&0\\0&0&s}
-
\myvec{0&1&0\\0&0&1\\-1&-2&-3}
=\myvec{s&-1&0\\0&s&-1\\1&2&s+3}
\end{align}
\\therfore
\begin{align}
det(sI-A)=s(s^2+3s+2)+1(1)=s^3+3s^2+2s+1
\end{align} 
\\so from equation 1.6.2 we can see that charcteristic equation is equal to the denominator of the transefer function
\end{enumerate}


\caption{}
\label{table:ee18btech11004}
\end{table}
\item Find the transfer function $\vec{H}(s)$ for the general system.
\\
\solution 
Taking Laplace transform on both sides we have the following equations
\begin{align}
 s\vec{IX}(s)-\vec{x}(0)&= \vec{AX}(s)+ \vec{BU}(s)\\
(s\vec{I}-\vec{A})\vec{X}(s)&= \vec{BU}(s)+ \vec{x}(0)\\
\vec{X}(s)&={(s\vec{I}-\vec{A})^{-1}}\vec{B U}(s)\\
& +(s\vec{I}-\vec{A})^{-1}\vec{x}(0)
\label{eq:x_init}
\end{align}
and
\begin{align}
\vec{Y}(s)&= \vec{CX}(s)+D\vec{IU}(s)
\end{align}
Substituting from \eqref{eq:x_init} in the above,
%
\begin{multline}
\vec{Y}(s)=( \vec{C}{(s\vec{I}-\vec{A})^{-1}}\vec{B}+D\vec{I}) \vec{U}(s) 
\\
+ \vec{C}(s\vec{I}-\vec{A})^{-1}\vec{x}(0)
\end{multline}
%
%
\item Find $H(s)$ for a SISO (single input single output) system.
\\
\solution
\begin{align}
\label{eq:ee18btech11004_siso}
H(s)= {\frac{Y(s)}{U(s)}}= C{(sI-A)^{-1}}B+DI
\end{align}

\item Given 
\begin{align}
H(s)&=\frac{1}{s^3+3s^2+2s+1}
\\
D&=0
\\
\vec{B}&= \myvec{0\\0\\1}
\end{align}
%
 find $\vec{A}$ and $\vec{C}$ such that the state-space realization is in {\em controllable canonical form}.
\\
\solution 
\begin{align} 
\because {\frac{Y(s)}{U(s)}}= \frac{Y(s)}{V(s)} \times \frac{V(s)}{U(s)},
\end{align}
letting
\begin{align}
 {\frac{Y(s)}{V(s)}}= 1, 
\end{align}
results in 
\begin{align}
{\frac{U(s)}{V(s)}}={s^3 + 3s^2+2s + 1}
\end{align}

giving
\begin{align}
U(s)= s^3 V(s) + 3s^2 V(s)+2sV(s) + V(s)
\end{align}

so equation 0.1.13 can be written as
\begin{align}
\myvec{sV(s)\\s^2V(s)\\s^3V(s)}
=
\myvec{0&1&0\\0&0&1\\-1&-2&-3}\myvec{V(s)\\s(s)\\s^2V(s)}
+
\myvec{0\\0\\1}  U
\end{align}
So 
\begin{align}
\vec{A}=\myvec{0&1&0\\0&0&1\\-1&-2&-3}
\end{align}

\begin{align}
Y=X_{1}(s)
=\myvec{1&0&0} \myvec{V(s)\\sV(s)\\s^2V(s)}
\end{align}
\begin{align}
\vec{C}=\myvec{1&0&0}
\end{align}

\item Obtain $\vec{A}$ and $\vec{C}$ so that the state-space realization in in {\em observable canonical form}.
\\
\solution  Given that
\begin{align}
H(s)&=\frac{1}{s^3+3s^2+2s+1}
\end{align}
\begin{align}
\frac{Y(s)}{U(s)}=\frac{1}{s^3+3s^2+2s+1} \\
Y(s) \times (s^3+3s^2+2s+1) = U(s)
\end{align}
\begin{align}
s^3Y(s)+3s^2Y(s)+2sY(s)+Y(s)=U(s)\\
s^3Y(s)=U(s)-3s^2Y(s)-2sY(s)-Y(s)\\
Y(s)=-3s^{-1}Y(s)-2s^{-2}Y(s)+s^{-3}(U(s)-Y(s))
\end{align}
\\ let $Y=aU+X_{1}$
\\ by comparing with equation 1.5.6 we get a=0 and
\begin{align}
Y=X_{1}
\end{align}
inverse laplace transform of above equation is 
\begin{align}
y=x_{1}
\end{align}
so from above equation 1.5.6 and 1.5.7
\begin{align}
X_{1}=-3s^{-1}Y(s)-2s^{-2}Y(s)+s^{-3}(U(s)-Y(s))\\
sX_{1}=-3Y(s)-2s^{-1}Y(s)+s^{-2}(U(s)-Y(s)) 
\end{align}
inverse laplace transform of above equation 
\begin{align}
\dot{x_{1}}=-3y+x_{2}
\end{align} 
where
\begin{align}
X_{2}=-2s^{-1}Y(s)+s^{-2}(U(s)-Y(s))\\
sX_{2}=-2Y(s)+s^{-1}(U(s)-Y(s))
\end{align} 
inverse laplace transform of above equation 
\begin{align}
\dot{x_{2}}=-2y+x_{3}
\end{align}
where
\begin{align}
X_{3}=s^{-1}(U(s)-Y(s))\\
sX_{3}=U(s)-Y(s)
\end{align} 
inverse laplace transform of above equation 
\begin{align}
\dot{x_{3}}=u-y
\end{align}
so we get four equations which are
\begin{align}
y=x_{1}\\
\dot{x_{1}}=-3y+x_{2}\\
\dot{x_{2}}=-2y+x_{3}\\
\dot{x_{3}}=u-y
\end{align} 
sub $ y=x_{1}$ in 1.5.19,1.5.20,1.5.21 we get
\begin{align}
 y=x_{1}\\
\dot{x_{1}}=-3x_{1}+x_{2}\\
\dot{x_{2}}=-2x_{1}+x_{3}\\
\dot{x_{3}}=u-x_{1}
\end{align} 
so above equations can be written as
\begin{align}
\myvec{\dot{x_{1}}\\\dot{x_{2}}\\\dot{x_{3}})}
=
\myvec{-3&1&0\\-2&0&1\\-1&0&0}\myvec{x_{1}\\x_{2}\\x_{3}}
+
\myvec{0\\0\\1}  U
\end{align}
So 
\begin{align}
\vec{A}=\myvec{-3&1&0\\-2&0&1\\-1&0&0}
\end{align}
\begin{align}
y=x_{1}
=\myvec{1&0&0} \myvec{x_{1}\\x_{2}\\x_{3}}
\end{align}
\begin{align}
\vec{C}=\myvec{1&0&0}
\end{align}


\item Find the eigenvaues of $\vec{A}$ and the poles of $H(s)$ using a python code.
\\
\solution The following code 
%
\begin{lstlisting}
codes/ee18btech11004.py
\end{lstlisting}
gives the necessary values.  The roots are the same as the eigenvalues.
%
\item Theoretically, show that eigenvaues of $\vec{A}$ are the poles of  $H(s)$.
\solution 
\\ as we know tthat  the characteristic equation is det(sI-A) 
\\\begin{align}
\vec{sI-A}=
\myvec{s&0&0\\0&s&0\\0&0&s}
-
\myvec{0&1&0\\0&0&1\\-1&-2&-3}
=\myvec{s&-1&0\\0&s&-1\\1&2&s+3}
\end{align}
\\therfore
\begin{align}
det(sI-A)=s(s^2+3s+2)+1(1)=s^3+3s^2+2s+1
\end{align} 
\\so from equation 1.6.2 we can see that charcteristic equation is equal to the denominator of the transefer function
\end{enumerate}


\begin{enumerate}[label=\thesection.\arabic*.,ref=\thesection.\theenumi]
\numberwithin{equation}{enumi}
\item State the general model of a state space system specifying the dimensions of the matrices and vectors.
\\
\solution The model is given by 
\begin{align}
\dot{\vec{x}}(t)&=\vec{A}\vec{x}(t)+\vec{B}\vec{u}(t) \\
 \vec{y}(t)&=\vec{C}\vec{x}(t)+\vec{D} \vec{u}(t)
\end{align}
with parameters listed in Table \ref{table:ee18btech11004}.
%
\begin{table}[!ht]
\centering
\begin{enumerate}[label=\thesubsection.\arabic*.,ref=\thesubsection.\theenumi]
\numberwithin{equation}{enumi}
\item State the general model of a state space system specifying the dimensions of the matrices and vectors.
\\
\solution The model is given by 
\begin{align}
\label{eq:ee18btech11004_state}
\dot{\vec{x}}(t)&=\vec{A}\vec{x}(t)+\vec{B}\vec{u}(t) \\
 \vec{y}(t)&=\vec{C}\vec{x}(t)+\vec{D} \vec{u}(t)
\end{align}
%
with parameters listed in Table \ref{table:ee18btech11004}.
%
\begin{table}[!ht]
\centering
\begin{enumerate}[label=\thesubsection.\arabic*.,ref=\thesubsection.\theenumi]
\numberwithin{equation}{enumi}
\item State the general model of a state space system specifying the dimensions of the matrices and vectors.
\\
\solution The model is given by 
\begin{align}
\label{eq:ee18btech11004_state}
\dot{\vec{x}}(t)&=\vec{A}\vec{x}(t)+\vec{B}\vec{u}(t) \\
 \vec{y}(t)&=\vec{C}\vec{x}(t)+\vec{D} \vec{u}(t)
\end{align}
%
with parameters listed in Table \ref{table:ee18btech11004}.
%
\begin{table}[!ht]
\centering
\input{./tables/ee18btech11004.tex}
\caption{}
\label{table:ee18btech11004}
\end{table}
\item Find the transfer function $\vec{H}(s)$ for the general system.
\\
\solution 
Taking Laplace transform on both sides we have the following equations
\begin{align}
 s\vec{IX}(s)-\vec{x}(0)&= \vec{AX}(s)+ \vec{BU}(s)\\
(s\vec{I}-\vec{A})\vec{X}(s)&= \vec{BU}(s)+ \vec{x}(0)\\
\vec{X}(s)&={(s\vec{I}-\vec{A})^{-1}}\vec{B U}(s)\\
& +(s\vec{I}-\vec{A})^{-1}\vec{x}(0)
\label{eq:x_init}
\end{align}
and
\begin{align}
\vec{Y}(s)&= \vec{CX}(s)+D\vec{IU}(s)
\end{align}
Substituting from \eqref{eq:x_init} in the above,
%
\begin{multline}
\vec{Y}(s)=( \vec{C}{(s\vec{I}-\vec{A})^{-1}}\vec{B}+D\vec{I}) \vec{U}(s) 
\\
+ \vec{C}(s\vec{I}-\vec{A})^{-1}\vec{x}(0)
\end{multline}
%
%
\item Find $H(s)$ for a SISO (single input single output) system.
\\
\solution
\begin{align}
\label{eq:ee18btech11004_siso}
H(s)= {\frac{Y(s)}{U(s)}}= C{(sI-A)^{-1}}B+DI
\end{align}

\item Given 
\begin{align}
H(s)&=\frac{1}{s^3+3s^2+2s+1}
\\
D&=0
\\
\vec{B}&= \myvec{0\\0\\1}
\end{align}
%
 find $\vec{A}$ and $\vec{C}$ such that the state-space realization is in {\em controllable canonical form}.
\\
\solution 
\begin{align} 
\because {\frac{Y(s)}{U(s)}}= \frac{Y(s)}{V(s)} \times \frac{V(s)}{U(s)},
\end{align}
letting
\begin{align}
 {\frac{Y(s)}{V(s)}}= 1, 
\end{align}
results in 
\begin{align}
{\frac{U(s)}{V(s)}}={s^3 + 3s^2+2s + 1}
\end{align}

giving
\begin{align}
U(s)= s^3 V(s) + 3s^2 V(s)+2sV(s) + V(s)
\end{align}

so equation 0.1.13 can be written as
\begin{align}
\myvec{sV(s)\\s^2V(s)\\s^3V(s)}
=
\myvec{0&1&0\\0&0&1\\-1&-2&-3}\myvec{V(s)\\s(s)\\s^2V(s)}
+
\myvec{0\\0\\1}  U
\end{align}
So 
\begin{align}
\vec{A}=\myvec{0&1&0\\0&0&1\\-1&-2&-3}
\end{align}

\begin{align}
Y=X_{1}(s)
=\myvec{1&0&0} \myvec{V(s)\\sV(s)\\s^2V(s)}
\end{align}
\begin{align}
\vec{C}=\myvec{1&0&0}
\end{align}

\item Obtain $\vec{A}$ and $\vec{C}$ so that the state-space realization in in {\em observable canonical form}.
\\
\solution  Given that
\begin{align}
H(s)&=\frac{1}{s^3+3s^2+2s+1}
\end{align}
\begin{align}
\frac{Y(s)}{U(s)}=\frac{1}{s^3+3s^2+2s+1} \\
Y(s) \times (s^3+3s^2+2s+1) = U(s)
\end{align}
\begin{align}
s^3Y(s)+3s^2Y(s)+2sY(s)+Y(s)=U(s)\\
s^3Y(s)=U(s)-3s^2Y(s)-2sY(s)-Y(s)\\
Y(s)=-3s^{-1}Y(s)-2s^{-2}Y(s)+s^{-3}(U(s)-Y(s))
\end{align}
\\ let $Y=aU+X_{1}$
\\ by comparing with equation 1.5.6 we get a=0 and
\begin{align}
Y=X_{1}
\end{align}
inverse laplace transform of above equation is 
\begin{align}
y=x_{1}
\end{align}
so from above equation 1.5.6 and 1.5.7
\begin{align}
X_{1}=-3s^{-1}Y(s)-2s^{-2}Y(s)+s^{-3}(U(s)-Y(s))\\
sX_{1}=-3Y(s)-2s^{-1}Y(s)+s^{-2}(U(s)-Y(s)) 
\end{align}
inverse laplace transform of above equation 
\begin{align}
\dot{x_{1}}=-3y+x_{2}
\end{align} 
where
\begin{align}
X_{2}=-2s^{-1}Y(s)+s^{-2}(U(s)-Y(s))\\
sX_{2}=-2Y(s)+s^{-1}(U(s)-Y(s))
\end{align} 
inverse laplace transform of above equation 
\begin{align}
\dot{x_{2}}=-2y+x_{3}
\end{align}
where
\begin{align}
X_{3}=s^{-1}(U(s)-Y(s))\\
sX_{3}=U(s)-Y(s)
\end{align} 
inverse laplace transform of above equation 
\begin{align}
\dot{x_{3}}=u-y
\end{align}
so we get four equations which are
\begin{align}
y=x_{1}\\
\dot{x_{1}}=-3y+x_{2}\\
\dot{x_{2}}=-2y+x_{3}\\
\dot{x_{3}}=u-y
\end{align} 
sub $ y=x_{1}$ in 1.5.19,1.5.20,1.5.21 we get
\begin{align}
 y=x_{1}\\
\dot{x_{1}}=-3x_{1}+x_{2}\\
\dot{x_{2}}=-2x_{1}+x_{3}\\
\dot{x_{3}}=u-x_{1}
\end{align} 
so above equations can be written as
\begin{align}
\myvec{\dot{x_{1}}\\\dot{x_{2}}\\\dot{x_{3}})}
=
\myvec{-3&1&0\\-2&0&1\\-1&0&0}\myvec{x_{1}\\x_{2}\\x_{3}}
+
\myvec{0\\0\\1}  U
\end{align}
So 
\begin{align}
\vec{A}=\myvec{-3&1&0\\-2&0&1\\-1&0&0}
\end{align}
\begin{align}
y=x_{1}
=\myvec{1&0&0} \myvec{x_{1}\\x_{2}\\x_{3}}
\end{align}
\begin{align}
\vec{C}=\myvec{1&0&0}
\end{align}


\item Find the eigenvaues of $\vec{A}$ and the poles of $H(s)$ using a python code.
\\
\solution The following code 
%
\begin{lstlisting}
codes/ee18btech11004.py
\end{lstlisting}
gives the necessary values.  The roots are the same as the eigenvalues.
%
\item Theoretically, show that eigenvaues of $\vec{A}$ are the poles of  $H(s)$.
\solution 
\\ as we know tthat  the characteristic equation is det(sI-A) 
\\\begin{align}
\vec{sI-A}=
\myvec{s&0&0\\0&s&0\\0&0&s}
-
\myvec{0&1&0\\0&0&1\\-1&-2&-3}
=\myvec{s&-1&0\\0&s&-1\\1&2&s+3}
\end{align}
\\therfore
\begin{align}
det(sI-A)=s(s^2+3s+2)+1(1)=s^3+3s^2+2s+1
\end{align} 
\\so from equation 1.6.2 we can see that charcteristic equation is equal to the denominator of the transefer function
\end{enumerate}


\caption{}
\label{table:ee18btech11004}
\end{table}
\item Find the transfer function $\vec{H}(s)$ for the general system.
\\
\solution 
Taking Laplace transform on both sides we have the following equations
\begin{align}
 s\vec{IX}(s)-\vec{x}(0)&= \vec{AX}(s)+ \vec{BU}(s)\\
(s\vec{I}-\vec{A})\vec{X}(s)&= \vec{BU}(s)+ \vec{x}(0)\\
\vec{X}(s)&={(s\vec{I}-\vec{A})^{-1}}\vec{B U}(s)\\
& +(s\vec{I}-\vec{A})^{-1}\vec{x}(0)
\label{eq:x_init}
\end{align}
and
\begin{align}
\vec{Y}(s)&= \vec{CX}(s)+D\vec{IU}(s)
\end{align}
Substituting from \eqref{eq:x_init} in the above,
%
\begin{multline}
\vec{Y}(s)=( \vec{C}{(s\vec{I}-\vec{A})^{-1}}\vec{B}+D\vec{I}) \vec{U}(s) 
\\
+ \vec{C}(s\vec{I}-\vec{A})^{-1}\vec{x}(0)
\end{multline}
%
%
\item Find $H(s)$ for a SISO (single input single output) system.
\\
\solution
\begin{align}
\label{eq:ee18btech11004_siso}
H(s)= {\frac{Y(s)}{U(s)}}= C{(sI-A)^{-1}}B+DI
\end{align}

\item Given 
\begin{align}
H(s)&=\frac{1}{s^3+3s^2+2s+1}
\\
D&=0
\\
\vec{B}&= \myvec{0\\0\\1}
\end{align}
%
 find $\vec{A}$ and $\vec{C}$ such that the state-space realization is in {\em controllable canonical form}.
\\
\solution 
\begin{align} 
\because {\frac{Y(s)}{U(s)}}= \frac{Y(s)}{V(s)} \times \frac{V(s)}{U(s)},
\end{align}
letting
\begin{align}
 {\frac{Y(s)}{V(s)}}= 1, 
\end{align}
results in 
\begin{align}
{\frac{U(s)}{V(s)}}={s^3 + 3s^2+2s + 1}
\end{align}

giving
\begin{align}
U(s)= s^3 V(s) + 3s^2 V(s)+2sV(s) + V(s)
\end{align}

so equation 0.1.13 can be written as
\begin{align}
\myvec{sV(s)\\s^2V(s)\\s^3V(s)}
=
\myvec{0&1&0\\0&0&1\\-1&-2&-3}\myvec{V(s)\\s(s)\\s^2V(s)}
+
\myvec{0\\0\\1}  U
\end{align}
So 
\begin{align}
\vec{A}=\myvec{0&1&0\\0&0&1\\-1&-2&-3}
\end{align}

\begin{align}
Y=X_{1}(s)
=\myvec{1&0&0} \myvec{V(s)\\sV(s)\\s^2V(s)}
\end{align}
\begin{align}
\vec{C}=\myvec{1&0&0}
\end{align}

\item Obtain $\vec{A}$ and $\vec{C}$ so that the state-space realization in in {\em observable canonical form}.
\\
\solution  Given that
\begin{align}
H(s)&=\frac{1}{s^3+3s^2+2s+1}
\end{align}
\begin{align}
\frac{Y(s)}{U(s)}=\frac{1}{s^3+3s^2+2s+1} \\
Y(s) \times (s^3+3s^2+2s+1) = U(s)
\end{align}
\begin{align}
s^3Y(s)+3s^2Y(s)+2sY(s)+Y(s)=U(s)\\
s^3Y(s)=U(s)-3s^2Y(s)-2sY(s)-Y(s)\\
Y(s)=-3s^{-1}Y(s)-2s^{-2}Y(s)+s^{-3}(U(s)-Y(s))
\end{align}
\\ let $Y=aU+X_{1}$
\\ by comparing with equation 1.5.6 we get a=0 and
\begin{align}
Y=X_{1}
\end{align}
inverse laplace transform of above equation is 
\begin{align}
y=x_{1}
\end{align}
so from above equation 1.5.6 and 1.5.7
\begin{align}
X_{1}=-3s^{-1}Y(s)-2s^{-2}Y(s)+s^{-3}(U(s)-Y(s))\\
sX_{1}=-3Y(s)-2s^{-1}Y(s)+s^{-2}(U(s)-Y(s)) 
\end{align}
inverse laplace transform of above equation 
\begin{align}
\dot{x_{1}}=-3y+x_{2}
\end{align} 
where
\begin{align}
X_{2}=-2s^{-1}Y(s)+s^{-2}(U(s)-Y(s))\\
sX_{2}=-2Y(s)+s^{-1}(U(s)-Y(s))
\end{align} 
inverse laplace transform of above equation 
\begin{align}
\dot{x_{2}}=-2y+x_{3}
\end{align}
where
\begin{align}
X_{3}=s^{-1}(U(s)-Y(s))\\
sX_{3}=U(s)-Y(s)
\end{align} 
inverse laplace transform of above equation 
\begin{align}
\dot{x_{3}}=u-y
\end{align}
so we get four equations which are
\begin{align}
y=x_{1}\\
\dot{x_{1}}=-3y+x_{2}\\
\dot{x_{2}}=-2y+x_{3}\\
\dot{x_{3}}=u-y
\end{align} 
sub $ y=x_{1}$ in 1.5.19,1.5.20,1.5.21 we get
\begin{align}
 y=x_{1}\\
\dot{x_{1}}=-3x_{1}+x_{2}\\
\dot{x_{2}}=-2x_{1}+x_{3}\\
\dot{x_{3}}=u-x_{1}
\end{align} 
so above equations can be written as
\begin{align}
\myvec{\dot{x_{1}}\\\dot{x_{2}}\\\dot{x_{3}})}
=
\myvec{-3&1&0\\-2&0&1\\-1&0&0}\myvec{x_{1}\\x_{2}\\x_{3}}
+
\myvec{0\\0\\1}  U
\end{align}
So 
\begin{align}
\vec{A}=\myvec{-3&1&0\\-2&0&1\\-1&0&0}
\end{align}
\begin{align}
y=x_{1}
=\myvec{1&0&0} \myvec{x_{1}\\x_{2}\\x_{3}}
\end{align}
\begin{align}
\vec{C}=\myvec{1&0&0}
\end{align}


\item Find the eigenvaues of $\vec{A}$ and the poles of $H(s)$ using a python code.
\\
\solution The following code 
%
\begin{lstlisting}
codes/ee18btech11004.py
\end{lstlisting}
gives the necessary values.  The roots are the same as the eigenvalues.
%
\item Theoretically, show that eigenvaues of $\vec{A}$ are the poles of  $H(s)$.
\solution 
\\ as we know tthat  the characteristic equation is det(sI-A) 
\\\begin{align}
\vec{sI-A}=
\myvec{s&0&0\\0&s&0\\0&0&s}
-
\myvec{0&1&0\\0&0&1\\-1&-2&-3}
=\myvec{s&-1&0\\0&s&-1\\1&2&s+3}
\end{align}
\\therfore
\begin{align}
det(sI-A)=s(s^2+3s+2)+1(1)=s^3+3s^2+2s+1
\end{align} 
\\so from equation 1.6.2 we can see that charcteristic equation is equal to the denominator of the transefer function
\end{enumerate}


\caption{}
\label{table:ee18btech11004}
\end{table}

\item Find the transfer function $\vec{H}(s)$ for the general system.
\\
\solution 
Taking Laplace transform on both sides we have the following equations
\begin{align}
 s\vec{IX}(s)-\vec{x}(0)&= \vec{AX}(s)+ \vec{BU}(s)\\
(s\vec{I}-\vec{A})\vec{X}(s)&= \vec{BU}(s)+ \vec{x}(0)\\
\vec{X}(s)&={(s\vec{I}-\vec{A})^{-1}}\vec{B U}(s)\\
& +(s\vec{I}-\vec{A})^{-1}\vec{x}(0)
\label{eq:x_init}
\end{align}
and
\begin{align}
\vec{Y}(s)&= \vec{CX}(s)+D\vec{IU}(s)
\end{align}
Substituting from \eqref{eq:x_init} in the above,
%
\begin{multline}
\vec{Y}(s)=( \vec{C}{(s\vec{I}-\vec{A})^{-1}}\vec{B}+D\vec{I}) \vec{U}(s) 
\\
+ \vec{C}(s\vec{I}-\vec{A})^{-1}\vec{x}(0)
\end{multline}
%
\item Find $H(s)$ for a SISO (single input single output) system.
\\
\solution
\begin{align}
\label{eq:ee18btech11004_siso}
H(s)= {\frac{Y(s)}{U(s)}}= \vec{C}{(s\vec{I}-\vec{A})^{-1}}\vec{B}+D\vec{I}
\end{align}

\item Given 
\begin{align}
\label{eq:ee18btech11004_system}
H(s)&=\frac{1}{s^3+3s^2+2s+1}
\\
D&=0
\\
\vec{B}&= \myvec{0\\0\\1}
\end{align}
%
 find $\vec{A}$ and $\vec{C}$ such that the state-space realization is in {\em controllable canonical form}.
\\
\solution 
\begin{align} 
\because {\frac{Y(s)}{U(s)}}= \frac{Y(s)}{V(s)} \times \frac{V(s)}{U(s)},
\end{align}
letting
\begin{align}
 {\frac{Y(s)}{V(s)}}= 1, 
\end{align}
results in 
\begin{align}
{\frac{U(s)}{V(s)}}={s^3 + 3s^2+2s + 1}
\end{align}

giving
\begin{align}
U(s)= s^3 V(s) + 3s^2 V(s)+2sV(s) + V(s)
\end{align}
%
so the above equation  can be written as
\begin{align}
\myvec{sV(s)\\s^2V(s)\\s^3V(s)}
=
\myvec{0&1&0\\0&0&1\\-1&-2&-3}\myvec{V(s)\\sV(s)\\s^2V(s)}
+
\myvec{0\\0\\1}  U
\end{align}
Letting
\begin{align}
\vec{A}&=\myvec{0&1&0\\0&0&1\\-1&-2&-3}
\\
\vec{X}_1 &= \myvec{sV(s)\\s^2V(s)\\s^3V(s)}
\\
\vec{X} &= \myvec{V(s)\\sV(s)\\s^2V(s)},
\end{align}
\\
\begin{align}
\vec{X}_{1}(s) &= \vec{A}\vec{X}(s)+ \vec{B}U(s)
\\
Y&=\vec{C}\vec{X}_{1}(s)
\end{align}
where
\begin{align}
\vec{C}=\myvec{1&0&0}
\end{align}

\item Obtain $\vec{A}$ and $\vec{C}$ so that the state-space realization in in {\em observable canonical form}.
\\
\solution  Given that
\begin{align}
H(s)&=\frac{1}{s^3+3s^2+2s+1},
\\
%\end{align}
%\begin{align}
\frac{Y(s)}{U(s)}&=\frac{1}{s^3+3s^2+2s+1} \\
\implies  U(s)&=Y(s)  (s^3+3s^2+2s+1)
%\end{align}
%\begin{align}
%U(s)&=s^3Y(s)+3s^2Y(s)+2sY(s)+Y(s)\\
%s^3Y(s)&=U(s)-3s^2Y(s)-2sY(s)-Y(s)\\
\\
\text{or, }Y(s)&=-3s^{-1}Y(s)-2s^{-2}Y(s)+
\nonumber \\
&\quad s^{-3}(U(s)-Y(s))
\end{align}
%\\ let $Y=aU+X_{1}$
%\\ by comparing with equation 1.5.6 we get a=0 and
%so from above equation 1.5.6 and 1.5.7
Let
\begin{align}
X_{1}(s)&=Y(s) = -3s^{-1}Y(s)-2s^{-2}Y(s) \nonumber \\
&\quad +s^{-3}(U(s)-Y(s))\\
%sX_{1}(s)&=-3Y(s)-2s^{-1}Y(s)\\ 
%&+s^{-2}(U(s)-Y(s))
%\end{align} 
%\begin{align}
%sX_{1}(s)&=-3Y(s)+X_{2}(s)
%\end{align} 
%where
%\begin{align}
X_{2}(s)&=-2s^{-1}Y(s)+s^{-2}(U(s)-Y(s))\\
%sX_{2}(s)&=-2Y(s)+s^{-1}(U(s)-Y(s))
%\end{align} 
%\begin{align}
%sX_{2}(s)&=-2Y(s)+X_{3}(s)
%\end{align}
%where
%\begin{align}
X_{3}(s)&=s^{-1}(U(s)-Y(s))
%sX_{3}(s)&=U(s)-Y(s)
\end{align} 
%\begin{align}
%X_{3}(s)&=U(s)-Y(s)
%\end{align}
\begin{align}
\implies
\begin{split}
sX_{1}(s)&=-3Y(s)+X_{2}(s)\\
sX_{2}(s)&=-2Y(s)+X_{3}(s)\\
sX_{3}(s)&=U(s)-Y(s)
\end{split}
\end{align} 
Substituting  $ Y=X_{1}(s)$ the above, 
\begin{align}
sX_{1}(s)&=-3X_{1}(s)+X_{2}(s)\\
sX_{2}(s)&=-2X_{1}(s)+X_{3}(s)\\
sX_{3}(s)&=U(s)-X_{1}(s)
\end{align} 
which can be expressed as
\begin{align}
\myvec{sX_{1}(s)\\sX_{2}(s)\\sX_{3}(s)}
&=
\myvec{-3&1&0\\-2&0&1\\-1&0&0}\myvec{X_{1}(s)\\X_{2}(s)\\X_{3}(s)}
+
\myvec{0\\0\\1}  U
\\
\text{or, }
\begin{split}
s\vec{X}(s) &= \vec{A}\vec{X}(s) + \vec{B}U(s)
\\
Y(s) &= \vec{B}\vec{X}(s)
\end{split}
\end{align}
where
\begin{align}
\vec{A}&=\myvec{-3&1&0\\-2&0&1\\-1&0&0}
\\
\vec{B}&= \myvec{1&0&0}
\end{align}
%\begin{align}
%Y(s)=X_{1}(s)
%=\myvec{1&0&0} \myvec{X_{1}(s)\\X_{2}(s)\\X_{3}(s)}
%\end{align}
%\begin{align}
%\vec{C}&=\myvec{1&0&0}
%\end{align}
%
%
%
%\\
%\solution  Given that
%\begin{align}
%H(s)&=\frac{1}{s^3+3s^2+2s+1},
%\\
%\frac{Y(s)}{U(s)}&=\frac{1}{s^3+3s^2+2s+1} \\
%%\implies  U(s)&= (s^3+3s^2+2s+1)Y(s) 
%%\\
%%\implies 
%%U(s)&=s^3Y(s)+3s^2Y(s)+2sY(s)+Y(s)\\
%%s^3Y(s)&=U(s)-3s^2Y(s)-2sY(s)-Y(s)\\
%\implies Y(s)&=-3s^{-1}Y(s)-2s^{-2}Y(s) \nonumber \\
%&\quad +s^{-3}\brak{U(s)-Y(s)}
%\end{align}
%%
%after some algebra.
%\\ let $Y=aU+X_{1}$
%\\ by comparing with equation 1.5.6 we get a=0 and
%\begin{align}
%Y=X_{1}
%\end{align}
%inverse laplace transform of above equation is 
%\begin{align}
%y=x_{1}
%\end{align}
%so from above equation 1.5.6 and 1.5.7
%\begin{align}
%X_{1}&=-3s^{-1}Y(s)-2s^{-2}Y(s)+s^{-3}(U(s)-Y(s))\\
%sX_{1}&=-3Y(s)-2s^{-1}Y(s)+s^{-2}(U(s)-Y(s)) 
%\end{align}
%inverse laplace transform of above equation 
%\begin{align}
%\dot{x_{1}}&=-3y+x_{2}
%\end{align} 
%where
%\begin{align}
%X_{2}&=-2s^{-1}Y(s)+s^{-2}(U(s)-Y(s))\\
%sX_{2}&=-2Y(s)+s^{-1}(U(s)-Y(s))
%\end{align} 
%inverse laplace transform of above equation 
%\begin{align}
%\dot{x_{2}}&=-2y+x_{3}
%\end{align}
%where
%\begin{align}
%X_{3}&=s^{-1}(U(s)-Y(s))\\
%sX_{3}&=U(s)-Y(s)
%\end{align} 
%inverse laplace transform of above equation 
%\begin{align}
%\dot{x_{3}}&=u-y
%\end{align}
%so we get four equations which are
%\begin{align}
%x_{1}&=y\\
%\dot{x_{1}}&=-3y+x_{2}\\
%\dot{x_{2}}&=-2y+x_{3}\\
%\dot{x_{3}}&=u-y
%\end{align} 
%sub $ y=x_{1}$ in 1.5.19,1.5.20,1.5.21 we get
%\begin{align}
%x_{1}&=y\\
%\dot{x_{1}}&=-3x_{1}+x_{2}\\
%\dot{x_{2}}&=-2x_{1}+x_{3}\\
%\dot{x_{3}}&=u-x_{1}
%\end{align} 
%so above equations can be written as
%\begin{align}
%\myvec{\dot{x_{1}}\\\dot{x_{2}}\\\dot{x_{3}})}
%=
%\myvec{-3&1&0\\-2&0&1\\-1&0&0}\myvec{x_{1}\\x_{2}\\x_{3}}
%+
%\myvec{0\\0\\1}  U
%\end{align}
%So 
%\begin{align}
%\vec{A}=\myvec{-3&1&0\\-2&0&1\\-1&0&0}
%\end{align}
%\begin{align}
%y=x_{1}
%=\myvec{1&0&0} \myvec{x_{1}\\x_{2}\\x_{3}}
%\end{align}
%\begin{align}
%\vec{C}&=\myvec{1&0&0}
%\end{align}
%

\item Find the eigenvaues of $\vec{A}$ and the poles of $H(s)$ using a python code.
\\
\solution The following code 
%
\begin{lstlisting}
codes/ee18btech11004.py
\end{lstlisting}
gives the necessary values.  The roots are the same as the eigenvalues.
%
\item Theoretically, show that eigenvaues of $\vec{A}$ are the poles of  $H(s)$.\\
\solution 
As we know that  the characteristic equation is det(sI-A) 
\\\begin{align}
s\vec{I}-\vec{A}=
\myvec{s&0&0\\0&s&0\\0&0&s}
-
\myvec{0&1&0\\0&0&1\\-1&-2&-3}\\
=\myvec{s&-1&0\\0&s&-1\\1&2&s+3}
\end{align}

\begin{align}
\implies \mydet{s\vec{I}-\vec{A}}&=s(s^2+3s+2)+1(1)\\
&=s^3+3s^2+2s+1
\end{align} 
which is the denominator of $H(s)$ in \eqref{eq:ee18btech11004_system}
%
\end{enumerate}


\subsection{Second Order System}
\begin{enumerate}[label=\thesubsection.\arabic*.,ref=\thesubsection.\theenumi]
\numberwithin{equation}{enumi}

\item Consider a state-variable model of a system 
\begin{align}
\label{eq:ee18btech11011_state}
\myvec{\dot{x_{1}}\\\dot{x_{2}}}
=
\myvec{0&1\\-\alpha&-2\beta}\myvec{x_{1}\\x_{2}}
+
\myvec{b_{1}\\b_{2}}  r
\end{align}
\begin{align}
y
=
\myvec{1&0}\myvec{x_{1}\\x_{2}}
\end{align}
where y is the output, and r is the input.
%
\item List the various state matrices in \eqref{eq:ee18btech11011_state}

\item Find the the system transfer function $H(s)$.

\solution From \eqref{eq:ee18btech11004_state} and ,
%
\eqref{eq:ee18btech11004_siso}, 
the transfer function for the state space model is
\begin{align}
H(s) &= C(sI - A)^{-1}B + D
\\
& = \frac
{
\myvec{1&0}\myvec{s+2\beta&1\\-\alpha&s}\myvec{b_{1}\\b_{2}}
}
{
s(s+2\beta) + \alpha
}
\\
&= {\frac{b_{1}(s+2\beta) + b_{2}}{s^{2}+2s\beta+\alpha}}
\\
   \implies H(s) &= {\frac{b_{1}s}{s^{2}+2s\beta+\alpha}} + \LARGE{\frac{2b_{1}\beta + b_{2}}{s^{2}+2s\beta+\alpha}}
\label{eq:ee18btech11011_second}
\end{align}
%\begin{table}[!ht]
%\centering
%\begin{enumerate}[label=\thesubsection.\arabic*.,ref=\thesubsection.\theenumi]
\numberwithin{equation}{enumi}

\item Consider a state-variable model of a system 
\begin{align}
\label{eq:ee18btech11011_state}
\myvec{\dot{x_{1}}\\\dot{x_{2}}}
=
\myvec{0&1\\-\alpha&-2\beta}\myvec{x_{1}\\x_{2}}
+
\myvec{b_{1}\\b_{2}}  r
\end{align}
\begin{align}
y
=
\myvec{1&0}\myvec{x_{1}\\x_{2}}
\end{align}
where y is the output, and r is the input.
%
\item List the various state matrices in \eqref{eq:ee18btech11011_state}

\item Find the the system transfer function $H(s)$.

\solution From \eqref{eq:ee18btech11004_state} and ,
%
\eqref{eq:ee18btech11004_siso}, 
the transfer function for the state space model is
\begin{align}
H(s) &= C(sI - A)^{-1}B + D
\\
& = \frac
{
\myvec{1&0}\myvec{s+2\beta&1\\-\alpha&s}\myvec{b_{1}\\b_{2}}
}
{
s(s+2\beta) + \alpha
}
\\
&= {\frac{b_{1}(s+2\beta) + b_{2}}{s^{2}+2s\beta+\alpha}}
\\
   \implies H(s) &= {\frac{b_{1}s}{s^{2}+2s\beta+\alpha}} + \LARGE{\frac{2b_{1}\beta + b_{2}}{s^{2}+2s\beta+\alpha}}
\label{eq:ee18btech11011_second}
\end{align}
%\begin{table}[!ht]
%\centering
%\begin{enumerate}[label=\thesubsection.\arabic*.,ref=\thesubsection.\theenumi]
\numberwithin{equation}{enumi}

\item Consider a state-variable model of a system 
\begin{align}
\label{eq:ee18btech11011_state}
\myvec{\dot{x_{1}}\\\dot{x_{2}}}
=
\myvec{0&1\\-\alpha&-2\beta}\myvec{x_{1}\\x_{2}}
+
\myvec{b_{1}\\b_{2}}  r
\end{align}
\begin{align}
y
=
\myvec{1&0}\myvec{x_{1}\\x_{2}}
\end{align}
where y is the output, and r is the input.
%
\item List the various state matrices in \eqref{eq:ee18btech11011_state}

\item Find the the system transfer function $H(s)$.

\solution From \eqref{eq:ee18btech11004_state} and ,
%
\eqref{eq:ee18btech11004_siso}, 
the transfer function for the state space model is
\begin{align}
H(s) &= C(sI - A)^{-1}B + D
\\
& = \frac
{
\myvec{1&0}\myvec{s+2\beta&1\\-\alpha&s}\myvec{b_{1}\\b_{2}}
}
{
s(s+2\beta) + \alpha
}
\\
&= {\frac{b_{1}(s+2\beta) + b_{2}}{s^{2}+2s\beta+\alpha}}
\\
   \implies H(s) &= {\frac{b_{1}s}{s^{2}+2s\beta+\alpha}} + \LARGE{\frac{2b_{1}\beta + b_{2}}{s^{2}+2s\beta+\alpha}}
\label{eq:ee18btech11011_second}
\end{align}
%\begin{table}[!ht]
%\centering
%\input{./tables/ee18btech11011}
%\caption{}
%\label{table:ee18btech11011}
%\end{table}
\item Find the Damping ratio $\zeta$ and the Undamped natural frequency $\omega_{n}$ of the system.
\\
\solution Generally for a second order system the transfer function is given by \ref{eq:ee18btech11012_second}
%
\begin{align}
H(s) = \LARGE{\frac{\omega_{n}^2}{s^{2}+2s\zeta\omega_{n}+\omega_{n}^2}}
\end{align}
%
Comparing the denominator of the above with \eqref{eq:ee18btech11011_second},
%
\begin{align}
2\zeta\omega_{n} &= 2\beta,
\\
\omega_{n}^2 &= \alpha
\\
\implies \zeta &= \frac{\beta}{\sqrt{\alpha}} , \omega_{n} = \sqrt{\alpha}
\end{align}
%
\item Using Table \ref{table:ee18btech11012}, explain how the damping conditions depend upon $\alpha$ and  $\beta$.

%\item How do $\alpha$ and $\beta$ affect the system performance?  Explain through plots.
%\\
%\solution The following code plots Fig. \ref{fig:ee18btech11011}
%%
%\begin{figure}[!ht]
%	\begin{center}
%\includegraphics[width=\columnwidth]{./figs/ee18btech11011.eps}		
%	\end{center}
%\caption{}
%\label{fig:ee18btech11011}
%\end{figure}

\end{enumerate}


%\caption{}
%\label{table:ee18btech11011}
%\end{table}
\item Find the Damping ratio $\zeta$ and the Undamped natural frequency $\omega_{n}$ of the system.
\\
\solution Generally for a second order system the transfer function is given by \ref{eq:ee18btech11012_second}
%
\begin{align}
H(s) = \LARGE{\frac{\omega_{n}^2}{s^{2}+2s\zeta\omega_{n}+\omega_{n}^2}}
\end{align}
%
Comparing the denominator of the above with \eqref{eq:ee18btech11011_second},
%
\begin{align}
2\zeta\omega_{n} &= 2\beta,
\\
\omega_{n}^2 &= \alpha
\\
\implies \zeta &= \frac{\beta}{\sqrt{\alpha}} , \omega_{n} = \sqrt{\alpha}
\end{align}
%
\item Using Table \ref{table:ee18btech11012}, explain how the damping conditions depend upon $\alpha$ and  $\beta$.

%\item How do $\alpha$ and $\beta$ affect the system performance?  Explain through plots.
%\\
%\solution The following code plots Fig. \ref{fig:ee18btech11011}
%%
%\begin{figure}[!ht]
%	\begin{center}
%\includegraphics[width=\columnwidth]{./figs/ee18btech11011.eps}		
%	\end{center}
%\caption{}
%\label{fig:ee18btech11011}
%\end{figure}

\end{enumerate}


%\caption{}
%\label{table:ee18btech11011}
%\end{table}
\item Find the Damping ratio $\zeta$ and the Undamped natural frequency $\omega_{n}$ of the system.
\\
\solution Generally for a second order system the transfer function is given by \ref{eq:ee18btech11012_second}
%
\begin{align}
H(s) = \LARGE{\frac{\omega_{n}^2}{s^{2}+2s\zeta\omega_{n}+\omega_{n}^2}}
\end{align}
%
Comparing the denominator of the above with \eqref{eq:ee18btech11011_second},
%
\begin{align}
2\zeta\omega_{n} &= 2\beta,
\\
\omega_{n}^2 &= \alpha
\\
\implies \zeta &= \frac{\beta}{\sqrt{\alpha}} , \omega_{n} = \sqrt{\alpha}
\end{align}
%
\item Using Table \ref{table:ee18btech11012}, explain how the damping conditions depend upon $\alpha$ and  $\beta$.

%\item How do $\alpha$ and $\beta$ affect the system performance?  Explain through plots.
%\\
%\solution The following code plots Fig. \ref{fig:ee18btech11011}
%%
%\begin{figure}[!ht]
%	\begin{center}
%\includegraphics[width=\columnwidth]{./figs/ee18btech11011.eps}		
%	\end{center}
%\caption{}
%\label{fig:ee18btech11011}
%\end{figure}

\end{enumerate}



\section{Nyquist Plot}
\begin{enumerate}[label=\thesection.\arabic*.,ref=\thesection.\theenumi]
\numberwithin{equation}{enumi}
\item The open loop transfer function of a unity feedback system is given by
\begin{align}
\label{eq:ee18btech11007_system}
 G(s)=\frac{\pi e^{-0.25s}}{s}
\end{align}
\item Find $\text{Re} \cbrak{G(\j \omega)}$ and $\text{Im} \cbrak{G(\j \omega)}$.
\\
\solution From \eqref{eq:ee18btech11007_system},
%
\begin{align}
G(j\omega)&=\frac{\pi}{\omega}(-\sin{0.25\omega}-j\cos{0.25\omega})
\\
\implies  \text{Re} \cbrak{G(\j \omega)}&=\frac{\pi}{\omega}(-\sin{0.25\omega}) 
\\
 \text{Im} \cbrak{G(\j \omega)}&=\frac{\pi}{\omega}(-j\cos{0.25\omega}) 
\end{align}
%
\item Sketch the Nyquist plot.
\\
\solution The Nyquist plot is a graph of $\text{Re} \cbrak{G(\j \omega)}$  vs $\text{Im} \cbrak{G(\j \omega)}$.
The following python code generates the Nyquist plot in Fig.  \ref{fig:ee18btech11007}
%
\begin{figure}[!h]
  \includegraphics[width=\columnwidth]{./figs/ee18btech11007.eps}
  \caption{}
  \label{fig:ee18btech11007}
\end{figure}
%
\item Find the point at which the Nyquist plot of G(s) passes through the negative real axis
\\
\solution  Nyquist plot cuts the negative real axis at $\omega $ for which 
\begin{align}
\angle G(\j\omega)=-\pi
\label{eq:ee18btech11007_system_neg_real}
\end{align}
From \eqref{eq:ee18btech11007_system},
\begin{align}
 G(\j\omega)&=\frac{\pi e^{-\frac{\j\omega}{4}}}{\j\omega} = \frac{\pi e^{-\j\brak{\frac{\omega}{4}+\frac{\pi}{2}}}}{\omega}
\\
\implies \angle{ G(\j\omega)} &= -\brak{\frac{\omega}{4}+\frac{\pi}{2}}
\label{eq:ee18btech11007_system_ang}
\end{align}
From \eqref{eq:ee18btech11007_system_ang} and \eqref{eq:ee18btech11007_system_neg_real}, 
\begin{align}
\frac{\omega}{4}+\frac{\pi}{2} &= \pi
\\
\implies \omega = 2\pi
\end{align}
Also, from \eqref{eq:ee18btech11007_system},
\begin{align}
\label{eq:ee18btech11007_system_mod}
\abs{ G(\j\omega)}&=\frac{\pi }{\abs{\omega}}
\\
\implies \abs{ G(\j2\pi)} &= \frac{1}{2}
\end{align}
%
%\item Find the value of $P$ defined in Table \ref{table:ee18btech11007} from Fig.  \ref{fig:ee18btech11007}.

%
%\solution $P = 0$.
%\item Find the value of $N$ defined in Table \ref{table:ee18btech11007} from  \eqref{eq:ee18btech11007_system}
%\\
%\solution $\because H(s) = 1$, $G(s)H(s) = G(s)$. Also, $G(s)$ has a pole at $s = 0$, hence $N = 0$.
\item Use the Nyquist Stability criterion to determine if the system in \eqref{eq:ee18btech11007_system_ang} is stable.
\begin{table}[!ht]
\centering
\begin{enumerate}[label=\thesection.\arabic*.,ref=\thesection.\theenumi]
\numberwithin{equation}{enumi}
\item The open loop transfer function of a unity feedback system is given by
\begin{align}
\label{eq:ee18btech11007_system}
 G(s)=\frac{\pi e^{-0.25s}}{s}
\end{align}
\item Find $\text{Re} \cbrak{G(\j \omega)}$ and $\text{Im} \cbrak{G(\j \omega)}$.
\\
\solution From \eqref{eq:ee18btech11007_system},
%
\begin{align}
G(j\omega)&=\frac{\pi}{\omega}(-\sin{0.25\omega}-j\cos{0.25\omega})
\\
\implies  \text{Re} \cbrak{G(\j \omega)}&=\frac{\pi}{\omega}(-\sin{0.25\omega}) 
\\
 \text{Im} \cbrak{G(\j \omega)}&=\frac{\pi}{\omega}(-j\cos{0.25\omega}) 
\end{align}
%
\item Sketch the Nyquist plot.
\\
\solution The Nyquist plot is a graph of $\text{Re} \cbrak{G(\j \omega)}$  vs $\text{Im} \cbrak{G(\j \omega)}$.
The following python code generates the Nyquist plot in Fig.  \ref{fig:ee18btech11007}
%
\begin{figure}[!h]
  \includegraphics[width=\columnwidth]{./figs/ee18btech11007.eps}
  \caption{}
  \label{fig:ee18btech11007}
\end{figure}
%
\item Find the point at which the Nyquist plot of G(s) passes through the negative real axis
\\
\solution  Nyquist plot cuts the negative real axis at $\omega $ for which 
\begin{align}
\angle G(\j\omega)=-\pi
\label{eq:ee18btech11007_system_neg_real}
\end{align}
From \eqref{eq:ee18btech11007_system},
\begin{align}
 G(\j\omega)&=\frac{\pi e^{-\frac{\j\omega}{4}}}{\j\omega} = \frac{\pi e^{-\j\brak{\frac{\omega}{4}+\frac{\pi}{2}}}}{\omega}
\\
\implies \angle{ G(\j\omega)} &= -\brak{\frac{\omega}{4}+\frac{\pi}{2}}
\label{eq:ee18btech11007_system_ang}
\end{align}
From \eqref{eq:ee18btech11007_system_ang} and \eqref{eq:ee18btech11007_system_neg_real}, 
\begin{align}
\frac{\omega}{4}+\frac{\pi}{2} &= \pi
\\
\implies \omega = 2\pi
\end{align}
Also, from \eqref{eq:ee18btech11007_system},
\begin{align}
\label{eq:ee18btech11007_system_mod}
\abs{ G(\j\omega)}&=\frac{\pi }{\abs{\omega}}
\\
\implies \abs{ G(\j2\pi)} &= \frac{1}{2}
\end{align}
%
%\item Find the value of $P$ defined in Table \ref{table:ee18btech11007} from Fig.  \ref{fig:ee18btech11007}.

%
%\solution $P = 0$.
%\item Find the value of $N$ defined in Table \ref{table:ee18btech11007} from  \eqref{eq:ee18btech11007_system}
%\\
%\solution $\because H(s) = 1$, $G(s)H(s) = G(s)$. Also, $G(s)$ has a pole at $s = 0$, hence $N = 0$.
\item Use the Nyquist Stability criterion to determine if the system in \eqref{eq:ee18btech11007_system_ang} is stable.
\begin{table}[!ht]
\centering
\begin{enumerate}[label=\thesection.\arabic*.,ref=\thesection.\theenumi]
\numberwithin{equation}{enumi}
\item The open loop transfer function of a unity feedback system is given by
\begin{align}
\label{eq:ee18btech11007_system}
 G(s)=\frac{\pi e^{-0.25s}}{s}
\end{align}
\item Find $\text{Re} \cbrak{G(\j \omega)}$ and $\text{Im} \cbrak{G(\j \omega)}$.
\\
\solution From \eqref{eq:ee18btech11007_system},
%
\begin{align}
G(j\omega)&=\frac{\pi}{\omega}(-\sin{0.25\omega}-j\cos{0.25\omega})
\\
\implies  \text{Re} \cbrak{G(\j \omega)}&=\frac{\pi}{\omega}(-\sin{0.25\omega}) 
\\
 \text{Im} \cbrak{G(\j \omega)}&=\frac{\pi}{\omega}(-j\cos{0.25\omega}) 
\end{align}
%
\item Sketch the Nyquist plot.
\\
\solution The Nyquist plot is a graph of $\text{Re} \cbrak{G(\j \omega)}$  vs $\text{Im} \cbrak{G(\j \omega)}$.
The following python code generates the Nyquist plot in Fig.  \ref{fig:ee18btech11007}
%
\begin{figure}[!h]
  \includegraphics[width=\columnwidth]{./figs/ee18btech11007.eps}
  \caption{}
  \label{fig:ee18btech11007}
\end{figure}
%
\item Find the point at which the Nyquist plot of G(s) passes through the negative real axis
\\
\solution  Nyquist plot cuts the negative real axis at $\omega $ for which 
\begin{align}
\angle G(\j\omega)=-\pi
\label{eq:ee18btech11007_system_neg_real}
\end{align}
From \eqref{eq:ee18btech11007_system},
\begin{align}
 G(\j\omega)&=\frac{\pi e^{-\frac{\j\omega}{4}}}{\j\omega} = \frac{\pi e^{-\j\brak{\frac{\omega}{4}+\frac{\pi}{2}}}}{\omega}
\\
\implies \angle{ G(\j\omega)} &= -\brak{\frac{\omega}{4}+\frac{\pi}{2}}
\label{eq:ee18btech11007_system_ang}
\end{align}
From \eqref{eq:ee18btech11007_system_ang} and \eqref{eq:ee18btech11007_system_neg_real}, 
\begin{align}
\frac{\omega}{4}+\frac{\pi}{2} &= \pi
\\
\implies \omega = 2\pi
\end{align}
Also, from \eqref{eq:ee18btech11007_system},
\begin{align}
\label{eq:ee18btech11007_system_mod}
\abs{ G(\j\omega)}&=\frac{\pi }{\abs{\omega}}
\\
\implies \abs{ G(\j2\pi)} &= \frac{1}{2}
\end{align}
%
%\item Find the value of $P$ defined in Table \ref{table:ee18btech11007} from Fig.  \ref{fig:ee18btech11007}.

%
%\solution $P = 0$.
%\item Find the value of $N$ defined in Table \ref{table:ee18btech11007} from  \eqref{eq:ee18btech11007_system}
%\\
%\solution $\because H(s) = 1$, $G(s)H(s) = G(s)$. Also, $G(s)$ has a pole at $s = 0$, hence $N = 0$.
\item Use the Nyquist Stability criterion to determine if the system in \eqref{eq:ee18btech11007_system_ang} is stable.
\begin{table}[!ht]
\centering
\input{./tables/ee18btech11007.tex}
\caption{}
\label{table:ee18btech11007}
\end{table}
\\
\solution Consider Table \ref{table:ee18btech11007}.  According to the Nyquist stability criterion, 
\begin{enumerate}
\item If the open-loop transfer function $G(s)$ has a zero pole of multiplicity $l$, then the Nyquist plot has a discontinuity at $\omega$ =0. During further analysis it should be assumed that the phasor travels l times clock-wise along a semicircle of infinite radius. After applying this rule, the zero poles should be neglected, i.e. if there are no other unstable poles, then the open-loop transfer function $G(s)$ should be considered stable.
\item If the open-loop transfer function $G(s)$ is stable, then the closed-loop system is unstable for any encirclement of the point -1.
If the open-loop transfer function $G(s)$ is unstable, then there must be one counter clock-wise encirclement of -1 for each pole of $G(s)$ in the right-half of the complex plane.
\label{them:ee18btech11007_nyquist3}
\item The number of surplus encirclements (N + P greater than 0) is exactly the number of unstable poles of the closed-loop system.
\item However, if the graph happens to pass through the point $-1+\j0$, then deciding upon even the marginal stability of the system becomes difficult and the only conclusion that can be drawn from the graph is that there exist zeros on the $\j \omega$  axis.
\end{enumerate}
From \eqref{eq:ee18btech11007_system}, $G(s)$ is stable since it has a single pole at $s = 0$.  Further,  from Fig.  \ref{fig:ee18btech11007}, the Nyquist plot doesnot encircle $s =  -1$.  From  Theorem \ref{them:ee18btech11007_nyquist3}, we may conclude that the system is stable.


\end{enumerate}

\caption{}
\label{table:ee18btech11007}
\end{table}
\\
\solution Consider Table \ref{table:ee18btech11007}.  According to the Nyquist stability criterion, 
\begin{enumerate}
\item If the open-loop transfer function $G(s)$ has a zero pole of multiplicity $l$, then the Nyquist plot has a discontinuity at $\omega$ =0. During further analysis it should be assumed that the phasor travels l times clock-wise along a semicircle of infinite radius. After applying this rule, the zero poles should be neglected, i.e. if there are no other unstable poles, then the open-loop transfer function $G(s)$ should be considered stable.
\item If the open-loop transfer function $G(s)$ is stable, then the closed-loop system is unstable for any encirclement of the point -1.
If the open-loop transfer function $G(s)$ is unstable, then there must be one counter clock-wise encirclement of -1 for each pole of $G(s)$ in the right-half of the complex plane.
\label{them:ee18btech11007_nyquist3}
\item The number of surplus encirclements (N + P greater than 0) is exactly the number of unstable poles of the closed-loop system.
\item However, if the graph happens to pass through the point $-1+\j0$, then deciding upon even the marginal stability of the system becomes difficult and the only conclusion that can be drawn from the graph is that there exist zeros on the $\j \omega$  axis.
\end{enumerate}
From \eqref{eq:ee18btech11007_system}, $G(s)$ is stable since it has a single pole at $s = 0$.  Further,  from Fig.  \ref{fig:ee18btech11007}, the Nyquist plot doesnot encircle $s =  -1$.  From  Theorem \ref{them:ee18btech11007_nyquist3}, we may conclude that the system is stable.


\end{enumerate}

\caption{}
\label{table:ee18btech11007}
\end{table}
\\
\solution Consider Table \ref{table:ee18btech11007}.  According to the Nyquist stability criterion, 
\begin{enumerate}
\item If the open-loop transfer function $G(s)$ has a zero pole of multiplicity $l$, then the Nyquist plot has a discontinuity at $\omega$ =0. During further analysis it should be assumed that the phasor travels l times clock-wise along a semicircle of infinite radius. After applying this rule, the zero poles should be neglected, i.e. if there are no other unstable poles, then the open-loop transfer function $G(s)$ should be considered stable.
\item If the open-loop transfer function $G(s)$ is stable, then the closed-loop system is unstable for any encirclement of the point -1.
If the open-loop transfer function $G(s)$ is unstable, then there must be one counter clock-wise encirclement of -1 for each pole of $G(s)$ in the right-half of the complex plane.
\label{them:ee18btech11007_nyquist3}
\item The number of surplus encirclements (N + P greater than 0) is exactly the number of unstable poles of the closed-loop system.
\item However, if the graph happens to pass through the point $-1+\j0$, then deciding upon even the marginal stability of the system becomes difficult and the only conclusion that can be drawn from the graph is that there exist zeros on the $\j \omega$  axis.
\end{enumerate}
From \eqref{eq:ee18btech11007_system}, $G(s)$ is stable since it has a single pole at $s = 0$.  Further,  from Fig.  \ref{fig:ee18btech11007}, the Nyquist plot doesnot encircle $s =  -1$.  From  Theorem \ref{them:ee18btech11007_nyquist3}, we may conclude that the system is stable.


\end{enumerate}


\section{Compensators}
\begin{enumerate}[label=\thesection.\arabic*.,ref=\thesection.\theenumi]
\numberwithin{equation}{enumi} 
\item 
The Transfer function of Phase Lead Compensator is given by \\

\begin{align}
D(s) = \frac{3(s+\frac{1}{3T})}{(s+\frac{1}{T})}
\end{align}

Find out the frequency (in rad/sec), at which $\angle D(j\omega)$ is maximum? \\
\label{prob:ee18btech11010_comp}
\solution
The basic requirement of the phase lead network is that all poles and zeros
of the transfer function of the network must lie on negative real axis
interlacing each other with a zero located as the nearest point to origin.

Substituting $s = j\omega$ in D(s), we get \\

\begin{align}
D(j\omega) = \frac{3(j\omega+\frac{1}{3T})}{(j\omega+\frac{1}{T})}
\end{align}

The phase of this transfer function $\phi(\omega)$ is given by,
\begin{align}
\phi(\omega) = \tan^{-1}(3\omega T)-\tan^{-1}(\omega T)
\end{align}

$\phi(\omega)$ has its maximum at $\omega_c$ Where $\phi '(\omega_c)=0$,

\begin{align}
\phi '(\omega_c) = 0 = \frac{3T}{1+(3\omega _c T)^2}-\frac{T}{1+(\omega _c T)^2}
\end{align}

After solving and Simplification , we have \\

\begin{align}
\omega _c ^2T^2 = \frac{1}{3}
\end{align}

\begin{align}
\omega _c = \sqrt{\frac{1}{3T^2}}
\end{align}

\item Verify your result through a plot.
\\
\solution 
The following plots the Phase value of the transfer function, 

\begin{figure}[htp]
	\centering
	\includegraphics[width=\columnwidth]{./figs/ee18btech11010.eps}
	\caption{}
	\label{fig:ee18btech11010}
\end{figure}
 
\textbf{Applications:}\\ \\
\begin{enumerate}
  \item Phase lead Compensators can be used as High pass filters,Differentiators.
  \item They are used to reduce steady state errors. 
  \item Increases Phase Margin , relative stability.
\end{enumerate}
\item What is purpose of of a Phase Lead Compensator?
\item Through an example, show how the compensator in Problem \ref{prob:ee18btech11010_comp} can be used in a control system.



\end{enumerate}



\section{Phase Margin}
\begin{enumerate}[label=\thesection.\arabic*.,ref=\thesection.\theenumi]
\numberwithin{equation}{enumi}

\item The open loop transfer function of a system is 
\begin{align}
G(s) = \frac{2}{(s+1)(s+2)}
\label{eq:ee18btech11017_system}
\end{align}
Find its magnitude and phase response.
\\
\solution Substituting $s = \j\omega$ in \eqref{eq:ee18btech11017_system},

\begin{align}
G\brak{\j\omega}&=\frac{1}{\brak{j\omega+1}\brak{j\omega+2}} 
\\
\implies 
\abs{G\brak{\j\omega}}&=\frac{2}{\brak{\sqrt{\omega^2+1}}\brak{\sqrt{\omega^2+4}}}
\label{eq:ee18btech11017_gain}
\\
\angle G\brak{\j\omega}&=- \tan^{-1}(\omega) - \tan^{-1}\brak{\frac{\omega}{2}} 
\label{eq:ee18btech11017_phase}
\end{align}

\item Find $\omega$ for which the gain of \eqref{eq:ee18btech11017_system} first becomes 1.
\\
\solution From \eqref{eq:ee18btech11017_gain}

\begin{align}
\abs{G\brak{\j\omega}}&=1
\\
\implies \frac{2}{\brak{\sqrt{\omega^2+1}}\brak{\sqrt{\omega^2+4}}}&=1
\\
\implies \omega_{gc}&=0
\end{align}
which is the desired frequency.

\item Find $\angle G(\j\omega_{gc}) + 180\degree$.  This is known as the {\em phase margin}(PM)
\\
\solution From \eqref{eq:ee18btech11017_phase},
%
\begin{align}
\angle G\brak{\j\omega} = 0\degree
\implies PM=180\degree
\label{eq:ee18btech11017_pm}
\end{align}
%
\item Verify your result by plotting the gain and phase plots of $G(\j\omega)$.
\\
\solution The following code plots Fig. \ref{fig:ee18btech11017}

\begin{lstlisting}
codes/ee18btech11017.py
\end{lstlisting}
%
The Phase plot is as shown,
\begin{figure}[!h]
  \includegraphics[width=\columnwidth]{./figs/ee18btech11017/ee18btech11017.eps}
  \caption{}
  \label{fig:ee18btech11017}
\end{figure}

\item A positive phase margin for the open loop system indicates a stable closed loop system.  \eqref{eq:ee18btech11017_pm} indicates that $G(s)$ with unity feedback is stable.  Show that the roots of $1+G(s)$ lie in the left half plane proving closed loop stability.
\\
\solution  Let the closed loop transfer function
\begin{align}
T(s)=\frac{G(s)}{1+G(s)}
\end{align}
Then
\begin{align}
1+G(s)&=0 
\\
\implies s^{2}+3s+4&=0 
\\
\text{or } s&=-1.5+1.3j,-1.5-1.3j
\end{align}
Since the roots are in the left half plane, the system is stable.
%

\item Instead of unity feedback, consider a system with 
%
\begin{align}
H(s)=\frac{50}{s+1}
\end{align}
%
Compute the open loop phase margin for this system.
\\
\solution 
%
\begin{align}
\because G(s)H(s)=\frac{100}{(s+1)^{2}(s+2)},
\label{eq:ee18btech11017_gh}
\end{align}
%
the  magnitude and phase are
\begin{align}
\label{eq:ee18btech11017_gh_mag}
\abs{G\brak{\j\omega}H\brak{\j\omega}}&=\frac{10^{2}}{ \sqrt{(\omega^{2}+1)^{2}}
\sqrt{\omega^{2}+4}} \\
\angle G\brak{\j\omega}H\brak{\j\omega}&=-\tan^{-1}\frac{\omega}{2}-2\tan^{-1}(\omega) 
\label{eq:ee18btech11017_gh_ang}
\end{align}
%
The gain crossover frequency is given by 
\begin{align}
\frac{10^{2}}{\sqrt{\omega_{gc}^{2}+4} \sqrt{(\omega_{gc}^{2}+10^{2})^{2}}}&=1 \\
\\
\omega_{gc}^{6}+6\omega_{gc}^{4}+9\omega_{gc}^{2}-9996&=0 
\\
\implies \omega_{gc} &= 4.42  
\label{eq:ee18btech11017_gh_wgc}
\end{align}
%
From \eqref{eq:ee18btech11017_gh_ang} and \eqref{eq:ee18btech11017_gh_wgc},
the phase margin is
\begin{align}
PM &=180\degree-2\tan^{-1}(\omega_{gc})-\tan^{-1}\brak{\frac{\omega_{gc}}{2}} \\
\implies  P.M &=-40.15\degree
\label{eq:ee18btech11017_gh_pm}
\end{align}
\item Verify your result through the magnitude and phase plot.
\\
\solution The following code plots Fig. \ref{fig:ee18btech11017_2}
\begin{lstlisting}
codes/ee18btech11017_2.py
\end{lstlisting}
%\begin{figure}[!h]
%  \includegraphics[width=\columnwidth]{./figures/ee18btech11017/ee18btech11017_2.eps}
%  \caption{}
%  \label{fig:ee18btech11017_2}
%\end{figure}
%
\item Since the PM in \eqref{eq:ee18btech11017_gh_pm} is negative, the closed loop system is unstable .
Verify this using the Routh-Hurwitz criterion.
%
\\
\solution 
The characteristic equation is 
\begin{align}
1+G(s)H(s)&=0 
\\
\implies s^{3}+4s^{2}+5s+102=0   
\label{eq:ee18btech11017_gh_char}
\end{align}
Constructing the routh array for \eqref{eq:ee18btech11017_gh_char},

\begin{align}
\mydet{s^3\\s^2\\s}
\mydet{1 & 5 & 0 \\ 4 & 102 & 0 \\ -20.5 & 0 & 0}
\\
\mydet{s^3\\s^2\\s\\s^0}
\mydet{1 & 5 & 0 \\ 4 & 102 & 0 \\ -20.5 & 0 & 0 \\ 102 & 0 & 0}
\end{align}

$\because $ there are two sign changes in the first column of the routh array,  two  poles lie on right half of s-plane.  Therefore,the system is unstable.


\end{enumerate}

\section{Gain Margin}
\begin{enumerate}[label=\thesection.\arabic*.,ref=\thesection.\theenumi]
\numberwithin{equation}{enumi} 
\item
 
The open loop transfer function of a feedback control system is  
\begin{align}
G(s) = \frac{1}{s(1+2s)(1+s)} 
\end{align}
Find the gain margin of this system and analyse the stability.
 \\


\solution
\\
\textbf{Gain Margin}:The greater the Gain Margin (GM), the greater the stability of the system. The gain margin refers to the amount of gain, which can be increased or decreased without making the system unstable. It is usually expressed as a magnitude in dB.
\\

We can usually read the gain margin directly from the Bode plot. This is done by calculating the vertical distance between the magnitude curve (on the Bode magnitude plot) and the x-axis at the frequency where the Bode phase plot = 180$^{\circ}$. This point is known as the phase crossover frequency.

Gain Margin is given by,
\begin{align}
G.M = -20log_{10}|G(j\omega_{pc})| = 20log_{10}k_{g}
\end{align}
where 
\begin{align}
k_{g}=\frac{1}{|G(j\omega_{pc})|} 
\end{align}
\\

Now let's put s = j$\omega$ in the equation of G(s) :
\begin{align}
G(j\omega) = \frac{1}{j\omega(1+2j\omega)(1+j\omega)} 
\end{align}
So,
\begin{align}
G(j\omega) = \frac{1}{j\omega(1+3j\omega-2\omega^2)}=\frac{1}{j\omega-3\omega^2-2j\omega^3}
\end{align}
Hence ,
\begin{align}
G(j\omega) = \frac{1}{-3\omega^2+j\omega(1-2\omega^2)} 
\end{align}

Now we know that $\omega_{pc}$ is the Phase crossover frequency (The frequency at which the phase of open-loop transfer function reaches -180$^{\circ}$ or +180$^{\circ}$ depending upon the range of tan inverse function).\\
Now,
\begin{align}
\angle G(j\omega)=- tan^{-1}(\frac{\omega(1-2\omega^2)}{-3\omega^2})
\end{align}
So,at $\omega=\omega_{pc}$ :
\begin{align}
\omega(1-2\omega^2) = 0 
\end{align}
i.e. the imaginary part of G(j$\omega$) = 0.So ,
\begin{align}
\omega_{pc} = \frac{1}{\sqrt{2}} 
\end{align}
as $\omega_{pc}$ should be positive and $\omega_{pc}$ shuld not be equal to zero.So now G(j$\omega_{pc}$) will be :
\begin{align}
G(j\omega_{pc}) = \frac{1}{-3\omega_{pc}^2}
\end{align}
i.e,
\begin{align}
|G(j\omega_{pc})| = \frac{1}{(\frac{3}{2})}
\end{align}
\begin{align}
k_{g}=\frac{1}{|G(j\omega_{pc})|} = \frac{3}{2}=1.5
\end{align}
So , Gain margin in terms of dB is :

\begin{align}
20log_{10}1.5 = 3.5dB
\end{align}
\\

Plot obtained for verification in python :

(You can download code from codes/ee18btech11016.py)
\begin{figure}[htp]
	\centering
	\includegraphics[width=\columnwidth,scale=2]{./figs/fig.eps}
	\caption{}
	\label{fig:Phase}
\end{figure}
\\

\section{Stability}

So,in the above figure, since $20log_{10}(G(j\omega_{pc}))$ = -3.5dB at $\omega_{pc} = -180^{\circ}$ so G.M = +3.5dB And since the gain margin is positive we can say that the system is stable more precisely the system is marginally stable as one of the pole lies on the imaginary axis.(Because for stability , both gain and phase margin should be positive.)
\\

\textbf{Analysis}:
Now,Let's try to analyse the stabilty of the system using Routh-Hurwitz Criterion :
\\

Let $T(s)$ be Closed loop transfer function ,
\begin{align}
T(s)=\frac{N(s)}{D(s)}=\frac{G(s)}{1+G(s)}
\end{align}
The characteristic equation is 
\begin{align}
D(s)=0  \\
1+G(s)=0 
\end{align}
So,the closed loop transfer function is given by 
\begin{align}
T(s) = \frac{1}{1 + (s(1+2s)(1+s))}
\end{align}

\begin{align}
=> D(s) = 1 + s(1+s)(1+2s) = 2s^3 + 3s^2 + s + 1 
\end{align}
\\
So,the characteristics equation is given by D(s) = 0.i.e,
\begin{align}
=> 2s^3 + 3s^2 + s + 1 = 0 
\end{align}
Constructing routh array for above equation of D(s),
\begin{align}
\mydet{s^3\\s^2\\s}
\mydet{2 & 1 & 0 \\ 3 & 1 & 0 \\ (1/3) & 0 & 0}
\end{align}\\

\begin{align}
\mydet{s^3\\s^2\\s\\s^0}
\mydet{2 & 1 & 0 \\ 3 & 1 & 0 \\ (1/3) & 0 & 0 \\ 1 & 0 & 0}
\end{align}\\


There are no sign changes in the first column of the routh array. So, no poles lie on right half of s-plane. 
\\
Therefore,the system is stable.\\

You can download the code for Routh-Hurwitz array from (codes/RH.py)
\\

Hence,we can say that from both Routh-hurwitz criterion and from the gain margin concept we are getting the same answers.

\end{enumerate}





\end{document}


