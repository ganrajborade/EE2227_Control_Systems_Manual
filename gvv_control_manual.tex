\documentclass[journal,12pt,twocolumn]{IEEEtran}
%
\usepackage{setspace}
\usepackage{gensymb}
%\doublespacing
\singlespacing

%\usepackage{graphicx}
%\usepackage{amssymb}
%\usepackage{relsize}
\usepackage[cmex10]{amsmath}
%\usepackage{amsthm}
%\interdisplaylinepenalty=2500
%\savesymbol{iint}
%\usepackage{txfonts}
%\restoresymbol{TXF}{iint}
%\usepackage{wasysym}
\usepackage{amsthm}
%\usepackage{iithtlc}
\usepackage{mathrsfs}
\usepackage{txfonts}
\usepackage{stfloats}
\usepackage{bm}
\usepackage{cite}
\usepackage{cases}
\usepackage{subfig}
%\usepackage{xtab}
\usepackage{longtable}
\usepackage{multirow}
%\usepackage{algorithm}
%\usepackage{algpseudocode}
\usepackage{enumitem}
\usepackage{mathtools}
\usepackage{tikz}
\usepackage{circuitikz}
\usepackage{verbatim}
%\usepackage{tfrupee}
%\usepackage[breaklinks=true]{hyperref}
%\usepackage{stmaryrd}
%\usepackage{tkz-euclide} % loads  TikZ and tkz-base
%\usetkzobj{all}
%\usepackage{listings}
%    \usepackage{color}                                            %%
%    \usepackage{array}                                            %%
%    \usepackage{longtable}                                        %%
%    \usepackage{calc}                                             %%
%    \usepackage{multirow}                                         %%
%    \usepackage{hhline}                                           %%
%    \usepackage{ifthen}                                           %%
  %optionally (for landscape tables embedded in another document): %%
%    \usepackage{lscape}     
%\usepackage{multicol}
%\usepackage{chngcntr}
%\usepackage{enumerate}
%\usepackage{wasysym}
%\newcounter{MYtempeqncnt}
\DeclareMathOperator*{\Res}{Res}
%\renewcommand{\baselinestretch}{2}
\renewcommand\thesection{\arabic{section}}
\renewcommand\thesubsection{\thesection.\arabic{subsection}}
\renewcommand\thesubsubsection{\thesubsection.\arabic{subsubsection}}

\renewcommand\thesectiondis{\arabic{section}}
\renewcommand\thesubsectiondis{\thesectiondis.\arabic{subsection}}
\renewcommand\thesubsubsectiondis{\thesubsectiondis.\arabic{subsubsection}}

% correct bad hyphenation here
\hyphenation{op-tical net-works semi-conduc-tor}
\def\inputGnumericTable{}                                 %%

%\lstset{
%language=C,
%frame=single, 
%breaklines=true,
%columns=fullflexible
%}
%\lstset{
%language=tex,
%frame=single, 
%breaklines=true
%}

\begin{document}
%


\newtheorem{theorem}{Theorem}[section]
\newtheorem{problem}{Problem}
\newtheorem{proposition}{Proposition}[section]
\newtheorem{lemma}{Lemma}[section]
\newtheorem{corollary}[theorem]{Corollary}
\newtheorem{example}{Example}[section]
\newtheorem{definition}[problem]{Definition}
%\newtheorem{thm}{Theorem}[section] 
%\newtheorem{defn}[thm]{Definition}
%\newtheorem{algorithm}{Algorithm}[section]
%\newtheorem{cor}{Corollary}
\newcommand{\BEQA}{\begin{eqnarray}}
\newcommand{\EEQA}{\end{eqnarray}}
\newcommand{\define}{\stackrel{\triangle}{=}}
\bibliographystyle{IEEEtran}
%\bibliographystyle{ieeetr}
\providecommand{\mbf}{\mathbf}
\providecommand{\pr}[1]{\ensuremath{\Pr\left(#1\right)}}
\providecommand{\qfunc}[1]{\ensuremath{Q\left(#1\right)}}
\providecommand{\sbrak}[1]{\ensuremath{{}\left[#1\right]}}
\providecommand{\lsbrak}[1]{\ensuremath{{}\left[#1\right.}}
\providecommand{\rsbrak}[1]{\ensuremath{{}\left.#1\right]}}
\providecommand{\brak}[1]{\ensuremath{\left(#1\right)}}
\providecommand{\lbrak}[1]{\ensuremath{\left(#1\right.}}
\providecommand{\rbrak}[1]{\ensuremath{\left.#1\right)}}
\providecommand{\cbrak}[1]{\ensuremath{\left\{#1\right\}}}
\providecommand{\lcbrak}[1]{\ensuremath{\left\{#1\right.}}
\providecommand{\rcbrak}[1]{\ensuremath{\left.#1\right\}}}
\theoremstyle{remark}
\newtheorem{rem}{Remark}
\newcommand{\sgn}{\mathop{\mathrm{sgn}}}
\providecommand{\abs}[1]{\left\vert#1\right\vert}
\providecommand{\res}[1]{\Res\displaylimits_{#1}} 
\providecommand{\norm}[1]{\left\lVert#1\right\rVert}
%\providecommand{\norm}[1]{\lVert#1\rVert}
\providecommand{\mtx}[1]{\mathbf{#1}}
\providecommand{\mean}[1]{E\left[ #1 \right]}
\providecommand{\fourier}{\overset{\mathcal{F}}{ \rightleftharpoons}}
%\providecommand{\hilbert}{\overset{\mathcal{H}}{ \rightleftharpoons}}
\providecommand{\system}{\overset{\mathcal{H}}{ \longleftrightarrow}}
	%\newcommand{\solution}[2]{\textbf{Solution:}{#1}}
\newcommand{\solution}{\noindent \textbf{Solution: }}
\newcommand{\cosec}{\,\text{cosec}\,}
\providecommand{\dec}[2]{\ensuremath{\overset{#1}{\underset{#2}{\gtrless}}}}
\newcommand{\myvec}[1]{\ensuremath{\begin{pmatrix}#1\end{pmatrix}}}
\newcommand{\mydet}[1]{\ensuremath{\begin{vmatrix}#1\end{vmatrix}}}
%\numberwithin{equation}{section}
\numberwithin{equation}{subsection}
%\numberwithin{problem}{section}
%\numberwithin{definition}{section}
\makeatletter
\@addtoreset{figure}{problem}
\makeatother
\let\StandardTheFigure\thefigure
\let\vec\mathbf
%\renewcommand{\thefigure}{\theproblem.\arabic{figure}}
\renewcommand{\thefigure}{\theproblem}
%\setlist[enumerate,1]{before=\renewcommand\theequation{\theenumi.\arabic{equation}}
%\counterwithin{equation}{enumi}
%\renewcommand{\theequation}{\arabic{subsection}.\arabic{equation}}
\def\putbox#1#2#3{\makebox[0in][l]{\makebox[#1][l]{}\raisebox{\baselineskip}[0in][0in]{\raisebox{#2}[0in][0in]{#3}}}}
     \def\rightbox#1{\makebox[0in][r]{#1}}
     \def\centbox#1{\makebox[0in]{#1}}
     \def\topbox#1{\raisebox{-\baselineskip}[0in][0in]{#1}}
     \def\midbox#1{\raisebox{-0.5\baselineskip}[0in][0in]{#1}}
\vspace{3cm}
\title{
%	\logo{
Control Systems
%	}
}
\author{ G V V Sharma$^{*}$% <-this % stops a space
	\thanks{*The author is with the Department
		of Electrical Engineering, Indian Institute of Technology, Hyderabad
		502285 India e-mail:  gadepall@iith.ac.in. All content in this manual is released under GNU GPL.  Free and open source.}
	
}	
%\title{
%	\logo{Matrix Analysis through Octave}{\begin{center}\includegraphics[scale=.24]{tlc}\end{center}}{}{HAMDSP}
%}
% paper title
% can use linebreaks \\ within to get better formatting as desired
%\title{Matrix Analysis through Octave}
%
%
% author names and IEEE memberships
% note positions of commas and nonbreaking spaces ( ~ ) LaTeX will not break
% a structure at a ~ so this keeps an author's name from being broken across
% two lines.
% use \thanks{} to gain access to the first footnote area
% a separate \thanks must be used for each paragraph as LaTeX2e's \thanks
% was not built to handle multiple paragraphs
%
%\author{<-this % stops a space
%\thanks{}}
%}
% note the % following the last \IEEEmembership and also \thanks - 
% these prevent an unwanted space from occurring between the last author name
% and the end of the author line. i.e., if you had this:
% 
% \author{....lastname \thanks{...} \thanks{...} }
%                     ^------------^------------^----Do not want these spaces!
%
% a space would be appended to the last name and could cause every name on that
% line to be shifted left slightly. This is one of those "LaTeX things". For
% instance, "\textbf{A} \textbf{B}" will typeset as "A B" not "AB". To get
% "AB" then you have to do: "\textbf{A}\textbf{B}"
% \thanks is no different in this regard, so shield the last } of each \thanks
% that ends a line with a % and do not let a space in before the next \thanks.
% Spaces after \IEEEmembership other than the last one are OK (and needed) as
% you are supposed to have spaces between the names. For what it is worth,
% this is a minor point as most people would not even notice if the said evil
% space somehow managed to creep in.
% The paper headers
%\markboth{Journal of \LaTeX\ Class Files,~Vol.~6, No.~1, January~2007}%
%{Shell \MakeLowercase{\textit{et al.}}: Bare Demo of IEEEtran.cls for Journals}
% The only time the second header will appear is for the odd numbered pages
% after the title page when using the twoside option.
% 
% * Note that you probably will NOT want to include the author's *
% * name in the headers of peer review papers.                   *
% You can use \ifCLASSOPTIONpeerreview for conditional compilation here if
% you desire.
% If you want to put a publisher's ID mark on the page you can do it like
% this:
%\IEEEpubid{0000--0000/00\$00.00~\copyright~2007 IEEE}
% Remember, if you use this you must call \IEEEpubidadjcol in the second
% column for its text to clear the IEEEpubid mark.
% make the title area
\maketitle
\newpage
\tableofcontents
\bigskip
\renewcommand{\thefigure}{\theenumi}
\renewcommand{\thetable}{\theenumi}
%\renewcommand{\theequation}{\theenumi}
%\begin{abstract}
%%\boldmath
%In this letter, an algorithm for evaluating the exact analytical bit error rate  (BER)  for the piecewise linear (PL) combiner for  multiple relays is presented. Previous results were available only for upto three relays. The algorithm is unique in the sense that  the actual mathematical expressions, that are prohibitively large, need not be explicitly obtained. The diversity gain due to multiple relays is shown through plots of the analytical BER, well supported by simulations. 
%
%\end{abstract}
% IEEEtran.cls defaults to using nonbold math in the Abstract.
% This preserves the distinction between vectors and scalars. However,
% if the journal you are submitting to favors bold math in the abstract,
% then you can use LaTeX's standard command \boldmath at the very start
% of the abstract to achieve this. Many IEEE journals frown on math
% in the abstract anyway.
% Note that keywords are not normally used for peerreview papers.
%\begin{IEEEkeywords}
%Cooperative diversity, decode and forward, piecewise linear
%\end{IEEEkeywords}
% For peer review papers, you can put extra information on the cover
% page as needed:
% \ifCLASSOPTIONpeerreview
% \begin{center} \bfseries EDICS Category: 3-BBND \end{center}
% \fi
%
% For peerreview papers, this IEEEtran command inserts a page break and
% creates the second title. It will be ignored for other modes.
%\IEEEpeerreviewmaketitle
\begin{abstract}
This manual is an introduction to control systems based on GATE problems.Links to sample Python codes are %available in the text.  
\end{abstract}
Download python codes using 
%\begin{lstlisting}
%svn co https://github.com/gadepall/school/trunk/control/codes
%\end{lstlisting}
%\section{Bode Plot}
%\begin{enumerate}[label=\thesection.\arabic*.,ref=\thesection.\theenumi]
%\numberwithin{equation}{enumi}
%\begin{enumerate}[label=\thesection.\arabic*.,ref=\thesection.\theenumi]
\numberwithin{equation}{enumi}

\item For an LTI system, the Bode plot for its gain defined as
\begin{align}
	G(s) = 20\log\abs{H(s)}
	\label{eq:ee18btech11001_gain}
\end{align}
is as illustrated in the Fig. \ref{fig:ee18btech11001_bode}. Express $G(f)$ in terms of $f$.\\
\begin{figure}[ht!]
    \includegraphics[width=\columnwidth]{./figs/ee18btech11001/ee18btech11001.eps}
    \caption{}
    \label{fig:ee18btech11001_bode}
\end{figure}\\

\solution
\begin{align}
 G(f) = 
 \begin{cases} 
        100 & 0 < f < 10^{1} \\
      120-20\log(f) & 10 < f < 10^{2} \\
      200-60\log(f) & 10^2 < f < 10^{3} \\
      140-40\log(f) & 10^{3} < f < 10^{4} \\
       -20 & 10^{4} < f < 10^{5} \\
      180-40\log(f) & 10^{5} < f < 10^{6} \\
      300-60\log(f) & 10^{6} < f < 10^{7}   
 \end{cases}
\end{align}

%-----------------------------------------------------------------------%

\item Express the slope of $G(f)$ in terms of $f$.
\\
\solution The desired slope is 
\begin{align}
\nabla G(f) &= \dfrac{d(G(f))}{d(\log(f))}
\end{align}

\begin{align}
 \nabla G(f) = 
 \begin{cases} 
        0 & 0 < f < 10^{1} \\
      -20 & 10 < f < 10^{2} \\
      -60 & 10^{2} < f < 10^{3} \\
      -40 & 10^{3} < f < 10^{4} \\
       0 & 10^{4} < f < 10^{5} \\
      -40 & 10^{5} < f < 10^{6} \\
      -60 & 10^{6} < f < 10^{7}   
 \end{cases}
\end{align}

%-----------------------------------------------------------------------%

\item Express the change of slope of $G(f)$ in terms of $f$.
\\
\solution\\
$\Delta(\nabla G(f))$  = Change of slope G(f) at f

\begin{align}
 \Delta(\nabla G(f)) = 
 \begin{cases} 
      -20 &  f = 10^{1} \\
      -40 &  f = 10^{2} \\
      +20 &  f = 10^{3} \\
      +40 &  f = 10^{4} \\
      -40 &  f = 10^{5} \\
      -20 &  f = 10^{6} 
 \end{cases}
\label{eq:ee18btech11001_slope_diff}
\end{align}

%-----------------------------------------------------------------------%

\item Tabulate the poles and zeros of $H(s)$ using \eqref{eq:ee18btech11001_slope_diff}.
\\
\solution Table \ref{table:ee18btech11001} provides the details.  
%
\begin{table}[!ht]
\centering
\begin{enumerate}[label=\thesection.\arabic*.,ref=\thesection.\theenumi]
\numberwithin{equation}{enumi}

\item For an LTI system, the Bode plot for its gain defined as
\begin{align}
	G(s) = 20\log\abs{H(s)}
	\label{eq:ee18btech11001_gain}
\end{align}
is as illustrated in the Fig. \ref{fig:ee18btech11001_bode}. Express $G(f)$ in terms of $f$.\\
\begin{figure}[ht!]
    \includegraphics[width=\columnwidth]{./figs/ee18btech11001/ee18btech11001.eps}
    \caption{}
    \label{fig:ee18btech11001_bode}
\end{figure}\\

\solution
\begin{align}
 G(f) = 
 \begin{cases} 
        100 & 0 < f < 10^{1} \\
      120-20\log(f) & 10 < f < 10^{2} \\
      200-60\log(f) & 10^2 < f < 10^{3} \\
      140-40\log(f) & 10^{3} < f < 10^{4} \\
       -20 & 10^{4} < f < 10^{5} \\
      180-40\log(f) & 10^{5} < f < 10^{6} \\
      300-60\log(f) & 10^{6} < f < 10^{7}   
 \end{cases}
\end{align}

%-----------------------------------------------------------------------%

\item Express the slope of $G(f)$ in terms of $f$.
\\
\solution The desired slope is 
\begin{align}
\nabla G(f) &= \dfrac{d(G(f))}{d(\log(f))}
\end{align}

\begin{align}
 \nabla G(f) = 
 \begin{cases} 
        0 & 0 < f < 10^{1} \\
      -20 & 10 < f < 10^{2} \\
      -60 & 10^{2} < f < 10^{3} \\
      -40 & 10^{3} < f < 10^{4} \\
       0 & 10^{4} < f < 10^{5} \\
      -40 & 10^{5} < f < 10^{6} \\
      -60 & 10^{6} < f < 10^{7}   
 \end{cases}
\end{align}

%-----------------------------------------------------------------------%

\item Express the change of slope of $G(f)$ in terms of $f$.
\\
\solution\\
$\Delta(\nabla G(f))$  = Change of slope G(f) at f

\begin{align}
 \Delta(\nabla G(f)) = 
 \begin{cases} 
      -20 &  f = 10^{1} \\
      -40 &  f = 10^{2} \\
      +20 &  f = 10^{3} \\
      +40 &  f = 10^{4} \\
      -40 &  f = 10^{5} \\
      -20 &  f = 10^{6} 
 \end{cases}
\label{eq:ee18btech11001_slope_diff}
\end{align}

%-----------------------------------------------------------------------%

\item Tabulate the poles and zeros of $H(s)$ using \eqref{eq:ee18btech11001_slope_diff}.
\\
\solution Table \ref{table:ee18btech11001} provides the details.  
%
\begin{table}[!ht]
\centering
\begin{enumerate}[label=\thesection.\arabic*.,ref=\thesection.\theenumi]
\numberwithin{equation}{enumi}

\item For an LTI system, the Bode plot for its gain defined as
\begin{align}
	G(s) = 20\log\abs{H(s)}
	\label{eq:ee18btech11001_gain}
\end{align}
is as illustrated in the Fig. \ref{fig:ee18btech11001_bode}. Express $G(f)$ in terms of $f$.\\
\begin{figure}[ht!]
    \includegraphics[width=\columnwidth]{./figs/ee18btech11001/ee18btech11001.eps}
    \caption{}
    \label{fig:ee18btech11001_bode}
\end{figure}\\

\solution
\begin{align}
 G(f) = 
 \begin{cases} 
        100 & 0 < f < 10^{1} \\
      120-20\log(f) & 10 < f < 10^{2} \\
      200-60\log(f) & 10^2 < f < 10^{3} \\
      140-40\log(f) & 10^{3} < f < 10^{4} \\
       -20 & 10^{4} < f < 10^{5} \\
      180-40\log(f) & 10^{5} < f < 10^{6} \\
      300-60\log(f) & 10^{6} < f < 10^{7}   
 \end{cases}
\end{align}

%-----------------------------------------------------------------------%

\item Express the slope of $G(f)$ in terms of $f$.
\\
\solution The desired slope is 
\begin{align}
\nabla G(f) &= \dfrac{d(G(f))}{d(\log(f))}
\end{align}

\begin{align}
 \nabla G(f) = 
 \begin{cases} 
        0 & 0 < f < 10^{1} \\
      -20 & 10 < f < 10^{2} \\
      -60 & 10^{2} < f < 10^{3} \\
      -40 & 10^{3} < f < 10^{4} \\
       0 & 10^{4} < f < 10^{5} \\
      -40 & 10^{5} < f < 10^{6} \\
      -60 & 10^{6} < f < 10^{7}   
 \end{cases}
\end{align}

%-----------------------------------------------------------------------%

\item Express the change of slope of $G(f)$ in terms of $f$.
\\
\solution\\
$\Delta(\nabla G(f))$  = Change of slope G(f) at f

\begin{align}
 \Delta(\nabla G(f)) = 
 \begin{cases} 
      -20 &  f = 10^{1} \\
      -40 &  f = 10^{2} \\
      +20 &  f = 10^{3} \\
      +40 &  f = 10^{4} \\
      -40 &  f = 10^{5} \\
      -20 &  f = 10^{6} 
 \end{cases}
\label{eq:ee18btech11001_slope_diff}
\end{align}

%-----------------------------------------------------------------------%

\item Tabulate the poles and zeros of $H(s)$ using \eqref{eq:ee18btech11001_slope_diff}.
\\
\solution Table \ref{table:ee18btech11001} provides the details.  
%
\begin{table}[!ht]
\centering
\input{./tables/ee18btech11001.tex}
\caption{}
\label{table:ee18btech11001}
\end{table}


%-----------------------------------------------------------------------%

\item Obtain the transfer function of $H(s)$.
\\
\solution From Table \ref{table:ee18btech11001},
{\footnotesize
\begin{align}
\label{eq:ee18btech11001_system}
	H(s) = \frac{K(s+j2\pi 10^{3})(s+j2\pi 10^{4})^{2}}{(s+j2\pi 10^{1})(s+j2\pi 10^{2})^{2}(s+j2\pi 10^{5})^{2}(s+j2\pi 10^{6})}
\end{align}
}
\item Justify the above results.
\\
\solution
Let us consider a generalized transfer gain
\begin{align}
	H(s) = k \dfrac{(s-z_{1})(s-z_{2})...(s-z_{m-1})(s-z_{m})}{(s-p_{1})(s-p_{2})....(s-p_{n-1})(s-p_{n})}
\end{align}
The gain
\begin{multline}
	G(f) = 20\log\abs{H(s)} 
\\
= 20\log \abs{k} + 20\log \abs{s-z_{1}} 
	    \\
	    + 20\log \abs{s-z_{2}} + \dots + 20\log \abs{s-z_{m}} 
\\
- 20\log \abs{s-p_{1}} 
	    - 20\log \abs{s-p_{2}} 
	    \\
- \dots - 20\log \abs{s-z_{n}} 
\end{multline}
%
Substituting $s = \j \omega$, for real $z_1$
\begin{align}
	20\log \abs{s-z_{1}} &= 20\log \abs{\sqrt{\omega^{2} + z_{1}^{2}}}
\\
&= 
\begin{cases}
20 \log \abs{z_{1}}, & \omega \ll z_{1}
\\
20 \log \abs{\omega}, & \omega \gg z_{1}
\end{cases}
\end{align}
%
Taking the derivative, 
\begin{align}
	\frac{d\brak{20\log \abs{s-z_{1}}}}{d\brak{\log \abs{\omega}}} 
= 
\begin{cases}
0, & \omega \ll z_{1}
\\
20, & \omega \gg z_{1}
\end{cases}
\end{align}
%
Thus, when a zero is encountered, the gradient of $H(\j\omega)$ jumps by +20 in the log scale.  When a pole is encountered, the gradient falls by -20. Note that this is a very loose justification, but works well in practice.

%-----------------------------------------------------------------------%

\item Obtain the Bode plot and the slope plot for $H(s)$ and verify with  Fig. \ref{fig:ee18btech11001_bode}
\\
\solution Bode Plot of obtained Transfer Function is 
\begin{figure}[htp]
    \centering
    \includegraphics[width=\columnwidth]{./figs/ee18btech11001/ee18btech11001_2.eps}
    \caption{}
    \label{fig:ee18btech11001_2}
\end{figure}
%
Fig. \ref{fig:ee18btech11001}, obtained from  \eqref{eq:ee18btech11001_system},
is a close reconstruction of Fig. \ref{fig:ee18btech11001}.
\end{enumerate}



\caption{}
\label{table:ee18btech11001}
\end{table}


%-----------------------------------------------------------------------%

\item Obtain the transfer function of $H(s)$.
\\
\solution From Table \ref{table:ee18btech11001},
{\footnotesize
\begin{align}
\label{eq:ee18btech11001_system}
	H(s) = \frac{K(s+j2\pi 10^{3})(s+j2\pi 10^{4})^{2}}{(s+j2\pi 10^{1})(s+j2\pi 10^{2})^{2}(s+j2\pi 10^{5})^{2}(s+j2\pi 10^{6})}
\end{align}
}
\item Justify the above results.
\\
\solution
Let us consider a generalized transfer gain
\begin{align}
	H(s) = k \dfrac{(s-z_{1})(s-z_{2})...(s-z_{m-1})(s-z_{m})}{(s-p_{1})(s-p_{2})....(s-p_{n-1})(s-p_{n})}
\end{align}
The gain
\begin{multline}
	G(f) = 20\log\abs{H(s)} 
\\
= 20\log \abs{k} + 20\log \abs{s-z_{1}} 
	    \\
	    + 20\log \abs{s-z_{2}} + \dots + 20\log \abs{s-z_{m}} 
\\
- 20\log \abs{s-p_{1}} 
	    - 20\log \abs{s-p_{2}} 
	    \\
- \dots - 20\log \abs{s-z_{n}} 
\end{multline}
%
Substituting $s = \j \omega$, for real $z_1$
\begin{align}
	20\log \abs{s-z_{1}} &= 20\log \abs{\sqrt{\omega^{2} + z_{1}^{2}}}
\\
&= 
\begin{cases}
20 \log \abs{z_{1}}, & \omega \ll z_{1}
\\
20 \log \abs{\omega}, & \omega \gg z_{1}
\end{cases}
\end{align}
%
Taking the derivative, 
\begin{align}
	\frac{d\brak{20\log \abs{s-z_{1}}}}{d\brak{\log \abs{\omega}}} 
= 
\begin{cases}
0, & \omega \ll z_{1}
\\
20, & \omega \gg z_{1}
\end{cases}
\end{align}
%
Thus, when a zero is encountered, the gradient of $H(\j\omega)$ jumps by +20 in the log scale.  When a pole is encountered, the gradient falls by -20. Note that this is a very loose justification, but works well in practice.

%-----------------------------------------------------------------------%

\item Obtain the Bode plot and the slope plot for $H(s)$ and verify with  Fig. \ref{fig:ee18btech11001_bode}
\\
\solution Bode Plot of obtained Transfer Function is 
\begin{figure}[htp]
    \centering
    \includegraphics[width=\columnwidth]{./figs/ee18btech11001/ee18btech11001_2.eps}
    \caption{}
    \label{fig:ee18btech11001_2}
\end{figure}
%
Fig. \ref{fig:ee18btech11001}, obtained from  \eqref{eq:ee18btech11001_system},
is a close reconstruction of Fig. \ref{fig:ee18btech11001}.
\end{enumerate}



\caption{}
\label{table:ee18btech11001}
\end{table}


%-----------------------------------------------------------------------%

\item Obtain the transfer function of $H(s)$.
\\
\solution From Table \ref{table:ee18btech11001},
{\footnotesize
\begin{align}
\label{eq:ee18btech11001_system}
	H(s) = \frac{K(s+j2\pi 10^{3})(s+j2\pi 10^{4})^{2}}{(s+j2\pi 10^{1})(s+j2\pi 10^{2})^{2}(s+j2\pi 10^{5})^{2}(s+j2\pi 10^{6})}
\end{align}
}
\item Justify the above results.
\\
\solution
Let us consider a generalized transfer gain
\begin{align}
	H(s) = k \dfrac{(s-z_{1})(s-z_{2})...(s-z_{m-1})(s-z_{m})}{(s-p_{1})(s-p_{2})....(s-p_{n-1})(s-p_{n})}
\end{align}
The gain
\begin{multline}
	G(f) = 20\log\abs{H(s)} 
\\
= 20\log \abs{k} + 20\log \abs{s-z_{1}} 
	    \\
	    + 20\log \abs{s-z_{2}} + \dots + 20\log \abs{s-z_{m}} 
\\
- 20\log \abs{s-p_{1}} 
	    - 20\log \abs{s-p_{2}} 
	    \\
- \dots - 20\log \abs{s-z_{n}} 
\end{multline}
%
Substituting $s = \j \omega$, for real $z_1$
\begin{align}
	20\log \abs{s-z_{1}} &= 20\log \abs{\sqrt{\omega^{2} + z_{1}^{2}}}
\\
&= 
\begin{cases}
20 \log \abs{z_{1}}, & \omega \ll z_{1}
\\
20 \log \abs{\omega}, & \omega \gg z_{1}
\end{cases}
\end{align}
%
Taking the derivative, 
\begin{align}
	\frac{d\brak{20\log \abs{s-z_{1}}}}{d\brak{\log \abs{\omega}}} 
= 
\begin{cases}
0, & \omega \ll z_{1}
\\
20, & \omega \gg z_{1}
\end{cases}
\end{align}
%
Thus, when a zero is encountered, the gradient of $H(\j\omega)$ jumps by +20 in the log scale.  When a pole is encountered, the gradient falls by -20. Note that this is a very loose justification, but works well in practice.

%-----------------------------------------------------------------------%

\item Obtain the Bode plot and the slope plot for $H(s)$ and verify with  Fig. \ref{fig:ee18btech11001_bode}
\\
\solution Bode Plot of obtained Transfer Function is 
\begin{figure}[htp]
    \centering
    \includegraphics[width=\columnwidth]{./figs/ee18btech11001/ee18btech11001_2.eps}
    \caption{}
    \label{fig:ee18btech11001_2}
\end{figure}
%
Fig. \ref{fig:ee18btech11001}, obtained from  \eqref{eq:ee18btech11001_system},
is a close reconstruction of Fig. \ref{fig:ee18btech11001}.
\end{enumerate}



%\end{enumerate}

\section{Gain Margin}

\begin{enumerate}[label=\thesection.\arabic*.,ref=\thesection.\theenumi]
\numberwithin{equation}{enumi} 
\item
 
The open loop transfer function of a feedback control system is  
\begin{align}
G(s) = \frac{1}{s(1+2s)(1+s)} 
\end{align}
Find the gain margin of this system and analyse the stability.
 \\


\solution
\\
\textbf{Gain Margin}:The greater the Gain Margin (GM), the greater the stability of the system. The gain margin refers to the amount of gain, which can be increased or decreased without making the system unstable. It is usually expressed as a magnitude in dB.
\\

We can usually read the gain margin directly from the Bode plot. This is done by calculating the vertical distance between the magnitude curve (on the Bode magnitude plot) and the x-axis at the frequency where the Bode phase plot = 180$^{\circ}$. This point is known as the phase crossover frequency.

Gain Margin is given by,
\begin{align}
G.M = -20log_{10}|G(j\omega_{pc})| = 20log_{10}k_{g}
\end{align}
where 
\begin{align}
k_{g}=\frac{1}{|G(j\omega_{pc})|} 
\end{align}
\\

Now let's put s = j$\omega$ in the equation of G(s) :
\begin{align}
G(j\omega) = \frac{1}{j\omega(1+2j\omega)(1+j\omega)} 
\end{align}
So,
\begin{align}
G(j\omega) = \frac{1}{j\omega(1+3j\omega-2\omega^2)}=\frac{1}{j\omega-3\omega^2-2j\omega^3}
\end{align}
Hence ,
\begin{align}
G(j\omega) = \frac{1}{-3\omega^2+j\omega(1-2\omega^2)} 
\end{align}

Now we know that $\omega_{pc}$ is the Phase crossover frequency (The frequency at which the phase of open-loop transfer function reaches -180$^{\circ}$ or +180$^{\circ}$ depending upon the range of tan inverse function).\\
Now,
\begin{align}
\angle G(j\omega)=- tan^{-1}(\frac{\omega(1-2\omega^2)}{-3\omega^2})
\end{align}
So,at $\omega=\omega_{pc}$ :
\begin{align}
\omega(1-2\omega^2) = 0 
\end{align}
i.e. the imaginary part of G(j$\omega$) = 0.So ,
\begin{align}
\omega_{pc} = \frac{1}{\sqrt{2}} 
\end{align}
as $\omega_{pc}$ should be positive and $\omega_{pc}$ shuld not be equal to zero.So now G(j$\omega_{pc}$) will be :
\begin{align}
G(j\omega_{pc}) = \frac{1}{-3\omega_{pc}^2}
\end{align}
i.e,
\begin{align}
|G(j\omega_{pc})| = \frac{1}{(\frac{3}{2})}
\end{align}
\begin{align}
k_{g}=\frac{1}{|G(j\omega_{pc})|} = \frac{3}{2}=1.5
\end{align}
So , Gain margin in terms of dB is :

\begin{align}
20log_{10}1.5 = 3.5dB
\end{align}
\\

Plot obtained for verification in python :

(You can download code from codes/ee18btech11016.py)
\begin{figure}[htp]
	\centering
	\includegraphics[width=\columnwidth,scale=2]{./figs/fig.eps}
	\caption{}
	\label{fig:Phase}
\end{figure}
\\

\section{Stability}

So,in the above figure, since $20log_{10}(G(j\omega_{pc}))$ = -3.5dB at $\omega_{pc} = -180^{\circ}$ so G.M = +3.5dB And since the gain margin is positive we can say that the system is stable more precisely the system is marginally stable as one of the pole lies on the imaginary axis.(Because for stability , both gain and phase margin should be positive.)
\\

\textbf{Analysis}:
Now,Let's try to analyse the stabilty of the system using Routh-Hurwitz Criterion :
\\

Let $T(s)$ be Closed loop transfer function ,
\begin{align}
T(s)=\frac{N(s)}{D(s)}=\frac{G(s)}{1+G(s)}
\end{align}
The characteristic equation is 
\begin{align}
D(s)=0  \\
1+G(s)=0 
\end{align}
So,the closed loop transfer function is given by 
\begin{align}
T(s) = \frac{1}{1 + (s(1+2s)(1+s))}
\end{align}

\begin{align}
=> D(s) = 1 + s(1+s)(1+2s) = 2s^3 + 3s^2 + s + 1 
\end{align}
\\
So,the characteristics equation is given by D(s) = 0.i.e,
\begin{align}
=> 2s^3 + 3s^2 + s + 1 = 0 
\end{align}
Constructing routh array for above equation of D(s),
\begin{align}
\mydet{s^3\\s^2\\s}
\mydet{2 & 1 & 0 \\ 3 & 1 & 0 \\ (1/3) & 0 & 0}
\end{align}\\

\begin{align}
\mydet{s^3\\s^2\\s\\s^0}
\mydet{2 & 1 & 0 \\ 3 & 1 & 0 \\ (1/3) & 0 & 0 \\ 1 & 0 & 0}
\end{align}\\


There are no sign changes in the first column of the routh array. So, no poles lie on right half of s-plane. 
\\
Therefore,the system is stable.\\

You can download the code for Routh-Hurwitz array from (codes/RH.py)
\\

Hence,we can say that from both Routh-hurwitz criterion and from the gain margin concept we are getting the same answers.

\end{enumerate}






\end{document}
